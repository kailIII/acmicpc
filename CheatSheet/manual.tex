%------------------------------------------------------------------------------%
%
%
%
%------------------------------------------------------------------------------%
\documentclass[10pt,letterpaper]{article}

%---------------------------------------------------------------
\usepackage[utf8]{inputenc}
\usepackage[spanish]{babel}
\usepackage{listings}
\usepackage[usenames,dvipsnames]{color}
\usepackage{amsmath}
\usepackage{verbatim}
% \usepackage[colorlinks]{hyperref}
\usepackage{longtable}
%\usepackage{color}
%---------------------------------------------------------------

\setlength{\topmargin}{-1.0in}
\setlength{\textheight}{9.5in} 
\setlength{\evensidemargin}{0.0in}
\setlength{\oddsidemargin}{0.0in}
\setlength{\textwidth}{6.5in} 

\begin{document}

%---------------------------------------------------------------
\title{ACM ICPC Bolivia CheatSheet}
\author{}
\date{}
\maketitle
\newpage
%---------------------------------------------------------------

%---------------------------------------------------------------
\tableofcontents
%\lstlistoflistings
\lstloadlanguages{C++}
%---------------------------------------------------------------
%---------------------------------------------------------------
%\section{Teoría de números}
\section{Matem\'atica}
%---------------------------------------------------------------

\subsection{Karatsuba}
% Generator: GNU source-highlight, by Lorenzo Bettini, http://www.gnu.org/software/src-highlite
{\ttfamily \raggedright {
\noindent
\mbox{}\textit{\textcolor{Brown}{//Mandar\ como\ Parametro\ N\ el\ numero\ de\ bits}} \\
\mbox{}\textbf{\textcolor{RoyalBlue}{import}}\ java\textcolor{BrickRed}{.}math\textcolor{BrickRed}{.}BigInteger\textcolor{BrickRed}{;} \\
\mbox{}\textbf{\textcolor{RoyalBlue}{import}}\ java\textcolor{BrickRed}{.}util\textcolor{BrickRed}{.}Random\textcolor{BrickRed}{;} \\
\mbox{}\textbf{\textcolor{Blue}{class}}\ \textcolor{TealBlue}{Karatsuba}\ \textcolor{Red}{\{} \\
\mbox{}\ \ \ \ \textbf{\textcolor{Blue}{private}}\ \textbf{\textcolor{Blue}{final}}\ \textbf{\textcolor{Blue}{static}}\ \textcolor{TealBlue}{BigInteger}\ ZERO\ \textcolor{BrickRed}{=}\ \textbf{\textcolor{Blue}{new}}\ \textbf{\textcolor{Black}{BigInteger}}\textcolor{BrickRed}{(}\texttt{\textcolor{Red}{"{}0"{}}}\textcolor{BrickRed}{);} \\
\mbox{}\ \ \ \ \textbf{\textcolor{Blue}{public}}\ \textbf{\textcolor{Blue}{static}}\ \textcolor{TealBlue}{BigInteger}\ \textbf{\textcolor{Black}{karatsuba}}\textcolor{BrickRed}{(}\textcolor{TealBlue}{BigInteger}\ x\textcolor{BrickRed}{,}\ \textcolor{TealBlue}{BigInteger}\ y\textcolor{BrickRed}{)}\ \textcolor{Red}{\{} \\
\mbox{}\ \ \ \ \ \ \ \ \\
\mbox{}\ \ \ \ \ \ \ \ \textcolor{ForestGreen}{int}\ N\ \textcolor{BrickRed}{=}\ Math\textcolor{BrickRed}{.}\textbf{\textcolor{Black}{max}}\textcolor{BrickRed}{(}x\textcolor{BrickRed}{.}\textbf{\textcolor{Black}{bitLength}}\textcolor{BrickRed}{(),}\ y\textcolor{BrickRed}{.}\textbf{\textcolor{Black}{bitLength}}\textcolor{BrickRed}{());} \\
\mbox{}\ \ \ \ \ \ \ \ \textbf{\textcolor{Blue}{if}}\ \textcolor{BrickRed}{(}N\ \textcolor{BrickRed}{$<$=}\ \textcolor{Purple}{2000}\textcolor{BrickRed}{)}\ \textbf{\textcolor{Blue}{return}}\ x\textcolor{BrickRed}{.}\textbf{\textcolor{Black}{multiply}}\textcolor{BrickRed}{(}y\textcolor{BrickRed}{);}\ \ \ \ \ \ \ \ \ \ \ \ \ \ \ \ \ \\
\mbox{}\ \ \ \ \ \ \ \ \\
\mbox{}\ \ \ \ \ \ \ \ N\ \textcolor{BrickRed}{=}\ \textcolor{BrickRed}{(}N\ \textcolor{BrickRed}{/}\ \textcolor{Purple}{2}\textcolor{BrickRed}{)}\ \textcolor{BrickRed}{+}\ \textcolor{BrickRed}{(}N\ \textcolor{BrickRed}{\%}\ \textcolor{Purple}{2}\textcolor{BrickRed}{);} \\
\mbox{}\ \ \ \ \ \ \ \ \\
\mbox{}\ \ \ \ \ \ \ \ \textcolor{TealBlue}{BigInteger}\ b\ \textcolor{BrickRed}{=}\ x\textcolor{BrickRed}{.}\textbf{\textcolor{Black}{shiftRight}}\textcolor{BrickRed}{(}N\textcolor{BrickRed}{);} \\
\mbox{}\ \ \ \ \ \ \ \ \textcolor{TealBlue}{BigInteger}\ a\ \textcolor{BrickRed}{=}\ x\textcolor{BrickRed}{.}\textbf{\textcolor{Black}{subtract}}\textcolor{BrickRed}{(}b\textcolor{BrickRed}{.}\textbf{\textcolor{Black}{shiftLeft}}\textcolor{BrickRed}{(}N\textcolor{BrickRed}{));} \\
\mbox{}\ \ \ \ \ \ \ \ \textcolor{TealBlue}{BigInteger}\ d\ \textcolor{BrickRed}{=}\ y\textcolor{BrickRed}{.}\textbf{\textcolor{Black}{shiftRight}}\textcolor{BrickRed}{(}N\textcolor{BrickRed}{);} \\
\mbox{}\ \ \ \ \ \ \ \ \textcolor{TealBlue}{BigInteger}\ c\ \textcolor{BrickRed}{=}\ y\textcolor{BrickRed}{.}\textbf{\textcolor{Black}{subtract}}\textcolor{BrickRed}{(}d\textcolor{BrickRed}{.}\textbf{\textcolor{Black}{shiftLeft}}\textcolor{BrickRed}{(}N\textcolor{BrickRed}{));} \\
\mbox{}\ \ \ \ \ \ \ \ \\
\mbox{}\ \ \ \ \ \ \ \ \textcolor{TealBlue}{BigInteger}\ ac\ \ \ \ \textcolor{BrickRed}{=}\ \textbf{\textcolor{Black}{karatsuba}}\textcolor{BrickRed}{(}a\textcolor{BrickRed}{,}\ c\textcolor{BrickRed}{);} \\
\mbox{}\ \ \ \ \ \ \ \ \textcolor{TealBlue}{BigInteger}\ bd\ \ \ \ \textcolor{BrickRed}{=}\ \textbf{\textcolor{Black}{karatsuba}}\textcolor{BrickRed}{(}b\textcolor{BrickRed}{,}\ d\textcolor{BrickRed}{);} \\
\mbox{}\ \ \ \ \ \ \ \ \textcolor{TealBlue}{BigInteger}\ abcd\ \ \textcolor{BrickRed}{=}\ \textbf{\textcolor{Black}{karatsuba}}\textcolor{BrickRed}{(}a\textcolor{BrickRed}{.}\textbf{\textcolor{Black}{add}}\textcolor{BrickRed}{(}b\textcolor{BrickRed}{),}\ c\textcolor{BrickRed}{.}\textbf{\textcolor{Black}{add}}\textcolor{BrickRed}{(}d\textcolor{BrickRed}{));} \\
\mbox{}\ \ \ \ \ \ \ \ \textbf{\textcolor{Blue}{return}}\ ac\textcolor{BrickRed}{.}\textbf{\textcolor{Black}{add}}\textcolor{BrickRed}{(}abcd\textcolor{BrickRed}{.}\textbf{\textcolor{Black}{subtract}}\textcolor{BrickRed}{(}ac\textcolor{BrickRed}{).}\textbf{\textcolor{Black}{subtract}}\textcolor{BrickRed}{(}bd\textcolor{BrickRed}{).}\textbf{\textcolor{Black}{shiftLeft}}\textcolor{BrickRed}{(}N\textcolor{BrickRed}{)).}\textbf{\textcolor{Black}{add}}\textcolor{BrickRed}{(}bd\textcolor{BrickRed}{.}\textbf{\textcolor{Black}{shiftLeft}}\textcolor{BrickRed}{(}\textcolor{Purple}{2}\textcolor{BrickRed}{*}N\textcolor{BrickRed}{));} \\
\mbox{}\ \ \ \ \textcolor{Red}{\}} \\
\mbox{}\ \ \ \ \textbf{\textcolor{Blue}{public}}\ \textbf{\textcolor{Blue}{static}}\ \textcolor{ForestGreen}{void}\ \textbf{\textcolor{Black}{main}}\textcolor{BrickRed}{(}String\textcolor{BrickRed}{[]}\ args\textcolor{BrickRed}{)}\ \textcolor{Red}{\{} \\
\mbox{}\ \ \ \ \ \ \ \ \textcolor{ForestGreen}{long}\ start\textcolor{BrickRed}{,}\ stop\textcolor{BrickRed}{,}\ elapsed\textcolor{BrickRed}{;} \\
\mbox{}\ \ \ \ \ \ \ \ \textcolor{TealBlue}{Random}\ random\ \textcolor{BrickRed}{=}\ \textbf{\textcolor{Blue}{new}}\ \textbf{\textcolor{Black}{Random}}\textcolor{BrickRed}{();} \\
\mbox{}\ \ \ \ \ \ \ \ \textcolor{ForestGreen}{int}\ N\ \textcolor{BrickRed}{=}\ Integer\textcolor{BrickRed}{.}\textbf{\textcolor{Black}{parseInt}}\textcolor{BrickRed}{(}args\textcolor{BrickRed}{[}\textcolor{Purple}{0}\textcolor{BrickRed}{]);} \\
\mbox{}\ \ \ \ \ \ \ \ \textcolor{TealBlue}{BigInteger}\ a\ \textcolor{BrickRed}{=}\ \textbf{\textcolor{Blue}{new}}\ \textbf{\textcolor{Black}{BigInteger}}\textcolor{BrickRed}{(}N\textcolor{BrickRed}{,}\ random\textcolor{BrickRed}{);} \\
\mbox{}\ \ \ \ \ \ \ \ \textcolor{TealBlue}{BigInteger}\ b\ \textcolor{BrickRed}{=}\ \textbf{\textcolor{Blue}{new}}\ \textbf{\textcolor{Black}{BigInteger}}\textcolor{BrickRed}{(}N\textcolor{BrickRed}{,}\ random\textcolor{BrickRed}{);} \\
\mbox{}\ \ \ \ \ \ \ \ start\ \textcolor{BrickRed}{=}\ System\textcolor{BrickRed}{.}\textbf{\textcolor{Black}{currentTimeMillis}}\textcolor{BrickRed}{();}\  \\
\mbox{}\ \ \ \ \ \ \ \ \textcolor{TealBlue}{BigInteger}\ c\ \textcolor{BrickRed}{=}\ \textbf{\textcolor{Black}{karatsuba}}\textcolor{BrickRed}{(}a\textcolor{BrickRed}{,}\ b\textcolor{BrickRed}{);} \\
\mbox{}\ \ \ \ \ \ \ \ stop\ \textcolor{BrickRed}{=}\ System\textcolor{BrickRed}{.}\textbf{\textcolor{Black}{currentTimeMillis}}\textcolor{BrickRed}{();} \\
\mbox{}\ \ \ \ \ \ \ \ System\textcolor{BrickRed}{.}out\textcolor{BrickRed}{.}\textbf{\textcolor{Black}{println}}\textcolor{BrickRed}{(}stop\ \textcolor{BrickRed}{-}\ start\textcolor{BrickRed}{);} \\
\mbox{}\ \ \ \ \ \ \ \ start\ \textcolor{BrickRed}{=}\ System\textcolor{BrickRed}{.}\textbf{\textcolor{Black}{currentTimeMillis}}\textcolor{BrickRed}{();}\  \\
\mbox{}\ \ \ \ \ \ \ \ \textcolor{TealBlue}{BigInteger}\ d\ \textcolor{BrickRed}{=}\ a\textcolor{BrickRed}{.}\textbf{\textcolor{Black}{multiply}}\textcolor{BrickRed}{(}b\textcolor{BrickRed}{);} \\
\mbox{}\ \ \ \ \ \ \ \ stop\ \textcolor{BrickRed}{=}\ System\textcolor{BrickRed}{.}\textbf{\textcolor{Black}{currentTimeMillis}}\textcolor{BrickRed}{();} \\
\mbox{}\ \ \ \ \ \ \ \ System\textcolor{BrickRed}{.}out\textcolor{BrickRed}{.}\textbf{\textcolor{Black}{println}}\textcolor{BrickRed}{(}stop\ \textcolor{BrickRed}{-}\ start\textcolor{BrickRed}{);} \\
\mbox{}\ \ \ \ \ \ \ \ System\textcolor{BrickRed}{.}out\textcolor{BrickRed}{.}\textbf{\textcolor{Black}{println}}\textcolor{BrickRed}{((}c\textcolor{BrickRed}{.}\textbf{\textcolor{Black}{equals}}\textcolor{BrickRed}{(}d\textcolor{BrickRed}{)));} \\
\mbox{}\ \ \ \ \textcolor{Red}{\}} \\
\mbox{}\textcolor{Red}{\}}
}%.tex

\subsection{Integraci\'on por Simpson}
$$
\int_a^b f(x) dx
$$
% Generator: GNU source-highlight, by Lorenzo Bettini, http://www.gnu.org/software/src-highlite
{\ttfamily \raggedright {
\noindent
\mbox{}\textcolor{ForestGreen}{double}\ a\textcolor{BrickRed}{,}\ b\textcolor{BrickRed}{;}\ \textit{\textcolor{Brown}{//\ limites}} \\
\mbox{}\textbf{\textcolor{Blue}{const}}\ \textcolor{ForestGreen}{int}\ N\ \textcolor{BrickRed}{=}\ \textcolor{Purple}{1000}\textcolor{BrickRed}{*}\textcolor{Purple}{1000}\textcolor{BrickRed}{;} \\
\mbox{}\textcolor{ForestGreen}{double}\ s\ \textcolor{BrickRed}{=}\ \textcolor{Purple}{0}\textcolor{BrickRed}{;} \\
\mbox{}\textbf{\textcolor{Blue}{for}}\ \textcolor{BrickRed}{(}\textcolor{ForestGreen}{int}\ i\textcolor{BrickRed}{=}\textcolor{Purple}{0}\textcolor{BrickRed}{;}\ i\textcolor{BrickRed}{$<$=}N\textcolor{BrickRed}{;}\ \textcolor{BrickRed}{++}i\textcolor{BrickRed}{)}\ \textcolor{Red}{\{} \\
\mbox{}\ \ \ \ \ \ \ \ \textcolor{ForestGreen}{double}\ x\ \textcolor{BrickRed}{=}\ a\ \textcolor{BrickRed}{+}\ \textcolor{BrickRed}{(}b\ \textcolor{BrickRed}{-}\ a\textcolor{BrickRed}{)}\ \textcolor{BrickRed}{*}\ i\ \textcolor{BrickRed}{/}\ N\textcolor{BrickRed}{;} \\
\mbox{}\ \ \ \ \ \ \ \ s\ \textcolor{BrickRed}{+=}\ \textbf{\textcolor{Black}{f}}\textcolor{BrickRed}{(}x\textcolor{BrickRed}{)}\ \textcolor{BrickRed}{*}\ \textcolor{BrickRed}{(}i\textcolor{BrickRed}{==}\textcolor{Purple}{0}\ \textcolor{BrickRed}{$|$$|$}\ i\textcolor{BrickRed}{==}N\ \textcolor{BrickRed}{?}\ \textcolor{Purple}{1}\ \textcolor{BrickRed}{:}\ \textcolor{BrickRed}{(}i\textcolor{BrickRed}{\&}\textcolor{Purple}{1}\textcolor{BrickRed}{)==}\textcolor{Purple}{0}\ \textcolor{BrickRed}{?}\ \textcolor{Purple}{2}\ \textcolor{BrickRed}{:}\ \textcolor{Purple}{4}\textcolor{BrickRed}{);} \\
\mbox{}\textcolor{Red}{\}} \\
\mbox{}\textcolor{ForestGreen}{double}\ delta\ \textcolor{BrickRed}{=}\ \textcolor{BrickRed}{(}b\ \textcolor{BrickRed}{-}\ a\textcolor{BrickRed}{)}\ \textcolor{BrickRed}{/}\ N\textcolor{BrickRed}{;} \\
\mbox{}s\ \textcolor{BrickRed}{*=}\ delta\ \textcolor{BrickRed}{/}\ \textcolor{Purple}{3.0}\textcolor{BrickRed}{;}
}
%.tex

\subsection{Phi de Euler}
% Generator: GNU source-highlight, by Lorenzo Bettini, http://www.gnu.org/software/src-highlite
{\ttfamily \raggedright {
\noindent
\mbox{}\textcolor{ForestGreen}{int}\ \textbf{\textcolor{Black}{phi}}\ \textcolor{BrickRed}{(}\textcolor{ForestGreen}{int}\ n\textcolor{BrickRed}{)}\ \textcolor{Red}{\{} \\
\mbox{}\ \ \ \ \ \ \ \ \textcolor{ForestGreen}{int}\ result\ \textcolor{BrickRed}{=}\ n\textcolor{BrickRed}{;} \\
\mbox{}\ \ \ \ \ \ \ \ \textbf{\textcolor{Blue}{for}}\ \textcolor{BrickRed}{(}\textcolor{ForestGreen}{int}\ i\textcolor{BrickRed}{=}\textcolor{Purple}{2}\textcolor{BrickRed}{;}\ i\textcolor{BrickRed}{*}i\textcolor{BrickRed}{$<$=}n\textcolor{BrickRed}{;}\ \textcolor{BrickRed}{++}i\textcolor{BrickRed}{)} \\
\mbox{}\ \ \ \ \ \ \ \ \ \ \ \ \ \ \ \ \textbf{\textcolor{Blue}{if}}\ \textcolor{BrickRed}{(}n\ \textcolor{BrickRed}{\%}\ i\ \textcolor{BrickRed}{==}\ \textcolor{Purple}{0}\textcolor{BrickRed}{)}\ \textcolor{Red}{\{} \\
\mbox{}\ \ \ \ \ \ \ \ \ \ \ \ \ \ \ \ \ \ \ \ \ \ \ \ \textbf{\textcolor{Blue}{while}}\ \textcolor{BrickRed}{(}n\ \textcolor{BrickRed}{\%}\ i\ \textcolor{BrickRed}{==}\ \textcolor{Purple}{0}\textcolor{BrickRed}{)} \\
\mbox{}\ \ \ \ \ \ \ \ \ \ \ \ \ \ \ \ \ \ \ \ \ \ \ \ \ \ \ \ \ \ \ \ n\ \textcolor{BrickRed}{/=}\ i\textcolor{BrickRed}{;} \\
\mbox{}\ \ \ \ \ \ \ \ \ \ \ \ \ \ \ \ \ \ \ \ \ \ \ \ result\ \textcolor{BrickRed}{-=}\ result\ \textcolor{BrickRed}{/}\ i\textcolor{BrickRed}{;} \\
\mbox{}\ \ \ \ \ \ \ \ \ \ \ \ \ \ \ \ \textcolor{Red}{\}} \\
\mbox{}\ \ \ \ \ \ \ \ \textbf{\textcolor{Blue}{if}}\ \textcolor{BrickRed}{(}n\ \textcolor{BrickRed}{$>$}\ \textcolor{Purple}{1}\textcolor{BrickRed}{)} \\
\mbox{}\ \ \ \ \ \ \ \ \ \ \ \ \ \ \ \ result\ \textcolor{BrickRed}{-=}\ result\ \textcolor{BrickRed}{/}\ n\textcolor{BrickRed}{;} \\
\mbox{}\ \ \ \ \ \ \ \ \textbf{\textcolor{Blue}{return}}\ result\textcolor{BrickRed}{;} \\
\mbox{}\textcolor{Red}{\}} \\
\mbox{}
}%.tex

\subsection{Modulo en Factorial}
% Generator: GNU source-highlight, by Lorenzo Bettini, http://www.gnu.org/software/src-highlite
{\ttfamily \raggedright {
\noindent
\mbox{}\textit{\textcolor{Brown}{//n!\ mod\ p}} \\
\mbox{}\textcolor{ForestGreen}{int}\ \textbf{\textcolor{Black}{factmod}}\ \textcolor{BrickRed}{(}\textcolor{ForestGreen}{int}\ n\textcolor{BrickRed}{,}\ \textcolor{ForestGreen}{int}\ p\textcolor{BrickRed}{)}\ \textcolor{Red}{\{} \\
\mbox{}\ \ \ \ \ \ \ \ \textcolor{ForestGreen}{long}\ \textcolor{ForestGreen}{long}\ res\ \textcolor{BrickRed}{=}\ \textcolor{Purple}{1}\textcolor{BrickRed}{;} \\
\mbox{}\ \ \ \ \ \ \ \ \textbf{\textcolor{Blue}{while}}\ \textcolor{BrickRed}{(}n\ \textcolor{BrickRed}{$>$}\ \textcolor{Purple}{1}\textcolor{BrickRed}{)}\ \textcolor{Red}{\{} \\
\mbox{}\ \ \ \ \ \ \ \ \ \ \ \ \ \ \ \ res\ \textcolor{BrickRed}{=}\ \textcolor{BrickRed}{(}res\ \textcolor{BrickRed}{*}\ \textbf{\textcolor{Black}{powmod}}\ \textcolor{BrickRed}{(}p\textcolor{BrickRed}{-}\textcolor{Purple}{1}\textcolor{BrickRed}{,}\ n\textcolor{BrickRed}{/}p\textcolor{BrickRed}{,}\ p\textcolor{BrickRed}{))}\ \textcolor{BrickRed}{\%}\ p\textcolor{BrickRed}{;} \\
\mbox{}\ \ \ \ \ \ \ \ \ \ \ \ \ \ \ \ \textbf{\textcolor{Blue}{for}}\ \textcolor{BrickRed}{(}\textcolor{ForestGreen}{int}\ i\textcolor{BrickRed}{=}\textcolor{Purple}{2}\textcolor{BrickRed}{;}\ i\textcolor{BrickRed}{$<$=}n\textcolor{BrickRed}{\%}p\textcolor{BrickRed}{;}\ \textcolor{BrickRed}{++}i\textcolor{BrickRed}{)} \\
\mbox{}\ \ \ \ \ \ \ \ \ \ \ \ \ \ \ \ \ \ \ \ \ \ \ \ res\ \textcolor{BrickRed}{=}\ \textcolor{BrickRed}{(}res\ \textcolor{BrickRed}{*}\ i\textcolor{BrickRed}{)}\ \textcolor{BrickRed}{\%}\ p\textcolor{BrickRed}{;} \\
\mbox{}\ \ \ \ \ \ \ \ \ \ \ \ \ \ \ \ n\ \textcolor{BrickRed}{/=}\ p\textcolor{BrickRed}{;} \\
\mbox{}\ \ \ \ \ \ \ \ \textcolor{Red}{\}} \\
\mbox{}\ \ \ \ \ \ \ \ \textbf{\textcolor{Blue}{return}}\ \textcolor{ForestGreen}{int}\ \textcolor{BrickRed}{(}res\ \textcolor{BrickRed}{\%}\ p\textcolor{BrickRed}{);} \\
\mbox{}\textcolor{Red}{\}}
}%.tex

\subsection{Exponenciaci\'on Binaria}
% Generator: GNU source-highlight, by Lorenzo Bettini, http://www.gnu.org/software/src-highlite
{\ttfamily \raggedright {
\noindent
\mbox{}\textcolor{ForestGreen}{int}\ \textbf{\textcolor{Black}{binpow}}\ \textcolor{BrickRed}{(}\textcolor{ForestGreen}{int}\ a\textcolor{BrickRed}{,}\ \textcolor{ForestGreen}{int}\ n\textcolor{BrickRed}{)}\ \textcolor{Red}{\{} \\
\mbox{}\ \ \ \ \ \ \ \ \textcolor{ForestGreen}{int}\ res\ \textcolor{BrickRed}{=}\ \textcolor{Purple}{1}\textcolor{BrickRed}{;} \\
\mbox{}\ \ \ \ \ \ \ \ \textbf{\textcolor{Blue}{while}}\ \textcolor{BrickRed}{(}n\textcolor{BrickRed}{)} \\
\mbox{}\ \ \ \ \ \ \ \ \ \ \ \ \ \ \ \ \textbf{\textcolor{Blue}{if}}\ \textcolor{BrickRed}{(}n\ \textcolor{BrickRed}{\&}\ \textcolor{Purple}{1}\textcolor{BrickRed}{)}\ \textcolor{Red}{\{} \\
\mbox{}\ \ \ \ \ \ \ \ \ \ \ \ \ \ \ \ \ \ \ \ \ \ \ \ res\ \textcolor{BrickRed}{*=}\ a\textcolor{BrickRed}{;} \\
\mbox{}\ \ \ \ \ \ \ \ \ \ \ \ \ \ \ \ \ \ \ \ \ \ \ \ \textcolor{BrickRed}{-\/-}n\textcolor{BrickRed}{;} \\
\mbox{}\ \ \ \ \ \ \ \ \ \ \ \ \ \ \ \ \textcolor{Red}{\}} \\
\mbox{}\ \ \ \ \ \ \ \ \ \ \ \ \ \ \ \ \textbf{\textcolor{Blue}{else}}\ \textcolor{Red}{\{} \\
\mbox{}\ \ \ \ \ \ \ \ \ \ \ \ \ \ \ \ \ \ \ \ \ \ \ \ a\ \textcolor{BrickRed}{*=}\ a\textcolor{BrickRed}{;} \\
\mbox{}\ \ \ \ \ \ \ \ \ \ \ \ \ \ \ \ \ \ \ \ \ \ \ \ n\ \textcolor{BrickRed}{$>$$>$=}\ \textcolor{Purple}{1}\textcolor{BrickRed}{;} \\
\mbox{}\ \ \ \ \ \ \ \ \ \ \ \ \ \ \ \ \textcolor{Red}{\}} \\
\mbox{}\ \ \ \ \ \ \ \ \textbf{\textcolor{Blue}{return}}\ res\textcolor{BrickRed}{;} \\
\mbox{}\textcolor{Red}{\}}
}%.tex

\section{Grafos}
\subsection{Ordenamiento Topologico}
{\ttfamily \raggedright {
% Generator: GNU source-highlight, by Lorenzo Bettini, http://www.gnu.org/software/src-highlite
\noindent
\mbox{}vector\ \textcolor{BrickRed}{$<$}\ \textcolor{TealBlue}{vector$<$int$>$\ $>$}\ g\textcolor{BrickRed}{;} \\
\mbox{}\textcolor{ForestGreen}{int}\ n\textcolor{BrickRed}{;} \\
\mbox{} \\
\mbox{}\textcolor{TealBlue}{vector$<$bool$>$}\ used\textcolor{BrickRed}{;} \\
\mbox{} \\
\mbox{}\textcolor{TealBlue}{list$<$int$>$}\ ans\textcolor{BrickRed}{;} \\
\mbox{} \\
\mbox{}\textcolor{ForestGreen}{void}\ \textbf{\textcolor{Black}{dfs}}\textcolor{BrickRed}{(}\textcolor{ForestGreen}{int}\ v\textcolor{BrickRed}{)} \\
\mbox{}\textcolor{Red}{\{} \\
\mbox{}\ \ used\textcolor{BrickRed}{[}v\textcolor{BrickRed}{]}\ \textcolor{BrickRed}{=}\ \textbf{\textcolor{Blue}{true}}\textcolor{BrickRed}{;} \\
\mbox{}\ \ \textbf{\textcolor{Blue}{for}}\textcolor{BrickRed}{(}vector\textcolor{BrickRed}{$<$}\textcolor{ForestGreen}{int}\textcolor{BrickRed}{$>$::}\textcolor{TealBlue}{itetator}\ i\textcolor{BrickRed}{=}g\textcolor{BrickRed}{[}v\textcolor{BrickRed}{].}\textbf{\textcolor{Black}{begin}}\textcolor{BrickRed}{();}\ i\textcolor{BrickRed}{!=}g\textcolor{BrickRed}{[}v\textcolor{BrickRed}{].}\textbf{\textcolor{Black}{end}}\textcolor{BrickRed}{();}\ \textcolor{BrickRed}{++}i\textcolor{BrickRed}{)} \\
\mbox{}\ \ \ \ \textbf{\textcolor{Blue}{if}}\textcolor{BrickRed}{(!}used\textcolor{BrickRed}{[*}i\textcolor{BrickRed}{])} \\
\mbox{}\ \ \ \ \ \ \textbf{\textcolor{Black}{dfs}}\textcolor{BrickRed}{(*}i\textcolor{BrickRed}{);} \\
\mbox{}\ \ ans\textcolor{BrickRed}{.}\textbf{\textcolor{Black}{push$\_$front}}\textcolor{BrickRed}{(}v\textcolor{BrickRed}{);} \\
\mbox{}\textcolor{Red}{\}} \\
\mbox{} \\
\mbox{}\textcolor{ForestGreen}{void}\ \textbf{\textcolor{Black}{topological$\_$sort}}\textcolor{BrickRed}{(}list\textcolor{BrickRed}{$<$}\textcolor{ForestGreen}{int}\textcolor{BrickRed}{$>$}\ \textcolor{BrickRed}{\&}\ result\textcolor{BrickRed}{)} \\
\mbox{}\textcolor{Red}{\{} \\
\mbox{}\ \ used\textcolor{BrickRed}{.}\textbf{\textcolor{Black}{assign}}\textcolor{BrickRed}{(}n\textcolor{BrickRed}{,}\ \textbf{\textcolor{Blue}{false}}\textcolor{BrickRed}{);} \\
\mbox{}\ \ \textbf{\textcolor{Blue}{for}}\textcolor{BrickRed}{(}\textcolor{ForestGreen}{int}\ i\textcolor{BrickRed}{=}\textcolor{Purple}{0}\textcolor{BrickRed}{;}\ i\textcolor{BrickRed}{$<$}n\textcolor{BrickRed}{;}\ \textcolor{BrickRed}{++}i\textcolor{BrickRed}{)} \\
\mbox{}\ \ \ \ \textbf{\textcolor{Blue}{if}}\textcolor{BrickRed}{(!}used\textcolor{BrickRed}{[}i\textcolor{BrickRed}{])} \\
\mbox{}\ \ \ \ \ \ \textbf{\textcolor{Black}{dfs}}\textcolor{BrickRed}{(}i\textcolor{BrickRed}{);} \\
\mbox{}\ \ result\ \textcolor{BrickRed}{=}\ ans\textcolor{BrickRed}{;} \\
\mbox{}\textcolor{Red}{\}} \\
\mbox{}
} \normalfont\normalsize
%.tex

\subsection{Componentes fuertemente conectados}
{\ttfamily \raggedright {
% Generator: GNU source-highlight, by Lorenzo Bettini, http://www.gnu.org/software/src-highlite
\noindent
\mbox{}vector\ \textcolor{BrickRed}{$<$}\ \textcolor{TealBlue}{vector$<$int$>$\ $>$}\ g\textcolor{BrickRed}{,}\ gr\textcolor{BrickRed}{;} \\
\mbox{}\textcolor{TealBlue}{vector$<$char$>$}\ used\textcolor{BrickRed}{;} \\
\mbox{}\textcolor{TealBlue}{vector$<$int$>$}\ order\textcolor{BrickRed}{,}\ component\textcolor{BrickRed}{;} \\
\mbox{}\  \\
\mbox{}\textcolor{ForestGreen}{void}\ \textbf{\textcolor{Black}{dfs1}}\textcolor{BrickRed}{(}\textcolor{ForestGreen}{int}\ v\textcolor{BrickRed}{)}\ \textcolor{Red}{\{} \\
\mbox{}\ \ used\textcolor{BrickRed}{[}v\textcolor{BrickRed}{]}\ \textcolor{BrickRed}{=}\ \textbf{\textcolor{Blue}{true}}\textcolor{BrickRed}{;} \\
\mbox{}\ \ \textbf{\textcolor{Blue}{for}}\textcolor{BrickRed}{(}\textcolor{TealBlue}{size$\_$t}\ i\textcolor{BrickRed}{=}\textcolor{Purple}{0}\textcolor{BrickRed}{;}\ i\textcolor{BrickRed}{$<$}g\textcolor{BrickRed}{[}v\textcolor{BrickRed}{].}\textbf{\textcolor{Black}{size}}\textcolor{BrickRed}{();}\ \textcolor{BrickRed}{++}i\textcolor{BrickRed}{)} \\
\mbox{}\ \ \ \ \textbf{\textcolor{Blue}{if}}\textcolor{BrickRed}{(!}used\textcolor{BrickRed}{[}\ g\textcolor{BrickRed}{[}v\textcolor{BrickRed}{][}i\textcolor{BrickRed}{]}\ \textcolor{BrickRed}{])} \\
\mbox{}\ \ \ \ \ \ \textbf{\textcolor{Black}{dfs1}}\textcolor{BrickRed}{(}g\textcolor{BrickRed}{[}v\textcolor{BrickRed}{][}i\textcolor{BrickRed}{]);} \\
\mbox{}\ \ order\textcolor{BrickRed}{.}\textbf{\textcolor{Black}{push$\_$back}}\textcolor{BrickRed}{(}v\textcolor{BrickRed}{);} \\
\mbox{}\textcolor{Red}{\}} \\
\mbox{}\  \\
\mbox{}\textcolor{ForestGreen}{void}\ \textbf{\textcolor{Black}{dfs2}}\textcolor{BrickRed}{(}\textcolor{ForestGreen}{int}\ v\textcolor{BrickRed}{)}\textcolor{Red}{\{} \\
\mbox{}\ \ used\textcolor{BrickRed}{[}v\textcolor{BrickRed}{]}\ \textcolor{BrickRed}{=}\ \textbf{\textcolor{Blue}{true}}\textcolor{BrickRed}{;} \\
\mbox{}\ \ component\textcolor{BrickRed}{.}\textbf{\textcolor{Black}{push$\_$back}}\ \textcolor{BrickRed}{(}v\textcolor{BrickRed}{);} \\
\mbox{}\ \ \textbf{\textcolor{Blue}{for}}\textcolor{BrickRed}{(}\textcolor{TealBlue}{size$\_$t}\ i\textcolor{BrickRed}{=}\textcolor{Purple}{0}\textcolor{BrickRed}{;}\ i\textcolor{BrickRed}{$<$}gr\textcolor{BrickRed}{[}v\textcolor{BrickRed}{].}\textbf{\textcolor{Black}{size}}\textcolor{BrickRed}{();}\ \textcolor{BrickRed}{++}i\textcolor{BrickRed}{)} \\
\mbox{}\ \ \ \ \textbf{\textcolor{Blue}{if}}\textcolor{BrickRed}{(!}used\textcolor{BrickRed}{[}\ gr\textcolor{BrickRed}{[}v\textcolor{BrickRed}{][}i\textcolor{BrickRed}{]}\ \textcolor{BrickRed}{])} \\
\mbox{}\ \ \ \ \ \ \textbf{\textcolor{Black}{dfs2}}\textcolor{BrickRed}{(}gr\textcolor{BrickRed}{[}v\textcolor{BrickRed}{][}i\textcolor{BrickRed}{]);} \\
\mbox{}\textcolor{Red}{\}} \\
\mbox{}\  \\
\mbox{}\textcolor{ForestGreen}{int}\ \textbf{\textcolor{Black}{main}}\textcolor{BrickRed}{()}\ \textcolor{Red}{\{} \\
\mbox{}\ \ \textcolor{ForestGreen}{int}\ n\textcolor{BrickRed}{;} \\
\mbox{}\ \ \textit{\textcolor{Brown}{//...\ read\ n\ ...}} \\
\mbox{}\ \ \textbf{\textcolor{Blue}{for}}\textcolor{BrickRed}{(;;)}\ \textcolor{Red}{\{} \\
\mbox{}\ \ \ \ \textcolor{ForestGreen}{int}\ a\textcolor{BrickRed}{,}\ b\textcolor{BrickRed}{;} \\
\mbox{}\ \ \ \ \textit{\textcolor{Brown}{//...\ read\ directed\ edge\ (a,b)\ ...}} \\
\mbox{}\ \ \ \ g\textcolor{BrickRed}{[}a\textcolor{BrickRed}{].}\textbf{\textcolor{Black}{push$\_$back}}\textcolor{BrickRed}{(}b\textcolor{BrickRed}{);} \\
\mbox{}\ \ \ \ gr\textcolor{BrickRed}{[}b\textcolor{BrickRed}{].}\textbf{\textcolor{Black}{push$\_$back}}\textcolor{BrickRed}{(}a\textcolor{BrickRed}{);} \\
\mbox{}\ \ \textcolor{Red}{\}} \\
\mbox{}\  \\
\mbox{}\ \ used\textcolor{BrickRed}{.}\textbf{\textcolor{Black}{assign}}\textcolor{BrickRed}{(}n\textcolor{BrickRed}{,}\ \textbf{\textcolor{Blue}{false}}\textcolor{BrickRed}{);} \\
\mbox{}\ \ \textbf{\textcolor{Blue}{for}}\textcolor{BrickRed}{(}\textcolor{ForestGreen}{int}\ i\textcolor{BrickRed}{=}\textcolor{Purple}{0}\textcolor{BrickRed}{;}\ i\textcolor{BrickRed}{$<$}n\textcolor{BrickRed}{;}\ \textcolor{BrickRed}{++}i\textcolor{BrickRed}{)} \\
\mbox{}\ \ \ \ \textbf{\textcolor{Blue}{if}}\textcolor{BrickRed}{(!}used\textcolor{BrickRed}{[}i\textcolor{BrickRed}{])} \\
\mbox{}\ \ \ \ \ \ \textbf{\textcolor{Black}{dfs1}}\textcolor{BrickRed}{(}i\textcolor{BrickRed}{);} \\
\mbox{}\ \ used\textcolor{BrickRed}{.}\textbf{\textcolor{Black}{assign}}\textcolor{BrickRed}{(}n\textcolor{BrickRed}{,}\ \textbf{\textcolor{Blue}{false}}\textcolor{BrickRed}{);} \\
\mbox{}\ \ \textbf{\textcolor{Blue}{for}}\textcolor{BrickRed}{(}\textcolor{ForestGreen}{int}\ i\textcolor{BrickRed}{=}\textcolor{Purple}{0}\textcolor{BrickRed}{;}\ i\textcolor{BrickRed}{$<$}n\textcolor{BrickRed}{;}\ \textcolor{BrickRed}{++}i\textcolor{BrickRed}{)}\ \textcolor{Red}{\{} \\
\mbox{}\ \ \ \ \textcolor{ForestGreen}{int}\ v\ \textcolor{BrickRed}{=}\ order\textcolor{BrickRed}{[}n\textcolor{BrickRed}{-}\textcolor{Purple}{1}\textcolor{BrickRed}{-}i\textcolor{BrickRed}{];} \\
\mbox{}\ \ \ \ \textbf{\textcolor{Blue}{if}}\textcolor{BrickRed}{(!}used\textcolor{BrickRed}{[}v\textcolor{BrickRed}{])}\ \textcolor{Red}{\{} \\
\mbox{}\ \ \ \ \ \ \textbf{\textcolor{Black}{dfs2}}\textcolor{BrickRed}{(}v\textcolor{BrickRed}{);} \\
\mbox{}\ \ \ \ \ \ \textit{\textcolor{Brown}{//...\ work\ with\ component\ ...}} \\
\mbox{}\ \ \ \ \ \ component\textcolor{BrickRed}{.}\textbf{\textcolor{Black}{clear}}\textcolor{BrickRed}{();} \\
\mbox{}\ \ \ \ \textcolor{Red}{\}} \\
\mbox{}\ \ \textcolor{Red}{\}} \\
\mbox{}\textcolor{Red}{\}} \\
\mbox{} \\
\mbox{}
} \normalfont\normalsize
%.tex

\subsection{K camino mas corto}
{\ttfamily \raggedright {
% Generator: GNU source-highlight, by Lorenzo Bettini, http://www.gnu.org/software/src-highlite
\noindent
\mbox{}\textbf{\textcolor{Blue}{const}}\ \textcolor{ForestGreen}{int}\ INF\ \textcolor{BrickRed}{=}\ \textcolor{Purple}{1000}\textcolor{BrickRed}{*}\textcolor{Purple}{1000}\textcolor{BrickRed}{*}\textcolor{Purple}{1000}\textcolor{BrickRed}{;} \\
\mbox{}\textbf{\textcolor{Blue}{const}}\ \textcolor{ForestGreen}{int}\ W\ \textcolor{BrickRed}{=}\ \textcolor{BrickRed}{...;}\ \textit{\textcolor{Brown}{//\ peso\ maximo}} \\
\mbox{} \\
\mbox{}\textcolor{ForestGreen}{int}\ n\textcolor{BrickRed}{,}\ s\textcolor{BrickRed}{,}\ t\textcolor{BrickRed}{;} \\
\mbox{}vector\ \textcolor{BrickRed}{$<$}\ vector\ \textcolor{BrickRed}{$<$}\ \textcolor{TealBlue}{pair$<$int,int$>$\ $>$\ $>$}\ g\textcolor{BrickRed}{;} \\
\mbox{}\textcolor{TealBlue}{vector$<$int$>$}\ dist\textcolor{BrickRed}{;} \\
\mbox{}\textcolor{TealBlue}{vector$<$char$>$}\ used\textcolor{BrickRed}{;} \\
\mbox{}\textcolor{TealBlue}{vector$<$int$>$}\ curpath\textcolor{BrickRed}{,}\ kth$\_$path\textcolor{BrickRed}{;} \\
\mbox{} \\
\mbox{}\textcolor{ForestGreen}{int}\ \textbf{\textcolor{Black}{kth$\_$path$\_$exists}}\textcolor{BrickRed}{(}\textcolor{ForestGreen}{int}\ k\textcolor{BrickRed}{,}\ \textcolor{ForestGreen}{int}\ maxlen\textcolor{BrickRed}{,}\ \textcolor{ForestGreen}{int}\ v\textcolor{BrickRed}{,}\ \textcolor{ForestGreen}{int}\ curlen\ \textcolor{BrickRed}{=}\ \textcolor{Purple}{0}\textcolor{BrickRed}{)}\ \textcolor{Red}{\{} \\
\mbox{}\ \ curpath\textcolor{BrickRed}{.}\textbf{\textcolor{Black}{push$\_$back}}\textcolor{BrickRed}{(}v\textcolor{BrickRed}{);} \\
\mbox{}\ \ \textbf{\textcolor{Blue}{if}}\textcolor{BrickRed}{(}v\ \textcolor{BrickRed}{==}\ t\textcolor{BrickRed}{)}\ \textcolor{Red}{\{} \\
\mbox{}\ \ \ \ \textbf{\textcolor{Blue}{if}}\textcolor{BrickRed}{(}curlen\ \textcolor{BrickRed}{==}\ maxlen\textcolor{BrickRed}{)} \\
\mbox{}\ \ \ \ \ \ kth$\_$path\ \textcolor{BrickRed}{=}\ curpath\textcolor{BrickRed}{;} \\
\mbox{}\ \ \ \ curpath\textcolor{BrickRed}{.}\textbf{\textcolor{Black}{pop$\_$back}}\textcolor{BrickRed}{();} \\
\mbox{}\ \ \ \ \textbf{\textcolor{Blue}{return}}\ \textcolor{Purple}{1}\textcolor{BrickRed}{;} \\
\mbox{}\ \ \textcolor{Red}{\}} \\
\mbox{}\ \ used\textcolor{BrickRed}{[}v\textcolor{BrickRed}{]}\ \textcolor{BrickRed}{=}\ \textbf{\textcolor{Blue}{true}}\textcolor{BrickRed}{;} \\
\mbox{}\ \ \textcolor{ForestGreen}{int}\ found\ \textcolor{BrickRed}{=}\ \textcolor{Purple}{0}\textcolor{BrickRed}{;} \\
\mbox{}\ \ \textbf{\textcolor{Blue}{for}}\textcolor{BrickRed}{(}\textcolor{TealBlue}{size$\_$t}\ i\textcolor{BrickRed}{=}\textcolor{Purple}{0}\textcolor{BrickRed}{;}\ i\textcolor{BrickRed}{$<$}g\textcolor{BrickRed}{[}v\textcolor{BrickRed}{].}\textbf{\textcolor{Black}{size}}\textcolor{BrickRed}{();}\ \textcolor{BrickRed}{++}i\textcolor{BrickRed}{)}\ \textcolor{Red}{\{} \\
\mbox{}\ \ \ \ \textcolor{ForestGreen}{int}\ to\ \textcolor{BrickRed}{=}\ g\textcolor{BrickRed}{[}v\textcolor{BrickRed}{][}i\textcolor{BrickRed}{].}first\textcolor{BrickRed}{,}\ \ len\ \textcolor{BrickRed}{=}\ g\textcolor{BrickRed}{[}v\textcolor{BrickRed}{][}i\textcolor{BrickRed}{].}second\textcolor{BrickRed}{;} \\
\mbox{}\ \ \ \ \textbf{\textcolor{Blue}{if}}\textcolor{BrickRed}{(!}used\textcolor{BrickRed}{[}to\textcolor{BrickRed}{]}\ \textcolor{BrickRed}{\&\&}\ curlen\ \textcolor{BrickRed}{+}\ len\ \textcolor{BrickRed}{+}\ dist\textcolor{BrickRed}{[}to\textcolor{BrickRed}{]}\ \textcolor{BrickRed}{$<$=}\ maxlen\textcolor{BrickRed}{)}\ \textcolor{Red}{\{} \\
\mbox{}\ \ \ \ \ \ found\ \textcolor{BrickRed}{+=}\ \textbf{\textcolor{Black}{kth$\_$path$\_$exists}}\textcolor{BrickRed}{(}k\ \textcolor{BrickRed}{-}\ found\textcolor{BrickRed}{,}\ maxlen\textcolor{BrickRed}{,}\ to\textcolor{BrickRed}{,}\ curlen\ \textcolor{BrickRed}{+}\ len\textcolor{BrickRed}{);} \\
\mbox{}\ \ \ \ \ \ \textbf{\textcolor{Blue}{if}}\textcolor{BrickRed}{(}found\ \textcolor{BrickRed}{==}\ k\textcolor{BrickRed}{)}\ \ \textbf{\textcolor{Blue}{break}}\textcolor{BrickRed}{;} \\
\mbox{}\ \ \ \ \textcolor{Red}{\}} \\
\mbox{}\ \ \textcolor{Red}{\}} \\
\mbox{}\ \ used\textcolor{BrickRed}{[}v\textcolor{BrickRed}{]}\ \textcolor{BrickRed}{=}\ \textbf{\textcolor{Blue}{false}}\textcolor{BrickRed}{;} \\
\mbox{}\ \ curpath\textcolor{BrickRed}{.}\textbf{\textcolor{Black}{pop$\_$back}}\textcolor{BrickRed}{();} \\
\mbox{}\ \ \textbf{\textcolor{Blue}{return}}\ found\textcolor{BrickRed}{;} \\
\mbox{}\textcolor{Red}{\}} \\
\mbox{} \\
\mbox{} \\
\mbox{}\textcolor{ForestGreen}{int}\ \textbf{\textcolor{Black}{main}}\textcolor{BrickRed}{()}\ \textcolor{Red}{\{} \\
\mbox{} \\
\mbox{}\ \ \textit{\textcolor{Brown}{//...\ inicializar\ (n,\ k,\ g,\ s,\ t)\ ...}} \\
\mbox{} \\
\mbox{}\ \ dist\textcolor{BrickRed}{.}\textbf{\textcolor{Black}{assign}}\textcolor{BrickRed}{(}n\textcolor{BrickRed}{,}\ INF\textcolor{BrickRed}{);} \\
\mbox{}\ \ dist\textcolor{BrickRed}{[}t\textcolor{BrickRed}{]}\ \textcolor{BrickRed}{=}\ \textcolor{Purple}{0}\textcolor{BrickRed}{;} \\
\mbox{}\ \ used\textcolor{BrickRed}{.}\textbf{\textcolor{Black}{assign}}\textcolor{BrickRed}{(}n\textcolor{BrickRed}{,}\ \textbf{\textcolor{Blue}{false}}\textcolor{BrickRed}{);} \\
\mbox{}\ \ \textbf{\textcolor{Blue}{for}}\textcolor{BrickRed}{(;;)}\ \textcolor{Red}{\{} \\
\mbox{}\ \ \ \ \textcolor{ForestGreen}{int}\ sel\ \textcolor{BrickRed}{=}\ \textcolor{BrickRed}{-}\textcolor{Purple}{1}\textcolor{BrickRed}{;} \\
\mbox{}\ \ \ \ \textbf{\textcolor{Blue}{for}}\textcolor{BrickRed}{(}\textcolor{ForestGreen}{int}\ i\textcolor{BrickRed}{=}\textcolor{Purple}{0}\textcolor{BrickRed}{;}\ i\textcolor{BrickRed}{$<$}n\textcolor{BrickRed}{;}\ \textcolor{BrickRed}{++}i\textcolor{BrickRed}{)} \\
\mbox{}\ \ \ \ \ \ \textbf{\textcolor{Blue}{if}}\textcolor{BrickRed}{(!}used\textcolor{BrickRed}{[}i\textcolor{BrickRed}{]}\ \textcolor{BrickRed}{\&\&}\ dist\textcolor{BrickRed}{[}i\textcolor{BrickRed}{]}\ \textcolor{BrickRed}{$<$}\ INF\ \textcolor{BrickRed}{\&\&}\ \textcolor{BrickRed}{(}sel\ \textcolor{BrickRed}{==}\ \textcolor{BrickRed}{-}\textcolor{Purple}{1}\ \textcolor{BrickRed}{$|$$|$}\ dist\textcolor{BrickRed}{[}i\textcolor{BrickRed}{]}\ \textcolor{BrickRed}{$<$}\ dist\textcolor{BrickRed}{[}sel\textcolor{BrickRed}{]))} \\
\mbox{}\ \ \ \ \ \ \ \ sel\ \textcolor{BrickRed}{=}\ i\textcolor{BrickRed}{;} \\
\mbox{}\ \ \ \ \textbf{\textcolor{Blue}{if}}\textcolor{BrickRed}{(}sel\ \textcolor{BrickRed}{==}\ \textcolor{BrickRed}{-}\textcolor{Purple}{1}\textcolor{BrickRed}{)}\ \ \textbf{\textcolor{Blue}{break}}\textcolor{BrickRed}{;} \\
\mbox{}\ \ \ \ used\textcolor{BrickRed}{[}sel\textcolor{BrickRed}{]}\ \textcolor{BrickRed}{=}\ \textbf{\textcolor{Blue}{true}}\textcolor{BrickRed}{;} \\
\mbox{}\ \ \ \ \textbf{\textcolor{Blue}{for}}\textcolor{BrickRed}{(}\textcolor{TealBlue}{size$\_$t}\ i\textcolor{BrickRed}{=}\textcolor{Purple}{0}\textcolor{BrickRed}{;}\ i\textcolor{BrickRed}{$<$}g\textcolor{BrickRed}{[}sel\textcolor{BrickRed}{].}\textbf{\textcolor{Black}{size}}\textcolor{BrickRed}{();}\ \textcolor{BrickRed}{++}i\textcolor{BrickRed}{)}\ \textcolor{Red}{\{} \\
\mbox{}\ \ \ \ \ \ \textcolor{ForestGreen}{int}\ to\ \textcolor{BrickRed}{=}\ g\textcolor{BrickRed}{[}sel\textcolor{BrickRed}{][}i\textcolor{BrickRed}{].}first\textcolor{BrickRed}{,}\ \ len\ \textcolor{BrickRed}{=}\ g\textcolor{BrickRed}{[}sel\textcolor{BrickRed}{][}i\textcolor{BrickRed}{].}second\textcolor{BrickRed}{;} \\
\mbox{}\ \ \ \ \ \ dist\textcolor{BrickRed}{[}to\textcolor{BrickRed}{]}\ \textcolor{BrickRed}{=}\ \textbf{\textcolor{Black}{min}}\ \textcolor{BrickRed}{(}dist\textcolor{BrickRed}{[}to\textcolor{BrickRed}{],}\ dist\textcolor{BrickRed}{[}sel\textcolor{BrickRed}{]}\ \textcolor{BrickRed}{+}\ len\textcolor{BrickRed}{);} \\
\mbox{}\ \ \ \ \textcolor{Red}{\}} \\
\mbox{}\ \ \textcolor{Red}{\}} \\
\mbox{} \\
\mbox{}\ \ \textcolor{ForestGreen}{int}\ minw\ \textcolor{BrickRed}{=}\ \textcolor{Purple}{0}\textcolor{BrickRed}{,}\ \ maxw\ \textcolor{BrickRed}{=}\ W\textcolor{BrickRed}{;} \\
\mbox{}\ \ \textbf{\textcolor{Blue}{while}}\textcolor{BrickRed}{(}minw\ \textcolor{BrickRed}{$<$}\ maxw\textcolor{BrickRed}{)}\ \textcolor{Red}{\{} \\
\mbox{}\ \ \ \ \textcolor{ForestGreen}{int}\ wlimit\ \textcolor{BrickRed}{=}\ \textcolor{BrickRed}{(}minw\ \textcolor{BrickRed}{+}\ maxw\textcolor{BrickRed}{)}\ \textcolor{BrickRed}{$>$$>$}\ \textcolor{Purple}{1}\textcolor{BrickRed}{;} \\
\mbox{}\ \ \ \ used\textcolor{BrickRed}{.}\textbf{\textcolor{Black}{assign}}\textcolor{BrickRed}{(}n\textcolor{BrickRed}{,}\ \textbf{\textcolor{Blue}{false}}\textcolor{BrickRed}{);} \\
\mbox{}\ \ \ \ \textbf{\textcolor{Blue}{if}}\textcolor{BrickRed}{(}\textbf{\textcolor{Black}{kth$\_$path$\_$exists}}\textcolor{BrickRed}{(}k\textcolor{BrickRed}{,}\ wlimit\textcolor{BrickRed}{,}\ s\textcolor{BrickRed}{)}\ \textcolor{BrickRed}{==}\ k\textcolor{BrickRed}{)} \\
\mbox{}\ \ \ \ \ \ maxw\ \textcolor{BrickRed}{=}\ wlimit\textcolor{BrickRed}{;} \\
\mbox{}\ \ \ \ \textbf{\textcolor{Blue}{else}} \\
\mbox{}\ \ \ \ \ \ minw\ \textcolor{BrickRed}{=}\ wlimit\ \textcolor{BrickRed}{+}\ \textcolor{Purple}{1}\textcolor{BrickRed}{;} \\
\mbox{}\ \ \textcolor{Red}{\}} \\
\mbox{} \\
\mbox{}\ \ used\textcolor{BrickRed}{.}\textbf{\textcolor{Black}{assign}}\textcolor{BrickRed}{(}n\textcolor{BrickRed}{,}\ \textbf{\textcolor{Blue}{false}}\textcolor{BrickRed}{);} \\
\mbox{}\ \ \textbf{\textcolor{Blue}{if}}\textcolor{BrickRed}{(}\textbf{\textcolor{Black}{kth$\_$path$\_$exists}}\textcolor{BrickRed}{(}k\textcolor{BrickRed}{,}\ minw\textcolor{BrickRed}{,}\ s\textcolor{BrickRed}{)}\ \textcolor{BrickRed}{$<$}\ k\textcolor{BrickRed}{)} \\
\mbox{}\ \ \ \ \textbf{\textcolor{Black}{puts}}\textcolor{BrickRed}{(}\texttt{\textcolor{Red}{"{}NO\ SOLUTION"{}}}\textcolor{BrickRed}{);} \\
\mbox{}\ \ \textbf{\textcolor{Blue}{else}}\ \textcolor{Red}{\{} \\
\mbox{}\ \ \ \ cout\ \textcolor{BrickRed}{$<$$<$}\ minw\ \textcolor{BrickRed}{$<$$<$}\ \texttt{\textcolor{Red}{'\ '}}\ \textcolor{BrickRed}{$<$$<$}\ kth$\_$path\textcolor{BrickRed}{.}\textbf{\textcolor{Black}{size}}\textcolor{BrickRed}{()}\ \textcolor{BrickRed}{$<$$<$}\ endl\textcolor{BrickRed}{;} \\
\mbox{}\ \ \ \ \textbf{\textcolor{Blue}{for}}\textcolor{BrickRed}{(}\textcolor{TealBlue}{size$\_$t}\ i\textcolor{BrickRed}{=}\textcolor{Purple}{0}\textcolor{BrickRed}{;}\ i\textcolor{BrickRed}{$<$}kth$\_$path\textcolor{BrickRed}{.}\textbf{\textcolor{Black}{size}}\textcolor{BrickRed}{();}\ \textcolor{BrickRed}{++}i\textcolor{BrickRed}{)} \\
\mbox{}\ \ \ \ \ \ cout\ \textcolor{BrickRed}{$<$$<$}\ kth$\_$path\textcolor{BrickRed}{[}i\textcolor{BrickRed}{]+}\textcolor{Purple}{1}\ \textcolor{BrickRed}{$<$$<$}\ \texttt{\textcolor{Red}{'\ '}}\textcolor{BrickRed}{;} \\
\mbox{}\ \ \textcolor{Red}{\}} \\
\mbox{} \\
\mbox{}\textcolor{Red}{\}} \\
\mbox{} \\
\mbox{}
} \normalfont\normalsize
%.tex

\subsection{Algoritmo de Dijkstra}
El peso de todas las aristas debe ser no negativo.
\\
\input{./src/grafos/dijkstra}%.tex

\subsection{Minimum spanning tree: Algoritmo de Kruskal + Union-Find}
\input{./src/grafos/kruskal}%.tex

\subsection{Algoritmo de Floyd-Warshall}
\emph{Complejidad:} $ O(n^3) $ \\
Se asume que no hay ciclos de costo negativo.
\input{./src/grafos/floyd}%.tex

\subsection{Algoritmo de Bellman-Ford}
Si no hay ciclos de coste negativo, encuentra la distancia más corta entre un nodo
y todos los demás. Si sí hay, permite saberlo. \\
El coste de las aristas \underline{sí} puede ser negativo.
\input{./src/grafos/bellman}%.tex

\subsection{Puntos de articulación}
\input{./src/grafos/puntos_articulacion}%.tex

\subsection{Máximo flujo: Método de Ford-Fulkerson, algoritmo de Edmonds-Karp}
\medskip
El algoritmo de Edmonds-Karp es una modificación al método de Ford-Fulkerson. Este último
utiliza DFS para hallar un camino de aumentación, pero la sugerencia de Edmonds-Karp
es utilizar BFS que lo hace más eficiente en algunos grafos.
\input{./src/grafos/ford_fulkerson}%.tex

\section{Programación dinámica}
\subsection{Longest common subsequence}
\input{./src/dp/lcs}%.tex

\subsection{M\'axima Submatriz de ceros}
% Generator: GNU source-highlight, by Lorenzo Bettini, http://www.gnu.org/software/src-highlite
{\ttfamily \raggedright {
\noindent
\mbox{}\textcolor{ForestGreen}{int}\ n\textcolor{BrickRed}{,}\ m\textcolor{BrickRed}{;} \\
\mbox{}cin\ \textcolor{BrickRed}{$>$$>$}\ n\ \textcolor{BrickRed}{$>$$>$}\ m\textcolor{BrickRed}{;} \\
\mbox{}vector\ \textcolor{BrickRed}{$<$}\ \textcolor{TealBlue}{vector$<$char$>$\ $>$}\ \textbf{\textcolor{Black}{a}}\ \textcolor{BrickRed}{(}n\textcolor{BrickRed}{,}\ vector\textcolor{BrickRed}{$<$}\textcolor{ForestGreen}{char}\textcolor{BrickRed}{$>$}\ \textcolor{BrickRed}{(}m\textcolor{BrickRed}{));} \\
\mbox{}\textbf{\textcolor{Blue}{for}}\ \textcolor{BrickRed}{(}\textcolor{ForestGreen}{int}\ i\textcolor{BrickRed}{=}\textcolor{Purple}{0}\textcolor{BrickRed}{;}\ i\textcolor{BrickRed}{$<$}n\textcolor{BrickRed}{;}\ \textcolor{BrickRed}{++}i\textcolor{BrickRed}{)} \\
\mbox{}\ \ \textbf{\textcolor{Blue}{for}}\ \textcolor{BrickRed}{(}\textcolor{ForestGreen}{int}\ j\textcolor{BrickRed}{=}\textcolor{Purple}{0}\textcolor{BrickRed}{;}\ j\textcolor{BrickRed}{$<$}m\textcolor{BrickRed}{;}\ \textcolor{BrickRed}{++}j\textcolor{BrickRed}{)} \\
\mbox{}\ \ \ \ cin\ \textcolor{BrickRed}{$>$$>$}\ a\textcolor{BrickRed}{[}i\textcolor{BrickRed}{][}j\textcolor{BrickRed}{];} \\
\mbox{} \\
\mbox{}\textcolor{ForestGreen}{int}\ ans\ \textcolor{BrickRed}{=}\ \textcolor{Purple}{0}\textcolor{BrickRed}{;} \\
\mbox{}\textcolor{TealBlue}{vector$<$int$>$}\ \textbf{\textcolor{Black}{d}}\ \textcolor{BrickRed}{(}m\textcolor{BrickRed}{,}\ \textcolor{BrickRed}{-}\textcolor{Purple}{1}\textcolor{BrickRed}{);} \\
\mbox{}\textcolor{TealBlue}{vector$<$int$>$}\ \textbf{\textcolor{Black}{dl}}\ \textcolor{BrickRed}{(}m\textcolor{BrickRed}{),}\ \textbf{\textcolor{Black}{dr}}\ \textcolor{BrickRed}{(}m\textcolor{BrickRed}{);} \\
\mbox{}\textcolor{TealBlue}{stack$<$int$>$}\ st\textcolor{BrickRed}{;} \\
\mbox{}\textbf{\textcolor{Blue}{for}}\ \textcolor{BrickRed}{(}\textcolor{ForestGreen}{int}\ i\textcolor{BrickRed}{=}\textcolor{Purple}{0}\textcolor{BrickRed}{;}\ i\textcolor{BrickRed}{$<$}n\textcolor{BrickRed}{;}\ \textcolor{BrickRed}{++}i\textcolor{BrickRed}{)}\ \textcolor{Red}{\{} \\
\mbox{}\ \ \textbf{\textcolor{Blue}{for}}\ \textcolor{BrickRed}{(}\textcolor{ForestGreen}{int}\ j\textcolor{BrickRed}{=}\textcolor{Purple}{0}\textcolor{BrickRed}{;}\ j\textcolor{BrickRed}{$<$}m\textcolor{BrickRed}{;}\ \textcolor{BrickRed}{++}j\textcolor{BrickRed}{)} \\
\mbox{}\ \ \ \ \textbf{\textcolor{Blue}{if}}\ \textcolor{BrickRed}{(}a\textcolor{BrickRed}{[}i\textcolor{BrickRed}{][}j\textcolor{BrickRed}{]}\ \textcolor{BrickRed}{==}\ \textcolor{Purple}{1}\textcolor{BrickRed}{)} \\
\mbox{}\ \ \ \ \ \ d\textcolor{BrickRed}{[}j\textcolor{BrickRed}{]}\ \textcolor{BrickRed}{=}\ i\textcolor{BrickRed}{;} \\
\mbox{}\ \ \textbf{\textcolor{Blue}{while}}\ \textcolor{BrickRed}{(!}st\textcolor{BrickRed}{.}\textbf{\textcolor{Black}{empty}}\textcolor{BrickRed}{())}\ st\textcolor{BrickRed}{.}\textbf{\textcolor{Black}{pop}}\textcolor{BrickRed}{();} \\
\mbox{}\ \ \textbf{\textcolor{Blue}{for}}\ \textcolor{BrickRed}{(}\textcolor{ForestGreen}{int}\ j\textcolor{BrickRed}{=}\textcolor{Purple}{0}\textcolor{BrickRed}{;}\ j\textcolor{BrickRed}{$<$}m\textcolor{BrickRed}{;}\ \textcolor{BrickRed}{++}j\textcolor{BrickRed}{)}\ \textcolor{Red}{\{} \\
\mbox{}\ \ \ \ \textbf{\textcolor{Blue}{while}}\ \textcolor{BrickRed}{(!}st\textcolor{BrickRed}{.}\textbf{\textcolor{Black}{empty}}\textcolor{BrickRed}{()}\ \textcolor{BrickRed}{\&\&}\ d\textcolor{BrickRed}{[}st\textcolor{BrickRed}{.}\textbf{\textcolor{Black}{top}}\textcolor{BrickRed}{()]}\ \textcolor{BrickRed}{$<$=}\ d\textcolor{BrickRed}{[}j\textcolor{BrickRed}{])}\ \ st\textcolor{BrickRed}{.}\textbf{\textcolor{Black}{pop}}\textcolor{BrickRed}{();} \\
\mbox{}\ \ \ \ dl\textcolor{BrickRed}{[}j\textcolor{BrickRed}{]}\ \textcolor{BrickRed}{=}\ st\textcolor{BrickRed}{.}\textbf{\textcolor{Black}{empty}}\textcolor{BrickRed}{()}\ \textcolor{BrickRed}{?}\ \textcolor{BrickRed}{-}\textcolor{Purple}{1}\ \textcolor{BrickRed}{:}\ st\textcolor{BrickRed}{.}\textbf{\textcolor{Black}{top}}\textcolor{BrickRed}{();} \\
\mbox{}\ \ \ \ st\textcolor{BrickRed}{.}\textbf{\textcolor{Black}{push}}\ \textcolor{BrickRed}{(}j\textcolor{BrickRed}{);} \\
\mbox{}\ \ \textcolor{Red}{\}} \\
\mbox{}\ \ \textbf{\textcolor{Blue}{while}}\ \textcolor{BrickRed}{(!}st\textcolor{BrickRed}{.}\textbf{\textcolor{Black}{empty}}\textcolor{BrickRed}{())}\ st\textcolor{BrickRed}{.}\textbf{\textcolor{Black}{pop}}\textcolor{BrickRed}{();} \\
\mbox{}\ \ \textbf{\textcolor{Blue}{for}}\ \textcolor{BrickRed}{(}\textcolor{ForestGreen}{int}\ j\textcolor{BrickRed}{=}m\textcolor{BrickRed}{-}\textcolor{Purple}{1}\textcolor{BrickRed}{;}\ j\textcolor{BrickRed}{$>$=}\textcolor{Purple}{0}\textcolor{BrickRed}{;}\ \textcolor{BrickRed}{-\/-}j\textcolor{BrickRed}{)}\ \textcolor{Red}{\{} \\
\mbox{}\ \ \ \ \textbf{\textcolor{Blue}{while}}\ \textcolor{BrickRed}{(!}st\textcolor{BrickRed}{.}\textbf{\textcolor{Black}{empty}}\textcolor{BrickRed}{()}\ \textcolor{BrickRed}{\&\&}\ d\textcolor{BrickRed}{[}st\textcolor{BrickRed}{.}\textbf{\textcolor{Black}{top}}\textcolor{BrickRed}{()]}\ \textcolor{BrickRed}{$<$=}\ d\textcolor{BrickRed}{[}j\textcolor{BrickRed}{])}\ \ st\textcolor{BrickRed}{.}\textbf{\textcolor{Black}{pop}}\textcolor{BrickRed}{();} \\
\mbox{}\ \ \ \ dr\textcolor{BrickRed}{[}j\textcolor{BrickRed}{]}\ \textcolor{BrickRed}{=}\ st\textcolor{BrickRed}{.}\textbf{\textcolor{Black}{empty}}\textcolor{BrickRed}{()}\ \textcolor{BrickRed}{?}\ m\ \textcolor{BrickRed}{:}\ st\textcolor{BrickRed}{.}\textbf{\textcolor{Black}{top}}\textcolor{BrickRed}{();} \\
\mbox{}\ \ \ \ st\textcolor{BrickRed}{.}\textbf{\textcolor{Black}{push}}\ \textcolor{BrickRed}{(}j\textcolor{BrickRed}{);} \\
\mbox{}\ \ \textcolor{Red}{\}} \\
\mbox{}\ \ \textbf{\textcolor{Blue}{for}}\ \textcolor{BrickRed}{(}\textcolor{ForestGreen}{int}\ j\textcolor{BrickRed}{=}\textcolor{Purple}{0}\textcolor{BrickRed}{;}\ j\textcolor{BrickRed}{$<$}m\textcolor{BrickRed}{;}\ \textcolor{BrickRed}{++}j\textcolor{BrickRed}{)} \\
\mbox{}\ \ \ \ ans\ \textcolor{BrickRed}{=}\ \textbf{\textcolor{Black}{max}}\ \textcolor{BrickRed}{(}ans\textcolor{BrickRed}{,}\ \textcolor{BrickRed}{(}i\ \textcolor{BrickRed}{-}\ d\textcolor{BrickRed}{[}j\textcolor{BrickRed}{])}\ \textcolor{BrickRed}{*}\ \textcolor{BrickRed}{(}dr\textcolor{BrickRed}{[}j\textcolor{BrickRed}{]}\ \textcolor{BrickRed}{-}\ dl\textcolor{BrickRed}{[}j\textcolor{BrickRed}{]}\ \textcolor{BrickRed}{-}\ \textcolor{Purple}{1}\textcolor{BrickRed}{));} \\
\mbox{}\textcolor{Red}{\}} \\
\mbox{} \\
\mbox{}cout\ \textcolor{BrickRed}{$<$$<$}\ ans\textcolor{BrickRed}{;}
}%.tex

\section{Geometría}
\subsection{Área de un polígono}
Si P es un polígono simple (no se intersecta a sí mismo) su área está dada por: \\

$ A(P) = \frac{1}{2} \displaystyle\sum_{i=0}^{n-1} (x_{i} \cdot y_{i+1} - x_{i+1} \cdot y_{i}) $ \\
\bigskip
\input{./src/geometria/polygon_area}%.tex

\subsection{Centro de masa de un polígono}
Si P es un polígono simple (no se intersecta a sí mismo) su centro de masa está dado por: \\

$ \displaystyle\bar{C}_{x} = \frac{ \displaystyle\iint_{R} x \, dA }{M} = \frac{1}{6M}\sum_{i=1}^{n} (y_{i+1} - y_{i}) (x_{i+1}^2 + x_{i+1} \cdot x_{i} + x_{i}^2) $

\medskip

$\displaystyle\bar{C}_{y} = \frac{ \displaystyle\iint_{R} y \, dA }{M} = \frac{1}{6M} \sum_{i=1}^{n} (x_{i} - x_{i+1}) (y_{i+1}^2 + y_{i+1} \cdot y_{i} + y_{i}^2)$

\medskip

Donde $ M $ es el área del polígono. \\

Otra posible fórmula equivalente:

$ \displaystyle\bar{C}_{x} = \frac{1}{6A} \sum_{i=0}^{n-1} (x_{i} + x_{i+1}) (x_{i} \cdot y_{i+1} - x_{i+1} \cdot y_{i}) $

\medskip

$ \displaystyle\bar{C}_{y} = \frac{1}{6A} \sum_{i=0}^{n-1} (y_{i} + y_{i+1}) (x_{i} \cdot y_{i+1} - x_{i+1} \cdot y_{i}) $


\subsection{Convex hull: Graham Scan}
\emph{Complejidad:} $ O(n \log_{2}{n}) $
\input{./src/geometria/grahamscan}%.tex

\subsection{Mínima distancia entre un punto y un segmento}
\input{./src/geometria/distance_point_to_segment}%.tex

\subsection{Mínima distancia entre un punto y una recta}
\input{./src/geometria/distance_point_to_line}%.tex

\subsection{Determinar si un polígono es convexo}
\input{./src/geometria/is_convex_polygon}%.tex

\end{document}
