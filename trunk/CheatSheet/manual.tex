%------------------------------------------------------------------------------%
%
%
%
%------------------------------------------------------------------------------%
\documentclass[10pt,letterpaper]{article}

%---------------------------------------------------------------
\usepackage[utf8]{inputenc}
\usepackage[spanish]{babel}
\usepackage{listings}
\usepackage[usenames,dvipsnames]{color}
\usepackage{amsmath}
\usepackage{verbatim}
% \usepackage[colorlinks]{hyperref}
\usepackage{longtable}
%\usepackage{color}
%---------------------------------------------------------------

\setlength{\topmargin}{-1.0in}
\setlength{\textheight}{9.5in} 
\setlength{\evensidemargin}{0.0in}
\setlength{\oddsidemargin}{0.0in}
\setlength{\textwidth}{6.5in} 

\begin{document}

%---------------------------------------------------------------
\title{ACM ICPC Bolivia CheatSheet}
\author{}
\date{}
\maketitle
\newpage
%---------------------------------------------------------------

%---------------------------------------------------------------
\tableofcontents
%\lstlistoflistings
\lstloadlanguages{C++}
%---------------------------------------------------------------
%---------------------------------------------------------------
%\section{Teoría de números}
\section{Matem\'atica}
%---------------------------------------------------------------

\subsection{Karatsuba}
% Generator: GNU source-highlight, by Lorenzo Bettini, http://www.gnu.org/software/src-highlite
{\ttfamily \raggedright {
\noindent
\mbox{}\textit{\textcolor{Brown}{//Mandar\ como\ Parametro\ N\ el\ numero\ de\ bits}} \\
\mbox{}\textbf{\textcolor{RoyalBlue}{import}}\ java\textcolor{BrickRed}{.}math\textcolor{BrickRed}{.}BigInteger\textcolor{BrickRed}{;} \\
\mbox{}\textbf{\textcolor{RoyalBlue}{import}}\ java\textcolor{BrickRed}{.}util\textcolor{BrickRed}{.}Random\textcolor{BrickRed}{;} \\
\mbox{}\textbf{\textcolor{Blue}{class}}\ \textcolor{TealBlue}{Karatsuba}\ \textcolor{Red}{\{} \\
\mbox{}\ \ \ \ \textbf{\textcolor{Blue}{private}}\ \textbf{\textcolor{Blue}{final}}\ \textbf{\textcolor{Blue}{static}}\ \textcolor{TealBlue}{BigInteger}\ ZERO\ \textcolor{BrickRed}{=}\ \textbf{\textcolor{Blue}{new}}\ \textbf{\textcolor{Black}{BigInteger}}\textcolor{BrickRed}{(}\texttt{\textcolor{Red}{"{}0"{}}}\textcolor{BrickRed}{);} \\
\mbox{}\ \ \ \ \textbf{\textcolor{Blue}{public}}\ \textbf{\textcolor{Blue}{static}}\ \textcolor{TealBlue}{BigInteger}\ \textbf{\textcolor{Black}{karatsuba}}\textcolor{BrickRed}{(}\textcolor{TealBlue}{BigInteger}\ x\textcolor{BrickRed}{,}\ \textcolor{TealBlue}{BigInteger}\ y\textcolor{BrickRed}{)}\ \textcolor{Red}{\{} \\
\mbox{}\ \ \ \ \ \ \ \ \\
\mbox{}\ \ \ \ \ \ \ \ \textcolor{ForestGreen}{int}\ N\ \textcolor{BrickRed}{=}\ Math\textcolor{BrickRed}{.}\textbf{\textcolor{Black}{max}}\textcolor{BrickRed}{(}x\textcolor{BrickRed}{.}\textbf{\textcolor{Black}{bitLength}}\textcolor{BrickRed}{(),}\ y\textcolor{BrickRed}{.}\textbf{\textcolor{Black}{bitLength}}\textcolor{BrickRed}{());} \\
\mbox{}\ \ \ \ \ \ \ \ \textbf{\textcolor{Blue}{if}}\ \textcolor{BrickRed}{(}N\ \textcolor{BrickRed}{$<$=}\ \textcolor{Purple}{2000}\textcolor{BrickRed}{)}\ \textbf{\textcolor{Blue}{return}}\ x\textcolor{BrickRed}{.}\textbf{\textcolor{Black}{multiply}}\textcolor{BrickRed}{(}y\textcolor{BrickRed}{);}\ \ \ \ \ \ \ \ \ \ \ \ \ \ \ \ \ \\
\mbox{}\ \ \ \ \ \ \ \ \\
\mbox{}\ \ \ \ \ \ \ \ N\ \textcolor{BrickRed}{=}\ \textcolor{BrickRed}{(}N\ \textcolor{BrickRed}{/}\ \textcolor{Purple}{2}\textcolor{BrickRed}{)}\ \textcolor{BrickRed}{+}\ \textcolor{BrickRed}{(}N\ \textcolor{BrickRed}{\%}\ \textcolor{Purple}{2}\textcolor{BrickRed}{);} \\
\mbox{}\ \ \ \ \ \ \ \ \\
\mbox{}\ \ \ \ \ \ \ \ \textcolor{TealBlue}{BigInteger}\ b\ \textcolor{BrickRed}{=}\ x\textcolor{BrickRed}{.}\textbf{\textcolor{Black}{shiftRight}}\textcolor{BrickRed}{(}N\textcolor{BrickRed}{);} \\
\mbox{}\ \ \ \ \ \ \ \ \textcolor{TealBlue}{BigInteger}\ a\ \textcolor{BrickRed}{=}\ x\textcolor{BrickRed}{.}\textbf{\textcolor{Black}{subtract}}\textcolor{BrickRed}{(}b\textcolor{BrickRed}{.}\textbf{\textcolor{Black}{shiftLeft}}\textcolor{BrickRed}{(}N\textcolor{BrickRed}{));} \\
\mbox{}\ \ \ \ \ \ \ \ \textcolor{TealBlue}{BigInteger}\ d\ \textcolor{BrickRed}{=}\ y\textcolor{BrickRed}{.}\textbf{\textcolor{Black}{shiftRight}}\textcolor{BrickRed}{(}N\textcolor{BrickRed}{);} \\
\mbox{}\ \ \ \ \ \ \ \ \textcolor{TealBlue}{BigInteger}\ c\ \textcolor{BrickRed}{=}\ y\textcolor{BrickRed}{.}\textbf{\textcolor{Black}{subtract}}\textcolor{BrickRed}{(}d\textcolor{BrickRed}{.}\textbf{\textcolor{Black}{shiftLeft}}\textcolor{BrickRed}{(}N\textcolor{BrickRed}{));} \\
\mbox{}\ \ \ \ \ \ \ \ \\
\mbox{}\ \ \ \ \ \ \ \ \textcolor{TealBlue}{BigInteger}\ ac\ \ \ \ \textcolor{BrickRed}{=}\ \textbf{\textcolor{Black}{karatsuba}}\textcolor{BrickRed}{(}a\textcolor{BrickRed}{,}\ c\textcolor{BrickRed}{);} \\
\mbox{}\ \ \ \ \ \ \ \ \textcolor{TealBlue}{BigInteger}\ bd\ \ \ \ \textcolor{BrickRed}{=}\ \textbf{\textcolor{Black}{karatsuba}}\textcolor{BrickRed}{(}b\textcolor{BrickRed}{,}\ d\textcolor{BrickRed}{);} \\
\mbox{}\ \ \ \ \ \ \ \ \textcolor{TealBlue}{BigInteger}\ abcd\ \ \textcolor{BrickRed}{=}\ \textbf{\textcolor{Black}{karatsuba}}\textcolor{BrickRed}{(}a\textcolor{BrickRed}{.}\textbf{\textcolor{Black}{add}}\textcolor{BrickRed}{(}b\textcolor{BrickRed}{),}\ c\textcolor{BrickRed}{.}\textbf{\textcolor{Black}{add}}\textcolor{BrickRed}{(}d\textcolor{BrickRed}{));} \\
\mbox{}\ \ \ \ \ \ \ \ \textbf{\textcolor{Blue}{return}}\ ac\textcolor{BrickRed}{.}\textbf{\textcolor{Black}{add}}\textcolor{BrickRed}{(}abcd\textcolor{BrickRed}{.}\textbf{\textcolor{Black}{subtract}}\textcolor{BrickRed}{(}ac\textcolor{BrickRed}{).}\textbf{\textcolor{Black}{subtract}}\textcolor{BrickRed}{(}bd\textcolor{BrickRed}{).}\textbf{\textcolor{Black}{shiftLeft}}\textcolor{BrickRed}{(}N\textcolor{BrickRed}{)).}\textbf{\textcolor{Black}{add}}\textcolor{BrickRed}{(}bd\textcolor{BrickRed}{.}\textbf{\textcolor{Black}{shiftLeft}}\textcolor{BrickRed}{(}\textcolor{Purple}{2}\textcolor{BrickRed}{*}N\textcolor{BrickRed}{));} \\
\mbox{}\ \ \ \ \textcolor{Red}{\}} \\
\mbox{}\ \ \ \ \textbf{\textcolor{Blue}{public}}\ \textbf{\textcolor{Blue}{static}}\ \textcolor{ForestGreen}{void}\ \textbf{\textcolor{Black}{main}}\textcolor{BrickRed}{(}String\textcolor{BrickRed}{[]}\ args\textcolor{BrickRed}{)}\ \textcolor{Red}{\{} \\
\mbox{}\ \ \ \ \ \ \ \ \textcolor{ForestGreen}{long}\ start\textcolor{BrickRed}{,}\ stop\textcolor{BrickRed}{,}\ elapsed\textcolor{BrickRed}{;} \\
\mbox{}\ \ \ \ \ \ \ \ \textcolor{TealBlue}{Random}\ random\ \textcolor{BrickRed}{=}\ \textbf{\textcolor{Blue}{new}}\ \textbf{\textcolor{Black}{Random}}\textcolor{BrickRed}{();} \\
\mbox{}\ \ \ \ \ \ \ \ \textcolor{ForestGreen}{int}\ N\ \textcolor{BrickRed}{=}\ Integer\textcolor{BrickRed}{.}\textbf{\textcolor{Black}{parseInt}}\textcolor{BrickRed}{(}args\textcolor{BrickRed}{[}\textcolor{Purple}{0}\textcolor{BrickRed}{]);} \\
\mbox{}\ \ \ \ \ \ \ \ \textcolor{TealBlue}{BigInteger}\ a\ \textcolor{BrickRed}{=}\ \textbf{\textcolor{Blue}{new}}\ \textbf{\textcolor{Black}{BigInteger}}\textcolor{BrickRed}{(}N\textcolor{BrickRed}{,}\ random\textcolor{BrickRed}{);} \\
\mbox{}\ \ \ \ \ \ \ \ \textcolor{TealBlue}{BigInteger}\ b\ \textcolor{BrickRed}{=}\ \textbf{\textcolor{Blue}{new}}\ \textbf{\textcolor{Black}{BigInteger}}\textcolor{BrickRed}{(}N\textcolor{BrickRed}{,}\ random\textcolor{BrickRed}{);} \\
\mbox{}\ \ \ \ \ \ \ \ start\ \textcolor{BrickRed}{=}\ System\textcolor{BrickRed}{.}\textbf{\textcolor{Black}{currentTimeMillis}}\textcolor{BrickRed}{();}\  \\
\mbox{}\ \ \ \ \ \ \ \ \textcolor{TealBlue}{BigInteger}\ c\ \textcolor{BrickRed}{=}\ \textbf{\textcolor{Black}{karatsuba}}\textcolor{BrickRed}{(}a\textcolor{BrickRed}{,}\ b\textcolor{BrickRed}{);} \\
\mbox{}\ \ \ \ \ \ \ \ stop\ \textcolor{BrickRed}{=}\ System\textcolor{BrickRed}{.}\textbf{\textcolor{Black}{currentTimeMillis}}\textcolor{BrickRed}{();} \\
\mbox{}\ \ \ \ \ \ \ \ System\textcolor{BrickRed}{.}out\textcolor{BrickRed}{.}\textbf{\textcolor{Black}{println}}\textcolor{BrickRed}{(}stop\ \textcolor{BrickRed}{-}\ start\textcolor{BrickRed}{);} \\
\mbox{}\ \ \ \ \ \ \ \ start\ \textcolor{BrickRed}{=}\ System\textcolor{BrickRed}{.}\textbf{\textcolor{Black}{currentTimeMillis}}\textcolor{BrickRed}{();}\  \\
\mbox{}\ \ \ \ \ \ \ \ \textcolor{TealBlue}{BigInteger}\ d\ \textcolor{BrickRed}{=}\ a\textcolor{BrickRed}{.}\textbf{\textcolor{Black}{multiply}}\textcolor{BrickRed}{(}b\textcolor{BrickRed}{);} \\
\mbox{}\ \ \ \ \ \ \ \ stop\ \textcolor{BrickRed}{=}\ System\textcolor{BrickRed}{.}\textbf{\textcolor{Black}{currentTimeMillis}}\textcolor{BrickRed}{();} \\
\mbox{}\ \ \ \ \ \ \ \ System\textcolor{BrickRed}{.}out\textcolor{BrickRed}{.}\textbf{\textcolor{Black}{println}}\textcolor{BrickRed}{(}stop\ \textcolor{BrickRed}{-}\ start\textcolor{BrickRed}{);} \\
\mbox{}\ \ \ \ \ \ \ \ System\textcolor{BrickRed}{.}out\textcolor{BrickRed}{.}\textbf{\textcolor{Black}{println}}\textcolor{BrickRed}{((}c\textcolor{BrickRed}{.}\textbf{\textcolor{Black}{equals}}\textcolor{BrickRed}{(}d\textcolor{BrickRed}{)));} \\
\mbox{}\ \ \ \ \textcolor{Red}{\}} \\
\mbox{}\textcolor{Red}{\}}
}%.tex

\subsection{Integraci\'on por Simpson}
$$
\int_a^b f(x) dx
$$
% Generator: GNU source-highlight, by Lorenzo Bettini, http://www.gnu.org/software/src-highlite
{\ttfamily \raggedright {
\noindent
\mbox{}\textcolor{ForestGreen}{double}\ a\textcolor{BrickRed}{,}\ b\textcolor{BrickRed}{;}\ \textit{\textcolor{Brown}{//\ limites}} \\
\mbox{}\textbf{\textcolor{Blue}{const}}\ \textcolor{ForestGreen}{int}\ N\ \textcolor{BrickRed}{=}\ \textcolor{Purple}{1000}\textcolor{BrickRed}{*}\textcolor{Purple}{1000}\textcolor{BrickRed}{;} \\
\mbox{}\textcolor{ForestGreen}{double}\ s\ \textcolor{BrickRed}{=}\ \textcolor{Purple}{0}\textcolor{BrickRed}{;} \\
\mbox{}\textbf{\textcolor{Blue}{for}}\ \textcolor{BrickRed}{(}\textcolor{ForestGreen}{int}\ i\textcolor{BrickRed}{=}\textcolor{Purple}{0}\textcolor{BrickRed}{;}\ i\textcolor{BrickRed}{$<$=}N\textcolor{BrickRed}{;}\ \textcolor{BrickRed}{++}i\textcolor{BrickRed}{)}\ \textcolor{Red}{\{} \\
\mbox{}\ \ \ \ \ \ \ \ \textcolor{ForestGreen}{double}\ x\ \textcolor{BrickRed}{=}\ a\ \textcolor{BrickRed}{+}\ \textcolor{BrickRed}{(}b\ \textcolor{BrickRed}{-}\ a\textcolor{BrickRed}{)}\ \textcolor{BrickRed}{*}\ i\ \textcolor{BrickRed}{/}\ N\textcolor{BrickRed}{;} \\
\mbox{}\ \ \ \ \ \ \ \ s\ \textcolor{BrickRed}{+=}\ \textbf{\textcolor{Black}{f}}\textcolor{BrickRed}{(}x\textcolor{BrickRed}{)}\ \textcolor{BrickRed}{*}\ \textcolor{BrickRed}{(}i\textcolor{BrickRed}{==}\textcolor{Purple}{0}\ \textcolor{BrickRed}{$|$$|$}\ i\textcolor{BrickRed}{==}N\ \textcolor{BrickRed}{?}\ \textcolor{Purple}{1}\ \textcolor{BrickRed}{:}\ \textcolor{BrickRed}{(}i\textcolor{BrickRed}{\&}\textcolor{Purple}{1}\textcolor{BrickRed}{)==}\textcolor{Purple}{0}\ \textcolor{BrickRed}{?}\ \textcolor{Purple}{2}\ \textcolor{BrickRed}{:}\ \textcolor{Purple}{4}\textcolor{BrickRed}{);} \\
\mbox{}\textcolor{Red}{\}} \\
\mbox{}\textcolor{ForestGreen}{double}\ delta\ \textcolor{BrickRed}{=}\ \textcolor{BrickRed}{(}b\ \textcolor{BrickRed}{-}\ a\textcolor{BrickRed}{)}\ \textcolor{BrickRed}{/}\ N\textcolor{BrickRed}{;} \\
\mbox{}s\ \textcolor{BrickRed}{*=}\ delta\ \textcolor{BrickRed}{/}\ \textcolor{Purple}{3.0}\textcolor{BrickRed}{;}
}
%.tex

\subsection{Phi de Euler}
% Generator: GNU source-highlight, by Lorenzo Bettini, http://www.gnu.org/software/src-highlite
{\ttfamily \raggedright {
\noindent
\mbox{}\textcolor{ForestGreen}{int}\ \textbf{\textcolor{Black}{phi}}\ \textcolor{BrickRed}{(}\textcolor{ForestGreen}{int}\ n\textcolor{BrickRed}{)}\ \textcolor{Red}{\{} \\
\mbox{}\ \ \ \ \ \ \ \ \textcolor{ForestGreen}{int}\ result\ \textcolor{BrickRed}{=}\ n\textcolor{BrickRed}{;} \\
\mbox{}\ \ \ \ \ \ \ \ \textbf{\textcolor{Blue}{for}}\ \textcolor{BrickRed}{(}\textcolor{ForestGreen}{int}\ i\textcolor{BrickRed}{=}\textcolor{Purple}{2}\textcolor{BrickRed}{;}\ i\textcolor{BrickRed}{*}i\textcolor{BrickRed}{$<$=}n\textcolor{BrickRed}{;}\ \textcolor{BrickRed}{++}i\textcolor{BrickRed}{)} \\
\mbox{}\ \ \ \ \ \ \ \ \ \ \ \ \ \ \ \ \textbf{\textcolor{Blue}{if}}\ \textcolor{BrickRed}{(}n\ \textcolor{BrickRed}{\%}\ i\ \textcolor{BrickRed}{==}\ \textcolor{Purple}{0}\textcolor{BrickRed}{)}\ \textcolor{Red}{\{} \\
\mbox{}\ \ \ \ \ \ \ \ \ \ \ \ \ \ \ \ \ \ \ \ \ \ \ \ \textbf{\textcolor{Blue}{while}}\ \textcolor{BrickRed}{(}n\ \textcolor{BrickRed}{\%}\ i\ \textcolor{BrickRed}{==}\ \textcolor{Purple}{0}\textcolor{BrickRed}{)} \\
\mbox{}\ \ \ \ \ \ \ \ \ \ \ \ \ \ \ \ \ \ \ \ \ \ \ \ \ \ \ \ \ \ \ \ n\ \textcolor{BrickRed}{/=}\ i\textcolor{BrickRed}{;} \\
\mbox{}\ \ \ \ \ \ \ \ \ \ \ \ \ \ \ \ \ \ \ \ \ \ \ \ result\ \textcolor{BrickRed}{-=}\ result\ \textcolor{BrickRed}{/}\ i\textcolor{BrickRed}{;} \\
\mbox{}\ \ \ \ \ \ \ \ \ \ \ \ \ \ \ \ \textcolor{Red}{\}} \\
\mbox{}\ \ \ \ \ \ \ \ \textbf{\textcolor{Blue}{if}}\ \textcolor{BrickRed}{(}n\ \textcolor{BrickRed}{$>$}\ \textcolor{Purple}{1}\textcolor{BrickRed}{)} \\
\mbox{}\ \ \ \ \ \ \ \ \ \ \ \ \ \ \ \ result\ \textcolor{BrickRed}{-=}\ result\ \textcolor{BrickRed}{/}\ n\textcolor{BrickRed}{;} \\
\mbox{}\ \ \ \ \ \ \ \ \textbf{\textcolor{Blue}{return}}\ result\textcolor{BrickRed}{;} \\
\mbox{}\textcolor{Red}{\}} \\
\mbox{}
}%.tex

\subsection{Modulo en Factorial}
% Generator: GNU source-highlight, by Lorenzo Bettini, http://www.gnu.org/software/src-highlite
{\ttfamily \raggedright {
\noindent
\mbox{}\textit{\textcolor{Brown}{//n!\ mod\ p}} \\
\mbox{}\textcolor{ForestGreen}{int}\ \textbf{\textcolor{Black}{factmod}}\ \textcolor{BrickRed}{(}\textcolor{ForestGreen}{int}\ n\textcolor{BrickRed}{,}\ \textcolor{ForestGreen}{int}\ p\textcolor{BrickRed}{)}\ \textcolor{Red}{\{} \\
\mbox{}\ \ \ \ \ \ \ \ \textcolor{ForestGreen}{long}\ \textcolor{ForestGreen}{long}\ res\ \textcolor{BrickRed}{=}\ \textcolor{Purple}{1}\textcolor{BrickRed}{;} \\
\mbox{}\ \ \ \ \ \ \ \ \textbf{\textcolor{Blue}{while}}\ \textcolor{BrickRed}{(}n\ \textcolor{BrickRed}{$>$}\ \textcolor{Purple}{1}\textcolor{BrickRed}{)}\ \textcolor{Red}{\{} \\
\mbox{}\ \ \ \ \ \ \ \ \ \ \ \ \ \ \ \ res\ \textcolor{BrickRed}{=}\ \textcolor{BrickRed}{(}res\ \textcolor{BrickRed}{*}\ \textbf{\textcolor{Black}{powmod}}\ \textcolor{BrickRed}{(}p\textcolor{BrickRed}{-}\textcolor{Purple}{1}\textcolor{BrickRed}{,}\ n\textcolor{BrickRed}{/}p\textcolor{BrickRed}{,}\ p\textcolor{BrickRed}{))}\ \textcolor{BrickRed}{\%}\ p\textcolor{BrickRed}{;} \\
\mbox{}\ \ \ \ \ \ \ \ \ \ \ \ \ \ \ \ \textbf{\textcolor{Blue}{for}}\ \textcolor{BrickRed}{(}\textcolor{ForestGreen}{int}\ i\textcolor{BrickRed}{=}\textcolor{Purple}{2}\textcolor{BrickRed}{;}\ i\textcolor{BrickRed}{$<$=}n\textcolor{BrickRed}{\%}p\textcolor{BrickRed}{;}\ \textcolor{BrickRed}{++}i\textcolor{BrickRed}{)} \\
\mbox{}\ \ \ \ \ \ \ \ \ \ \ \ \ \ \ \ \ \ \ \ \ \ \ \ res\ \textcolor{BrickRed}{=}\ \textcolor{BrickRed}{(}res\ \textcolor{BrickRed}{*}\ i\textcolor{BrickRed}{)}\ \textcolor{BrickRed}{\%}\ p\textcolor{BrickRed}{;} \\
\mbox{}\ \ \ \ \ \ \ \ \ \ \ \ \ \ \ \ n\ \textcolor{BrickRed}{/=}\ p\textcolor{BrickRed}{;} \\
\mbox{}\ \ \ \ \ \ \ \ \textcolor{Red}{\}} \\
\mbox{}\ \ \ \ \ \ \ \ \textbf{\textcolor{Blue}{return}}\ \textcolor{ForestGreen}{int}\ \textcolor{BrickRed}{(}res\ \textcolor{BrickRed}{\%}\ p\textcolor{BrickRed}{);} \\
\mbox{}\textcolor{Red}{\}}
}%.tex

\subsection{Exponenciaci\'on Binaria}
% Generator: GNU source-highlight, by Lorenzo Bettini, http://www.gnu.org/software/src-highlite
{\ttfamily \raggedright {
\noindent
\mbox{}\textcolor{ForestGreen}{int}\ \textbf{\textcolor{Black}{binpow}}\ \textcolor{BrickRed}{(}\textcolor{ForestGreen}{int}\ a\textcolor{BrickRed}{,}\ \textcolor{ForestGreen}{int}\ n\textcolor{BrickRed}{)}\ \textcolor{Red}{\{} \\
\mbox{}\ \ \ \ \ \ \ \ \textcolor{ForestGreen}{int}\ res\ \textcolor{BrickRed}{=}\ \textcolor{Purple}{1}\textcolor{BrickRed}{;} \\
\mbox{}\ \ \ \ \ \ \ \ \textbf{\textcolor{Blue}{while}}\ \textcolor{BrickRed}{(}n\textcolor{BrickRed}{)} \\
\mbox{}\ \ \ \ \ \ \ \ \ \ \ \ \ \ \ \ \textbf{\textcolor{Blue}{if}}\ \textcolor{BrickRed}{(}n\ \textcolor{BrickRed}{\&}\ \textcolor{Purple}{1}\textcolor{BrickRed}{)}\ \textcolor{Red}{\{} \\
\mbox{}\ \ \ \ \ \ \ \ \ \ \ \ \ \ \ \ \ \ \ \ \ \ \ \ res\ \textcolor{BrickRed}{*=}\ a\textcolor{BrickRed}{;} \\
\mbox{}\ \ \ \ \ \ \ \ \ \ \ \ \ \ \ \ \ \ \ \ \ \ \ \ \textcolor{BrickRed}{-\/-}n\textcolor{BrickRed}{;} \\
\mbox{}\ \ \ \ \ \ \ \ \ \ \ \ \ \ \ \ \textcolor{Red}{\}} \\
\mbox{}\ \ \ \ \ \ \ \ \ \ \ \ \ \ \ \ \textbf{\textcolor{Blue}{else}}\ \textcolor{Red}{\{} \\
\mbox{}\ \ \ \ \ \ \ \ \ \ \ \ \ \ \ \ \ \ \ \ \ \ \ \ a\ \textcolor{BrickRed}{*=}\ a\textcolor{BrickRed}{;} \\
\mbox{}\ \ \ \ \ \ \ \ \ \ \ \ \ \ \ \ \ \ \ \ \ \ \ \ n\ \textcolor{BrickRed}{$>$$>$=}\ \textcolor{Purple}{1}\textcolor{BrickRed}{;} \\
\mbox{}\ \ \ \ \ \ \ \ \ \ \ \ \ \ \ \ \textcolor{Red}{\}} \\
\mbox{}\ \ \ \ \ \ \ \ \textbf{\textcolor{Blue}{return}}\ res\textcolor{BrickRed}{;} \\
\mbox{}\textcolor{Red}{\}}
}%.tex

\section{Grafos}
\subsection{Ordenamiento Topologico}
{\ttfamily \raggedright {
% Generator: GNU source-highlight, by Lorenzo Bettini, http://www.gnu.org/software/src-highlite
\noindent
\mbox{}vector\ \textcolor{BrickRed}{$<$}\ \textcolor{TealBlue}{vector$<$int$>$\ $>$}\ g\textcolor{BrickRed}{;} \\
\mbox{}\textcolor{ForestGreen}{int}\ n\textcolor{BrickRed}{;} \\
\mbox{} \\
\mbox{}\textcolor{TealBlue}{vector$<$bool$>$}\ used\textcolor{BrickRed}{;} \\
\mbox{} \\
\mbox{}\textcolor{TealBlue}{list$<$int$>$}\ ans\textcolor{BrickRed}{;} \\
\mbox{} \\
\mbox{}\textcolor{ForestGreen}{void}\ \textbf{\textcolor{Black}{dfs}}\textcolor{BrickRed}{(}\textcolor{ForestGreen}{int}\ v\textcolor{BrickRed}{)} \\
\mbox{}\textcolor{Red}{\{} \\
\mbox{}\ \ used\textcolor{BrickRed}{[}v\textcolor{BrickRed}{]}\ \textcolor{BrickRed}{=}\ \textbf{\textcolor{Blue}{true}}\textcolor{BrickRed}{;} \\
\mbox{}\ \ \textbf{\textcolor{Blue}{for}}\textcolor{BrickRed}{(}vector\textcolor{BrickRed}{$<$}\textcolor{ForestGreen}{int}\textcolor{BrickRed}{$>$::}\textcolor{TealBlue}{itetator}\ i\textcolor{BrickRed}{=}g\textcolor{BrickRed}{[}v\textcolor{BrickRed}{].}\textbf{\textcolor{Black}{begin}}\textcolor{BrickRed}{();}\ i\textcolor{BrickRed}{!=}g\textcolor{BrickRed}{[}v\textcolor{BrickRed}{].}\textbf{\textcolor{Black}{end}}\textcolor{BrickRed}{();}\ \textcolor{BrickRed}{++}i\textcolor{BrickRed}{)} \\
\mbox{}\ \ \ \ \textbf{\textcolor{Blue}{if}}\textcolor{BrickRed}{(!}used\textcolor{BrickRed}{[*}i\textcolor{BrickRed}{])} \\
\mbox{}\ \ \ \ \ \ \textbf{\textcolor{Black}{dfs}}\textcolor{BrickRed}{(*}i\textcolor{BrickRed}{);} \\
\mbox{}\ \ ans\textcolor{BrickRed}{.}\textbf{\textcolor{Black}{push$\_$front}}\textcolor{BrickRed}{(}v\textcolor{BrickRed}{);} \\
\mbox{}\textcolor{Red}{\}} \\
\mbox{} \\
\mbox{}\textcolor{ForestGreen}{void}\ \textbf{\textcolor{Black}{topological$\_$sort}}\textcolor{BrickRed}{(}list\textcolor{BrickRed}{$<$}\textcolor{ForestGreen}{int}\textcolor{BrickRed}{$>$}\ \textcolor{BrickRed}{\&}\ result\textcolor{BrickRed}{)} \\
\mbox{}\textcolor{Red}{\{} \\
\mbox{}\ \ used\textcolor{BrickRed}{.}\textbf{\textcolor{Black}{assign}}\textcolor{BrickRed}{(}n\textcolor{BrickRed}{,}\ \textbf{\textcolor{Blue}{false}}\textcolor{BrickRed}{);} \\
\mbox{}\ \ \textbf{\textcolor{Blue}{for}}\textcolor{BrickRed}{(}\textcolor{ForestGreen}{int}\ i\textcolor{BrickRed}{=}\textcolor{Purple}{0}\textcolor{BrickRed}{;}\ i\textcolor{BrickRed}{$<$}n\textcolor{BrickRed}{;}\ \textcolor{BrickRed}{++}i\textcolor{BrickRed}{)} \\
\mbox{}\ \ \ \ \textbf{\textcolor{Blue}{if}}\textcolor{BrickRed}{(!}used\textcolor{BrickRed}{[}i\textcolor{BrickRed}{])} \\
\mbox{}\ \ \ \ \ \ \textbf{\textcolor{Black}{dfs}}\textcolor{BrickRed}{(}i\textcolor{BrickRed}{);} \\
\mbox{}\ \ result\ \textcolor{BrickRed}{=}\ ans\textcolor{BrickRed}{;} \\
\mbox{}\textcolor{Red}{\}} \\
\mbox{}
} \normalfont\normalsize
%.tex

\subsection{Componentes fuertemente conectados}
{\ttfamily \raggedright {
% Generator: GNU source-highlight, by Lorenzo Bettini, http://www.gnu.org/software/src-highlite
\noindent
\mbox{}vector\ \textcolor{BrickRed}{$<$}\ \textcolor{TealBlue}{vector$<$int$>$\ $>$}\ g\textcolor{BrickRed}{,}\ gr\textcolor{BrickRed}{;} \\
\mbox{}\textcolor{TealBlue}{vector$<$char$>$}\ used\textcolor{BrickRed}{;} \\
\mbox{}\textcolor{TealBlue}{vector$<$int$>$}\ order\textcolor{BrickRed}{,}\ component\textcolor{BrickRed}{;} \\
\mbox{}\  \\
\mbox{}\textcolor{ForestGreen}{void}\ \textbf{\textcolor{Black}{dfs1}}\textcolor{BrickRed}{(}\textcolor{ForestGreen}{int}\ v\textcolor{BrickRed}{)}\ \textcolor{Red}{\{} \\
\mbox{}\ \ used\textcolor{BrickRed}{[}v\textcolor{BrickRed}{]}\ \textcolor{BrickRed}{=}\ \textbf{\textcolor{Blue}{true}}\textcolor{BrickRed}{;} \\
\mbox{}\ \ \textbf{\textcolor{Blue}{for}}\textcolor{BrickRed}{(}\textcolor{TealBlue}{size$\_$t}\ i\textcolor{BrickRed}{=}\textcolor{Purple}{0}\textcolor{BrickRed}{;}\ i\textcolor{BrickRed}{$<$}g\textcolor{BrickRed}{[}v\textcolor{BrickRed}{].}\textbf{\textcolor{Black}{size}}\textcolor{BrickRed}{();}\ \textcolor{BrickRed}{++}i\textcolor{BrickRed}{)} \\
\mbox{}\ \ \ \ \textbf{\textcolor{Blue}{if}}\textcolor{BrickRed}{(!}used\textcolor{BrickRed}{[}\ g\textcolor{BrickRed}{[}v\textcolor{BrickRed}{][}i\textcolor{BrickRed}{]}\ \textcolor{BrickRed}{])} \\
\mbox{}\ \ \ \ \ \ \textbf{\textcolor{Black}{dfs1}}\textcolor{BrickRed}{(}g\textcolor{BrickRed}{[}v\textcolor{BrickRed}{][}i\textcolor{BrickRed}{]);} \\
\mbox{}\ \ order\textcolor{BrickRed}{.}\textbf{\textcolor{Black}{push$\_$back}}\textcolor{BrickRed}{(}v\textcolor{BrickRed}{);} \\
\mbox{}\textcolor{Red}{\}} \\
\mbox{}\  \\
\mbox{}\textcolor{ForestGreen}{void}\ \textbf{\textcolor{Black}{dfs2}}\textcolor{BrickRed}{(}\textcolor{ForestGreen}{int}\ v\textcolor{BrickRed}{)}\textcolor{Red}{\{} \\
\mbox{}\ \ used\textcolor{BrickRed}{[}v\textcolor{BrickRed}{]}\ \textcolor{BrickRed}{=}\ \textbf{\textcolor{Blue}{true}}\textcolor{BrickRed}{;} \\
\mbox{}\ \ component\textcolor{BrickRed}{.}\textbf{\textcolor{Black}{push$\_$back}}\ \textcolor{BrickRed}{(}v\textcolor{BrickRed}{);} \\
\mbox{}\ \ \textbf{\textcolor{Blue}{for}}\textcolor{BrickRed}{(}\textcolor{TealBlue}{size$\_$t}\ i\textcolor{BrickRed}{=}\textcolor{Purple}{0}\textcolor{BrickRed}{;}\ i\textcolor{BrickRed}{$<$}gr\textcolor{BrickRed}{[}v\textcolor{BrickRed}{].}\textbf{\textcolor{Black}{size}}\textcolor{BrickRed}{();}\ \textcolor{BrickRed}{++}i\textcolor{BrickRed}{)} \\
\mbox{}\ \ \ \ \textbf{\textcolor{Blue}{if}}\textcolor{BrickRed}{(!}used\textcolor{BrickRed}{[}\ gr\textcolor{BrickRed}{[}v\textcolor{BrickRed}{][}i\textcolor{BrickRed}{]}\ \textcolor{BrickRed}{])} \\
\mbox{}\ \ \ \ \ \ \textbf{\textcolor{Black}{dfs2}}\textcolor{BrickRed}{(}gr\textcolor{BrickRed}{[}v\textcolor{BrickRed}{][}i\textcolor{BrickRed}{]);} \\
\mbox{}\textcolor{Red}{\}} \\
\mbox{}\  \\
\mbox{}\textcolor{ForestGreen}{int}\ \textbf{\textcolor{Black}{main}}\textcolor{BrickRed}{()}\ \textcolor{Red}{\{} \\
\mbox{}\ \ \textcolor{ForestGreen}{int}\ n\textcolor{BrickRed}{;} \\
\mbox{}\ \ \textit{\textcolor{Brown}{//...\ read\ n\ ...}} \\
\mbox{}\ \ \textbf{\textcolor{Blue}{for}}\textcolor{BrickRed}{(;;)}\ \textcolor{Red}{\{} \\
\mbox{}\ \ \ \ \textcolor{ForestGreen}{int}\ a\textcolor{BrickRed}{,}\ b\textcolor{BrickRed}{;} \\
\mbox{}\ \ \ \ \textit{\textcolor{Brown}{//...\ read\ directed\ edge\ (a,b)\ ...}} \\
\mbox{}\ \ \ \ g\textcolor{BrickRed}{[}a\textcolor{BrickRed}{].}\textbf{\textcolor{Black}{push$\_$back}}\textcolor{BrickRed}{(}b\textcolor{BrickRed}{);} \\
\mbox{}\ \ \ \ gr\textcolor{BrickRed}{[}b\textcolor{BrickRed}{].}\textbf{\textcolor{Black}{push$\_$back}}\textcolor{BrickRed}{(}a\textcolor{BrickRed}{);} \\
\mbox{}\ \ \textcolor{Red}{\}} \\
\mbox{}\  \\
\mbox{}\ \ used\textcolor{BrickRed}{.}\textbf{\textcolor{Black}{assign}}\textcolor{BrickRed}{(}n\textcolor{BrickRed}{,}\ \textbf{\textcolor{Blue}{false}}\textcolor{BrickRed}{);} \\
\mbox{}\ \ \textbf{\textcolor{Blue}{for}}\textcolor{BrickRed}{(}\textcolor{ForestGreen}{int}\ i\textcolor{BrickRed}{=}\textcolor{Purple}{0}\textcolor{BrickRed}{;}\ i\textcolor{BrickRed}{$<$}n\textcolor{BrickRed}{;}\ \textcolor{BrickRed}{++}i\textcolor{BrickRed}{)} \\
\mbox{}\ \ \ \ \textbf{\textcolor{Blue}{if}}\textcolor{BrickRed}{(!}used\textcolor{BrickRed}{[}i\textcolor{BrickRed}{])} \\
\mbox{}\ \ \ \ \ \ \textbf{\textcolor{Black}{dfs1}}\textcolor{BrickRed}{(}i\textcolor{BrickRed}{);} \\
\mbox{}\ \ used\textcolor{BrickRed}{.}\textbf{\textcolor{Black}{assign}}\textcolor{BrickRed}{(}n\textcolor{BrickRed}{,}\ \textbf{\textcolor{Blue}{false}}\textcolor{BrickRed}{);} \\
\mbox{}\ \ \textbf{\textcolor{Blue}{for}}\textcolor{BrickRed}{(}\textcolor{ForestGreen}{int}\ i\textcolor{BrickRed}{=}\textcolor{Purple}{0}\textcolor{BrickRed}{;}\ i\textcolor{BrickRed}{$<$}n\textcolor{BrickRed}{;}\ \textcolor{BrickRed}{++}i\textcolor{BrickRed}{)}\ \textcolor{Red}{\{} \\
\mbox{}\ \ \ \ \textcolor{ForestGreen}{int}\ v\ \textcolor{BrickRed}{=}\ order\textcolor{BrickRed}{[}n\textcolor{BrickRed}{-}\textcolor{Purple}{1}\textcolor{BrickRed}{-}i\textcolor{BrickRed}{];} \\
\mbox{}\ \ \ \ \textbf{\textcolor{Blue}{if}}\textcolor{BrickRed}{(!}used\textcolor{BrickRed}{[}v\textcolor{BrickRed}{])}\ \textcolor{Red}{\{} \\
\mbox{}\ \ \ \ \ \ \textbf{\textcolor{Black}{dfs2}}\textcolor{BrickRed}{(}v\textcolor{BrickRed}{);} \\
\mbox{}\ \ \ \ \ \ \textit{\textcolor{Brown}{//...\ work\ with\ component\ ...}} \\
\mbox{}\ \ \ \ \ \ component\textcolor{BrickRed}{.}\textbf{\textcolor{Black}{clear}}\textcolor{BrickRed}{();} \\
\mbox{}\ \ \ \ \textcolor{Red}{\}} \\
\mbox{}\ \ \textcolor{Red}{\}} \\
\mbox{}\textcolor{Red}{\}} \\
\mbox{} \\
\mbox{}
} \normalfont\normalsize
%.tex

\subsection{K camino mas corto}
{\ttfamily \raggedright {
% Generator: GNU source-highlight, by Lorenzo Bettini, http://www.gnu.org/software/src-highlite
\noindent
\mbox{}\textbf{\textcolor{Blue}{const}}\ \textcolor{ForestGreen}{int}\ INF\ \textcolor{BrickRed}{=}\ \textcolor{Purple}{1000}\textcolor{BrickRed}{*}\textcolor{Purple}{1000}\textcolor{BrickRed}{*}\textcolor{Purple}{1000}\textcolor{BrickRed}{;} \\
\mbox{}\textbf{\textcolor{Blue}{const}}\ \textcolor{ForestGreen}{int}\ W\ \textcolor{BrickRed}{=}\ \textcolor{BrickRed}{...;}\ \textit{\textcolor{Brown}{//\ peso\ maximo}} \\
\mbox{} \\
\mbox{}\textcolor{ForestGreen}{int}\ n\textcolor{BrickRed}{,}\ s\textcolor{BrickRed}{,}\ t\textcolor{BrickRed}{;} \\
\mbox{}vector\ \textcolor{BrickRed}{$<$}\ vector\ \textcolor{BrickRed}{$<$}\ \textcolor{TealBlue}{pair$<$int,int$>$\ $>$\ $>$}\ g\textcolor{BrickRed}{;} \\
\mbox{}\textcolor{TealBlue}{vector$<$int$>$}\ dist\textcolor{BrickRed}{;} \\
\mbox{}\textcolor{TealBlue}{vector$<$char$>$}\ used\textcolor{BrickRed}{;} \\
\mbox{}\textcolor{TealBlue}{vector$<$int$>$}\ curpath\textcolor{BrickRed}{,}\ kth$\_$path\textcolor{BrickRed}{;} \\
\mbox{} \\
\mbox{}\textcolor{ForestGreen}{int}\ \textbf{\textcolor{Black}{kth$\_$path$\_$exists}}\textcolor{BrickRed}{(}\textcolor{ForestGreen}{int}\ k\textcolor{BrickRed}{,}\ \textcolor{ForestGreen}{int}\ maxlen\textcolor{BrickRed}{,}\ \textcolor{ForestGreen}{int}\ v\textcolor{BrickRed}{,}\ \textcolor{ForestGreen}{int}\ curlen\ \textcolor{BrickRed}{=}\ \textcolor{Purple}{0}\textcolor{BrickRed}{)}\ \textcolor{Red}{\{} \\
\mbox{}\ \ curpath\textcolor{BrickRed}{.}\textbf{\textcolor{Black}{push$\_$back}}\textcolor{BrickRed}{(}v\textcolor{BrickRed}{);} \\
\mbox{}\ \ \textbf{\textcolor{Blue}{if}}\textcolor{BrickRed}{(}v\ \textcolor{BrickRed}{==}\ t\textcolor{BrickRed}{)}\ \textcolor{Red}{\{} \\
\mbox{}\ \ \ \ \textbf{\textcolor{Blue}{if}}\textcolor{BrickRed}{(}curlen\ \textcolor{BrickRed}{==}\ maxlen\textcolor{BrickRed}{)} \\
\mbox{}\ \ \ \ \ \ kth$\_$path\ \textcolor{BrickRed}{=}\ curpath\textcolor{BrickRed}{;} \\
\mbox{}\ \ \ \ curpath\textcolor{BrickRed}{.}\textbf{\textcolor{Black}{pop$\_$back}}\textcolor{BrickRed}{();} \\
\mbox{}\ \ \ \ \textbf{\textcolor{Blue}{return}}\ \textcolor{Purple}{1}\textcolor{BrickRed}{;} \\
\mbox{}\ \ \textcolor{Red}{\}} \\
\mbox{}\ \ used\textcolor{BrickRed}{[}v\textcolor{BrickRed}{]}\ \textcolor{BrickRed}{=}\ \textbf{\textcolor{Blue}{true}}\textcolor{BrickRed}{;} \\
\mbox{}\ \ \textcolor{ForestGreen}{int}\ found\ \textcolor{BrickRed}{=}\ \textcolor{Purple}{0}\textcolor{BrickRed}{;} \\
\mbox{}\ \ \textbf{\textcolor{Blue}{for}}\textcolor{BrickRed}{(}\textcolor{TealBlue}{size$\_$t}\ i\textcolor{BrickRed}{=}\textcolor{Purple}{0}\textcolor{BrickRed}{;}\ i\textcolor{BrickRed}{$<$}g\textcolor{BrickRed}{[}v\textcolor{BrickRed}{].}\textbf{\textcolor{Black}{size}}\textcolor{BrickRed}{();}\ \textcolor{BrickRed}{++}i\textcolor{BrickRed}{)}\ \textcolor{Red}{\{} \\
\mbox{}\ \ \ \ \textcolor{ForestGreen}{int}\ to\ \textcolor{BrickRed}{=}\ g\textcolor{BrickRed}{[}v\textcolor{BrickRed}{][}i\textcolor{BrickRed}{].}first\textcolor{BrickRed}{,}\ \ len\ \textcolor{BrickRed}{=}\ g\textcolor{BrickRed}{[}v\textcolor{BrickRed}{][}i\textcolor{BrickRed}{].}second\textcolor{BrickRed}{;} \\
\mbox{}\ \ \ \ \textbf{\textcolor{Blue}{if}}\textcolor{BrickRed}{(!}used\textcolor{BrickRed}{[}to\textcolor{BrickRed}{]}\ \textcolor{BrickRed}{\&\&}\ curlen\ \textcolor{BrickRed}{+}\ len\ \textcolor{BrickRed}{+}\ dist\textcolor{BrickRed}{[}to\textcolor{BrickRed}{]}\ \textcolor{BrickRed}{$<$=}\ maxlen\textcolor{BrickRed}{)}\ \textcolor{Red}{\{} \\
\mbox{}\ \ \ \ \ \ found\ \textcolor{BrickRed}{+=}\ \textbf{\textcolor{Black}{kth$\_$path$\_$exists}}\textcolor{BrickRed}{(}k\ \textcolor{BrickRed}{-}\ found\textcolor{BrickRed}{,}\ maxlen\textcolor{BrickRed}{,}\ to\textcolor{BrickRed}{,}\ curlen\ \textcolor{BrickRed}{+}\ len\textcolor{BrickRed}{);} \\
\mbox{}\ \ \ \ \ \ \textbf{\textcolor{Blue}{if}}\textcolor{BrickRed}{(}found\ \textcolor{BrickRed}{==}\ k\textcolor{BrickRed}{)}\ \ \textbf{\textcolor{Blue}{break}}\textcolor{BrickRed}{;} \\
\mbox{}\ \ \ \ \textcolor{Red}{\}} \\
\mbox{}\ \ \textcolor{Red}{\}} \\
\mbox{}\ \ used\textcolor{BrickRed}{[}v\textcolor{BrickRed}{]}\ \textcolor{BrickRed}{=}\ \textbf{\textcolor{Blue}{false}}\textcolor{BrickRed}{;} \\
\mbox{}\ \ curpath\textcolor{BrickRed}{.}\textbf{\textcolor{Black}{pop$\_$back}}\textcolor{BrickRed}{();} \\
\mbox{}\ \ \textbf{\textcolor{Blue}{return}}\ found\textcolor{BrickRed}{;} \\
\mbox{}\textcolor{Red}{\}} \\
\mbox{} \\
\mbox{} \\
\mbox{}\textcolor{ForestGreen}{int}\ \textbf{\textcolor{Black}{main}}\textcolor{BrickRed}{()}\ \textcolor{Red}{\{} \\
\mbox{} \\
\mbox{}\ \ \textit{\textcolor{Brown}{//...\ inicializar\ (n,\ k,\ g,\ s,\ t)\ ...}} \\
\mbox{} \\
\mbox{}\ \ dist\textcolor{BrickRed}{.}\textbf{\textcolor{Black}{assign}}\textcolor{BrickRed}{(}n\textcolor{BrickRed}{,}\ INF\textcolor{BrickRed}{);} \\
\mbox{}\ \ dist\textcolor{BrickRed}{[}t\textcolor{BrickRed}{]}\ \textcolor{BrickRed}{=}\ \textcolor{Purple}{0}\textcolor{BrickRed}{;} \\
\mbox{}\ \ used\textcolor{BrickRed}{.}\textbf{\textcolor{Black}{assign}}\textcolor{BrickRed}{(}n\textcolor{BrickRed}{,}\ \textbf{\textcolor{Blue}{false}}\textcolor{BrickRed}{);} \\
\mbox{}\ \ \textbf{\textcolor{Blue}{for}}\textcolor{BrickRed}{(;;)}\ \textcolor{Red}{\{} \\
\mbox{}\ \ \ \ \textcolor{ForestGreen}{int}\ sel\ \textcolor{BrickRed}{=}\ \textcolor{BrickRed}{-}\textcolor{Purple}{1}\textcolor{BrickRed}{;} \\
\mbox{}\ \ \ \ \textbf{\textcolor{Blue}{for}}\textcolor{BrickRed}{(}\textcolor{ForestGreen}{int}\ i\textcolor{BrickRed}{=}\textcolor{Purple}{0}\textcolor{BrickRed}{;}\ i\textcolor{BrickRed}{$<$}n\textcolor{BrickRed}{;}\ \textcolor{BrickRed}{++}i\textcolor{BrickRed}{)} \\
\mbox{}\ \ \ \ \ \ \textbf{\textcolor{Blue}{if}}\textcolor{BrickRed}{(!}used\textcolor{BrickRed}{[}i\textcolor{BrickRed}{]}\ \textcolor{BrickRed}{\&\&}\ dist\textcolor{BrickRed}{[}i\textcolor{BrickRed}{]}\ \textcolor{BrickRed}{$<$}\ INF\ \textcolor{BrickRed}{\&\&}\ \textcolor{BrickRed}{(}sel\ \textcolor{BrickRed}{==}\ \textcolor{BrickRed}{-}\textcolor{Purple}{1}\ \textcolor{BrickRed}{$|$$|$}\ dist\textcolor{BrickRed}{[}i\textcolor{BrickRed}{]}\ \textcolor{BrickRed}{$<$}\ dist\textcolor{BrickRed}{[}sel\textcolor{BrickRed}{]))} \\
\mbox{}\ \ \ \ \ \ \ \ sel\ \textcolor{BrickRed}{=}\ i\textcolor{BrickRed}{;} \\
\mbox{}\ \ \ \ \textbf{\textcolor{Blue}{if}}\textcolor{BrickRed}{(}sel\ \textcolor{BrickRed}{==}\ \textcolor{BrickRed}{-}\textcolor{Purple}{1}\textcolor{BrickRed}{)}\ \ \textbf{\textcolor{Blue}{break}}\textcolor{BrickRed}{;} \\
\mbox{}\ \ \ \ used\textcolor{BrickRed}{[}sel\textcolor{BrickRed}{]}\ \textcolor{BrickRed}{=}\ \textbf{\textcolor{Blue}{true}}\textcolor{BrickRed}{;} \\
\mbox{}\ \ \ \ \textbf{\textcolor{Blue}{for}}\textcolor{BrickRed}{(}\textcolor{TealBlue}{size$\_$t}\ i\textcolor{BrickRed}{=}\textcolor{Purple}{0}\textcolor{BrickRed}{;}\ i\textcolor{BrickRed}{$<$}g\textcolor{BrickRed}{[}sel\textcolor{BrickRed}{].}\textbf{\textcolor{Black}{size}}\textcolor{BrickRed}{();}\ \textcolor{BrickRed}{++}i\textcolor{BrickRed}{)}\ \textcolor{Red}{\{} \\
\mbox{}\ \ \ \ \ \ \textcolor{ForestGreen}{int}\ to\ \textcolor{BrickRed}{=}\ g\textcolor{BrickRed}{[}sel\textcolor{BrickRed}{][}i\textcolor{BrickRed}{].}first\textcolor{BrickRed}{,}\ \ len\ \textcolor{BrickRed}{=}\ g\textcolor{BrickRed}{[}sel\textcolor{BrickRed}{][}i\textcolor{BrickRed}{].}second\textcolor{BrickRed}{;} \\
\mbox{}\ \ \ \ \ \ dist\textcolor{BrickRed}{[}to\textcolor{BrickRed}{]}\ \textcolor{BrickRed}{=}\ \textbf{\textcolor{Black}{min}}\ \textcolor{BrickRed}{(}dist\textcolor{BrickRed}{[}to\textcolor{BrickRed}{],}\ dist\textcolor{BrickRed}{[}sel\textcolor{BrickRed}{]}\ \textcolor{BrickRed}{+}\ len\textcolor{BrickRed}{);} \\
\mbox{}\ \ \ \ \textcolor{Red}{\}} \\
\mbox{}\ \ \textcolor{Red}{\}} \\
\mbox{} \\
\mbox{}\ \ \textcolor{ForestGreen}{int}\ minw\ \textcolor{BrickRed}{=}\ \textcolor{Purple}{0}\textcolor{BrickRed}{,}\ \ maxw\ \textcolor{BrickRed}{=}\ W\textcolor{BrickRed}{;} \\
\mbox{}\ \ \textbf{\textcolor{Blue}{while}}\textcolor{BrickRed}{(}minw\ \textcolor{BrickRed}{$<$}\ maxw\textcolor{BrickRed}{)}\ \textcolor{Red}{\{} \\
\mbox{}\ \ \ \ \textcolor{ForestGreen}{int}\ wlimit\ \textcolor{BrickRed}{=}\ \textcolor{BrickRed}{(}minw\ \textcolor{BrickRed}{+}\ maxw\textcolor{BrickRed}{)}\ \textcolor{BrickRed}{$>$$>$}\ \textcolor{Purple}{1}\textcolor{BrickRed}{;} \\
\mbox{}\ \ \ \ used\textcolor{BrickRed}{.}\textbf{\textcolor{Black}{assign}}\textcolor{BrickRed}{(}n\textcolor{BrickRed}{,}\ \textbf{\textcolor{Blue}{false}}\textcolor{BrickRed}{);} \\
\mbox{}\ \ \ \ \textbf{\textcolor{Blue}{if}}\textcolor{BrickRed}{(}\textbf{\textcolor{Black}{kth$\_$path$\_$exists}}\textcolor{BrickRed}{(}k\textcolor{BrickRed}{,}\ wlimit\textcolor{BrickRed}{,}\ s\textcolor{BrickRed}{)}\ \textcolor{BrickRed}{==}\ k\textcolor{BrickRed}{)} \\
\mbox{}\ \ \ \ \ \ maxw\ \textcolor{BrickRed}{=}\ wlimit\textcolor{BrickRed}{;} \\
\mbox{}\ \ \ \ \textbf{\textcolor{Blue}{else}} \\
\mbox{}\ \ \ \ \ \ minw\ \textcolor{BrickRed}{=}\ wlimit\ \textcolor{BrickRed}{+}\ \textcolor{Purple}{1}\textcolor{BrickRed}{;} \\
\mbox{}\ \ \textcolor{Red}{\}} \\
\mbox{} \\
\mbox{}\ \ used\textcolor{BrickRed}{.}\textbf{\textcolor{Black}{assign}}\textcolor{BrickRed}{(}n\textcolor{BrickRed}{,}\ \textbf{\textcolor{Blue}{false}}\textcolor{BrickRed}{);} \\
\mbox{}\ \ \textbf{\textcolor{Blue}{if}}\textcolor{BrickRed}{(}\textbf{\textcolor{Black}{kth$\_$path$\_$exists}}\textcolor{BrickRed}{(}k\textcolor{BrickRed}{,}\ minw\textcolor{BrickRed}{,}\ s\textcolor{BrickRed}{)}\ \textcolor{BrickRed}{$<$}\ k\textcolor{BrickRed}{)} \\
\mbox{}\ \ \ \ \textbf{\textcolor{Black}{puts}}\textcolor{BrickRed}{(}\texttt{\textcolor{Red}{"{}NO\ SOLUTION"{}}}\textcolor{BrickRed}{);} \\
\mbox{}\ \ \textbf{\textcolor{Blue}{else}}\ \textcolor{Red}{\{} \\
\mbox{}\ \ \ \ cout\ \textcolor{BrickRed}{$<$$<$}\ minw\ \textcolor{BrickRed}{$<$$<$}\ \texttt{\textcolor{Red}{'\ '}}\ \textcolor{BrickRed}{$<$$<$}\ kth$\_$path\textcolor{BrickRed}{.}\textbf{\textcolor{Black}{size}}\textcolor{BrickRed}{()}\ \textcolor{BrickRed}{$<$$<$}\ endl\textcolor{BrickRed}{;} \\
\mbox{}\ \ \ \ \textbf{\textcolor{Blue}{for}}\textcolor{BrickRed}{(}\textcolor{TealBlue}{size$\_$t}\ i\textcolor{BrickRed}{=}\textcolor{Purple}{0}\textcolor{BrickRed}{;}\ i\textcolor{BrickRed}{$<$}kth$\_$path\textcolor{BrickRed}{.}\textbf{\textcolor{Black}{size}}\textcolor{BrickRed}{();}\ \textcolor{BrickRed}{++}i\textcolor{BrickRed}{)} \\
\mbox{}\ \ \ \ \ \ cout\ \textcolor{BrickRed}{$<$$<$}\ kth$\_$path\textcolor{BrickRed}{[}i\textcolor{BrickRed}{]+}\textcolor{Purple}{1}\ \textcolor{BrickRed}{$<$$<$}\ \texttt{\textcolor{Red}{'\ '}}\textcolor{BrickRed}{;} \\
\mbox{}\ \ \textcolor{Red}{\}} \\
\mbox{} \\
\mbox{}\textcolor{Red}{\}} \\
\mbox{} \\
\mbox{}
} \normalfont\normalsize
%.tex

\subsection{Algoritmo de Dijkstra}
El peso de todas las aristas debe ser no negativo.
\\
% Generator: GNU source-highlight, by Lorenzo Bettini, http://www.gnu.org/software/src-highlite

{\ttfamily \raggedright {
\noindent
\mbox{}\textbf{\textcolor{RoyalBlue}{\#include}}\ \texttt{\textcolor{Red}{$<$iostream$>$}} \\
\mbox{}\textbf{\textcolor{RoyalBlue}{\#include}}\ \texttt{\textcolor{Red}{$<$algorithm$>$}} \\
\mbox{}\textbf{\textcolor{RoyalBlue}{\#include}}\ \texttt{\textcolor{Red}{$<$queue$>$}} \\
\mbox{} \\
\mbox{}\textbf{\textcolor{Blue}{using}}\ \textbf{\textcolor{Blue}{namespace}}\ std\textcolor{BrickRed}{;} \\
\mbox{} \\
\mbox{}\textbf{\textcolor{Blue}{struct}}\ edge\textcolor{Red}{\{} \\
\mbox{}\ \ \textcolor{ForestGreen}{int}\ to\textcolor{BrickRed}{,}\ weight\textcolor{BrickRed}{;} \\
\mbox{}\ \ \textbf{\textcolor{Black}{edge}}\textcolor{BrickRed}{()}\ \textcolor{Red}{\{\}} \\
\mbox{}\ \ \textbf{\textcolor{Black}{edge}}\textcolor{BrickRed}{(}\textcolor{ForestGreen}{int}\ t\textcolor{BrickRed}{,}\ \textcolor{ForestGreen}{int}\ w\textcolor{BrickRed}{)}\ \textcolor{BrickRed}{:}\ \textbf{\textcolor{Black}{to}}\textcolor{BrickRed}{(}t\textcolor{BrickRed}{),}\ \textbf{\textcolor{Black}{weight}}\textcolor{BrickRed}{(}w\textcolor{BrickRed}{)}\ \textcolor{Red}{\{\}} \\
\mbox{}\ \ \textcolor{ForestGreen}{bool}\ \textbf{\textcolor{Blue}{operator}}\ \textcolor{BrickRed}{$<$}\ \textcolor{BrickRed}{(}\textbf{\textcolor{Blue}{const}}\ edge\ \textcolor{BrickRed}{\&}that\textcolor{BrickRed}{)}\ \textbf{\textcolor{Blue}{const}}\ \textcolor{Red}{\{} \\
\mbox{}\ \ \ \ \textbf{\textcolor{Blue}{return}}\ weight\ \textcolor{BrickRed}{$>$}\ that\textcolor{BrickRed}{.}weight\textcolor{BrickRed}{;} \\
\mbox{}\ \ \textcolor{Red}{\}} \\
\mbox{}\textcolor{Red}{\}}\textcolor{BrickRed}{;} \\
\mbox{} \\
\mbox{}\textcolor{ForestGreen}{int}\ \textbf{\textcolor{Black}{main}}\textcolor{BrickRed}{()}\textcolor{Red}{\{} \\
\mbox{}\ \ \textcolor{ForestGreen}{int}\ N\textcolor{BrickRed}{,}\ C\textcolor{BrickRed}{=}\textcolor{Purple}{0}\textcolor{BrickRed}{;} \\
\mbox{}\ \ \textbf{\textcolor{Black}{scanf}}\textcolor{BrickRed}{(}\texttt{\textcolor{Red}{"{}\%d"{}}}\textcolor{BrickRed}{,}\ \textcolor{BrickRed}{\&}N\textcolor{BrickRed}{);} \\
\mbox{}\ \ \textbf{\textcolor{Blue}{while}}\ \textcolor{BrickRed}{(}N\textcolor{BrickRed}{-\/-}\ \textcolor{BrickRed}{\&\&}\ \textcolor{BrickRed}{++}C\textcolor{BrickRed}{)}\textcolor{Red}{\{} \\
\mbox{}\ \ \ \ \textcolor{ForestGreen}{int}\ n\textcolor{BrickRed}{,}\ m\textcolor{BrickRed}{,}\ s\textcolor{BrickRed}{,}\ t\textcolor{BrickRed}{;} \\
\mbox{}\ \ \ \ \textbf{\textcolor{Black}{scanf}}\textcolor{BrickRed}{(}\texttt{\textcolor{Red}{"{}\%d\ \%d\ \%d\ \%d"{}}}\textcolor{BrickRed}{,}\ \textcolor{BrickRed}{\&}n\textcolor{BrickRed}{,}\ \textcolor{BrickRed}{\&}m\textcolor{BrickRed}{,}\ \textcolor{BrickRed}{\&}s\textcolor{BrickRed}{,}\ \textcolor{BrickRed}{\&}t\textcolor{BrickRed}{);} \\
\mbox{}\ \ \ \ vector\textcolor{BrickRed}{$<$}edge\textcolor{BrickRed}{$>$}\ g\textcolor{BrickRed}{[}n\textcolor{BrickRed}{];} \\
\mbox{}\ \ \ \ \textbf{\textcolor{Blue}{while}}\ \textcolor{BrickRed}{(}m\textcolor{BrickRed}{-\/-)}\textcolor{Red}{\{} \\
\mbox{}\ \ \ \ \ \ \textcolor{ForestGreen}{int}\ u\textcolor{BrickRed}{,}\ v\textcolor{BrickRed}{,}\ w\textcolor{BrickRed}{;} \\
\mbox{}\ \ \ \ \ \ \textbf{\textcolor{Black}{scanf}}\textcolor{BrickRed}{(}\texttt{\textcolor{Red}{"{}\%d\ \%d\ \%d"{}}}\textcolor{BrickRed}{,}\ \textcolor{BrickRed}{\&}u\textcolor{BrickRed}{,}\ \textcolor{BrickRed}{\&}v\textcolor{BrickRed}{,}\ \textcolor{BrickRed}{\&}w\textcolor{BrickRed}{);} \\
\mbox{}\ \ \ \ \ \ g\textcolor{BrickRed}{[}u\textcolor{BrickRed}{].}\textbf{\textcolor{Black}{push$\_$back}}\textcolor{BrickRed}{(}\textbf{\textcolor{Black}{edge}}\textcolor{BrickRed}{(}v\textcolor{BrickRed}{,}\ w\textcolor{BrickRed}{));} \\
\mbox{}\ \ \ \ \ \ g\textcolor{BrickRed}{[}v\textcolor{BrickRed}{].}\textbf{\textcolor{Black}{push$\_$back}}\textcolor{BrickRed}{(}\textbf{\textcolor{Black}{edge}}\textcolor{BrickRed}{(}u\textcolor{BrickRed}{,}\ w\textcolor{BrickRed}{));} \\
\mbox{}\ \ \ \ \textcolor{Red}{\}} \\
\mbox{} \\
\mbox{}\ \ \ \ \textcolor{ForestGreen}{int}\ d\textcolor{BrickRed}{[}n\textcolor{BrickRed}{];} \\
\mbox{}\ \ \ \ \textbf{\textcolor{Blue}{for}}\ \textcolor{BrickRed}{(}\textcolor{ForestGreen}{int}\ i\textcolor{BrickRed}{=}\textcolor{Purple}{0}\textcolor{BrickRed}{;}\ i\textcolor{BrickRed}{$<$}n\textcolor{BrickRed}{;}\ \textcolor{BrickRed}{++}i\textcolor{BrickRed}{)}\ d\textcolor{BrickRed}{[}i\textcolor{BrickRed}{]}\ \textcolor{BrickRed}{=}\ INT$\_$MAX\textcolor{BrickRed}{;} \\
\mbox{}\ \ \ \ d\textcolor{BrickRed}{[}s\textcolor{BrickRed}{]}\ \textcolor{BrickRed}{=}\ \textcolor{Purple}{0}\textcolor{BrickRed}{;} \\
\mbox{}\ \ \ \ priority$\_$queue\textcolor{BrickRed}{$<$}edge\textcolor{BrickRed}{$>$}\ q\textcolor{BrickRed}{;} \\
\mbox{}\ \ \ \ q\textcolor{BrickRed}{.}\textbf{\textcolor{Black}{push}}\textcolor{BrickRed}{(}\textbf{\textcolor{Black}{edge}}\textcolor{BrickRed}{(}s\textcolor{BrickRed}{,}\ \textcolor{Purple}{0}\textcolor{BrickRed}{));} \\
\mbox{}\ \ \ \ \textbf{\textcolor{Blue}{while}}\ \textcolor{BrickRed}{(}q\textcolor{BrickRed}{.}\textbf{\textcolor{Black}{empty}}\textcolor{BrickRed}{()}\ \textcolor{BrickRed}{==}\ \textbf{\textcolor{Blue}{false}}\textcolor{BrickRed}{)}\textcolor{Red}{\{} \\
\mbox{}\ \ \ \ \ \ \textcolor{ForestGreen}{int}\ node\ \textcolor{BrickRed}{=}\ q\textcolor{BrickRed}{.}\textbf{\textcolor{Black}{top}}\textcolor{BrickRed}{().}to\textcolor{BrickRed}{;} \\
\mbox{}\ \ \ \ \ \ \textcolor{ForestGreen}{int}\ dist\ \textcolor{BrickRed}{=}\ q\textcolor{BrickRed}{.}\textbf{\textcolor{Black}{top}}\textcolor{BrickRed}{().}weight\textcolor{BrickRed}{;} \\
\mbox{}\ \ \ \ \ \ q\textcolor{BrickRed}{.}\textbf{\textcolor{Black}{pop}}\textcolor{BrickRed}{();} \\
\mbox{} \\
\mbox{}\ \ \ \ \ \ \textbf{\textcolor{Blue}{if}}\ \textcolor{BrickRed}{(}dist\ \textcolor{BrickRed}{$>$}\ d\textcolor{BrickRed}{[}node\textcolor{BrickRed}{])}\ \textbf{\textcolor{Blue}{continue}}\textcolor{BrickRed}{;} \\
\mbox{}\ \ \ \ \ \ \textbf{\textcolor{Blue}{if}}\ \textcolor{BrickRed}{(}node\ \textcolor{BrickRed}{==}\ t\textcolor{BrickRed}{)}\ \textbf{\textcolor{Blue}{break}}\textcolor{BrickRed}{;} \\
\mbox{} \\
\mbox{}\ \ \ \ \ \ \textbf{\textcolor{Blue}{for}}\ \textcolor{BrickRed}{(}\textcolor{ForestGreen}{int}\ i\textcolor{BrickRed}{=}\textcolor{Purple}{0}\textcolor{BrickRed}{;}\ i\textcolor{BrickRed}{$<$}g\textcolor{BrickRed}{[}node\textcolor{BrickRed}{].}\textbf{\textcolor{Black}{size}}\textcolor{BrickRed}{();}\ \textcolor{BrickRed}{++}i\textcolor{BrickRed}{)}\textcolor{Red}{\{} \\
\mbox{}\ \ \ \ \ \ \ \ \textcolor{ForestGreen}{int}\ to\ \textcolor{BrickRed}{=}\ g\textcolor{BrickRed}{[}node\textcolor{BrickRed}{][}i\textcolor{BrickRed}{].}to\textcolor{BrickRed}{;} \\
\mbox{}\ \ \ \ \ \ \ \ \textcolor{ForestGreen}{int}\ w$\_$extra\ \textcolor{BrickRed}{=}\ g\textcolor{BrickRed}{[}node\textcolor{BrickRed}{][}i\textcolor{BrickRed}{].}weight\textcolor{BrickRed}{;} \\
\mbox{} \\
\mbox{}\ \ \ \ \ \ \ \ \textbf{\textcolor{Blue}{if}}\ \textcolor{BrickRed}{(}dist\ \textcolor{BrickRed}{+}\ w$\_$extra\ \textcolor{BrickRed}{$<$}\ d\textcolor{BrickRed}{[}to\textcolor{BrickRed}{])}\textcolor{Red}{\{} \\
\mbox{}\ \ \ \ \ \ \ \ \ \ d\textcolor{BrickRed}{[}to\textcolor{BrickRed}{]}\ \textcolor{BrickRed}{=}\ dist\ \textcolor{BrickRed}{+}\ w$\_$extra\textcolor{BrickRed}{;} \\
\mbox{}\ \ \ \ \ \ \ \ \ \ q\textcolor{BrickRed}{.}\textbf{\textcolor{Black}{push}}\textcolor{BrickRed}{(}\textbf{\textcolor{Black}{edge}}\textcolor{BrickRed}{(}to\textcolor{BrickRed}{,}\ d\textcolor{BrickRed}{[}to\textcolor{BrickRed}{]));} \\
\mbox{}\ \ \ \ \ \ \ \ \textcolor{Red}{\}} \\
\mbox{}\ \ \ \ \ \ \textcolor{Red}{\}} \\
\mbox{}\ \ \ \ \textcolor{Red}{\}} \\
\mbox{}\ \ \ \ \textbf{\textcolor{Black}{printf}}\textcolor{BrickRed}{(}\texttt{\textcolor{Red}{"{}Case\ \#\%d:\ "{}}}\textcolor{BrickRed}{,}\ C\textcolor{BrickRed}{);} \\
\mbox{}\ \ \ \ \textbf{\textcolor{Blue}{if}}\ \textcolor{BrickRed}{(}d\textcolor{BrickRed}{[}t\textcolor{BrickRed}{]}\ \textcolor{BrickRed}{$<$}\ INT$\_$MAX\textcolor{BrickRed}{)}\ \textbf{\textcolor{Black}{printf}}\textcolor{BrickRed}{(}\texttt{\textcolor{Red}{"{}\%d}}\texttt{\textcolor{CarnationPink}{\textbackslash{}n}}\texttt{\textcolor{Red}{"{}}}\textcolor{BrickRed}{,}\ d\textcolor{BrickRed}{[}t\textcolor{BrickRed}{]);} \\
\mbox{}\ \ \ \ \textbf{\textcolor{Blue}{else}}\ \textbf{\textcolor{Black}{printf}}\textcolor{BrickRed}{(}\texttt{\textcolor{Red}{"{}unreachable}}\texttt{\textcolor{CarnationPink}{\textbackslash{}n}}\texttt{\textcolor{Red}{"{}}}\textcolor{BrickRed}{);} \\
\mbox{}\ \ \textcolor{Red}{\}} \\
\mbox{}\ \ \textbf{\textcolor{Blue}{return}}\ \textcolor{Purple}{0}\textcolor{BrickRed}{;} \\
\mbox{}\textcolor{Red}{\}} \\

} \normalfont\normalsize
%.tex

\subsection{Minimum spanning tree: Algoritmo de Kruskal + Union-Find}
% Generator: GNU source-highlight, by Lorenzo Bettini, http://www.gnu.org/software/src-highlite

{\ttfamily \raggedright {
\noindent
\mbox{}\textbf{\textcolor{RoyalBlue}{\#include}}\ \texttt{\textcolor{Red}{$<$iostream$>$}} \\
\mbox{}\textbf{\textcolor{RoyalBlue}{\#include}}\ \texttt{\textcolor{Red}{$<$vector$>$}} \\
\mbox{}\textbf{\textcolor{RoyalBlue}{\#include}}\ \texttt{\textcolor{Red}{$<$algorithm$>$}} \\
\mbox{} \\
\mbox{}\textbf{\textcolor{Blue}{using}}\ \textbf{\textcolor{Blue}{namespace}}\ std\textcolor{BrickRed}{;} \\
\mbox{} \\
\mbox{}\textit{\textcolor{Brown}{/*}} \\
\mbox{}\textit{\textcolor{Brown}{Algoritmo\ de\ Kruskal,\ para\ encontrar\ el\ árbol\ de\ recubrimiento\ de\ mínima\ suma.}} \\
\mbox{}\textit{\textcolor{Brown}{*/}} \\
\mbox{} \\
\mbox{}\textbf{\textcolor{Blue}{struct}}\ edge\textcolor{Red}{\{} \\
\mbox{}\ \ \textcolor{ForestGreen}{int}\ start\textcolor{BrickRed}{,}\ end\textcolor{BrickRed}{,}\ weight\textcolor{BrickRed}{;} \\
\mbox{}\ \ \textcolor{ForestGreen}{bool}\ \textbf{\textcolor{Blue}{operator}}\ \textcolor{BrickRed}{$<$}\ \textcolor{BrickRed}{(}\textbf{\textcolor{Blue}{const}}\ edge\ \textcolor{BrickRed}{\&}that\textcolor{BrickRed}{)}\ \textbf{\textcolor{Blue}{const}}\ \textcolor{Red}{\{} \\
\mbox{}\ \ \ \ \textit{\textcolor{Brown}{//Si\ se\ desea\ encontrar\ el\ árbol\ de\ recubrimiento\ de\ máxima\ suma,\ cambiar\ el\ $<$\ por\ un\ $>$}} \\
\mbox{}\ \ \ \ \textbf{\textcolor{Blue}{return}}\ weight\ \textcolor{BrickRed}{$<$}\ that\textcolor{BrickRed}{.}weight\textcolor{BrickRed}{;} \\
\mbox{}\ \ \textcolor{Red}{\}} \\
\mbox{}\textcolor{Red}{\}}\textcolor{BrickRed}{;} \\
\mbox{} \\
\mbox{} \\
\mbox{}\textit{\textcolor{Brown}{/*\ Union\ find\ */}} \\
\mbox{}\textcolor{ForestGreen}{int}\ p\textcolor{BrickRed}{[}\textcolor{Purple}{10001}\textcolor{BrickRed}{],}\ rank\textcolor{BrickRed}{[}\textcolor{Purple}{10001}\textcolor{BrickRed}{];} \\
\mbox{}\textcolor{ForestGreen}{void}\ \textbf{\textcolor{Black}{make$\_$set}}\textcolor{BrickRed}{(}\textcolor{ForestGreen}{int}\ x\textcolor{BrickRed}{)}\textcolor{Red}{\{}\ p\textcolor{BrickRed}{[}x\textcolor{BrickRed}{]}\ \textcolor{BrickRed}{=}\ x\textcolor{BrickRed}{,}\ rank\textcolor{BrickRed}{[}x\textcolor{BrickRed}{]}\ \textcolor{BrickRed}{=}\ \textcolor{Purple}{0}\textcolor{BrickRed}{;}\ \textcolor{Red}{\}} \\
\mbox{}\textcolor{ForestGreen}{void}\ \textbf{\textcolor{Black}{link}}\textcolor{BrickRed}{(}\textcolor{ForestGreen}{int}\ x\textcolor{BrickRed}{,}\ \textcolor{ForestGreen}{int}\ y\textcolor{BrickRed}{)}\textcolor{Red}{\{}\ rank\textcolor{BrickRed}{[}x\textcolor{BrickRed}{]}\ \textcolor{BrickRed}{$>$}\ rank\textcolor{BrickRed}{[}y\textcolor{BrickRed}{]}\ \textcolor{BrickRed}{?}\ p\textcolor{BrickRed}{[}y\textcolor{BrickRed}{]}\ \textcolor{BrickRed}{=}\ x\ \textcolor{BrickRed}{:}\ p\textcolor{BrickRed}{[}x\textcolor{BrickRed}{]}\ \textcolor{BrickRed}{=}\ y\textcolor{BrickRed}{,}\ rank\textcolor{BrickRed}{[}x\textcolor{BrickRed}{]}\ \textcolor{BrickRed}{==}\ rank\textcolor{BrickRed}{[}y\textcolor{BrickRed}{]}\ \textcolor{BrickRed}{?}\ rank\textcolor{BrickRed}{[}y\textcolor{BrickRed}{]++:}\ \textcolor{Purple}{0}\textcolor{BrickRed}{;}\ \textcolor{Red}{\}} \\
\mbox{}\textcolor{ForestGreen}{int}\ \textbf{\textcolor{Black}{find$\_$set}}\textcolor{BrickRed}{(}\textcolor{ForestGreen}{int}\ x\textcolor{BrickRed}{)}\textcolor{Red}{\{}\ \textbf{\textcolor{Blue}{return}}\ x\ \textcolor{BrickRed}{!=}\ p\textcolor{BrickRed}{[}x\textcolor{BrickRed}{]}\ \textcolor{BrickRed}{?}\ p\textcolor{BrickRed}{[}x\textcolor{BrickRed}{]}\ \textcolor{BrickRed}{=}\ \textbf{\textcolor{Black}{find$\_$set}}\textcolor{BrickRed}{(}p\textcolor{BrickRed}{[}x\textcolor{BrickRed}{])}\ \textcolor{BrickRed}{:}\ p\textcolor{BrickRed}{[}x\textcolor{BrickRed}{];}\ \textcolor{Red}{\}} \\
\mbox{}\textcolor{ForestGreen}{void}\ \textbf{\textcolor{Black}{merge}}\textcolor{BrickRed}{(}\textcolor{ForestGreen}{int}\ x\textcolor{BrickRed}{,}\ \textcolor{ForestGreen}{int}\ y\textcolor{BrickRed}{)}\textcolor{Red}{\{}\ \textbf{\textcolor{Black}{link}}\textcolor{BrickRed}{(}\textbf{\textcolor{Black}{find$\_$set}}\textcolor{BrickRed}{(}x\textcolor{BrickRed}{),}\ \textbf{\textcolor{Black}{find$\_$set}}\textcolor{BrickRed}{(}y\textcolor{BrickRed}{));}\ \textcolor{Red}{\}} \\
\mbox{}\textit{\textcolor{Brown}{/*\ End\ union\ find\ */}} \\
\mbox{} \\
\mbox{} \\
\mbox{}\textcolor{ForestGreen}{int}\ \textbf{\textcolor{Black}{main}}\textcolor{BrickRed}{()}\textcolor{Red}{\{} \\
\mbox{}\ \ \textcolor{ForestGreen}{int}\ c\textcolor{BrickRed}{;} \\
\mbox{}\ \ cin\ \textcolor{BrickRed}{$>$$>$}\ c\textcolor{BrickRed}{;} \\
\mbox{}\ \ \textbf{\textcolor{Blue}{while}}\ \textcolor{BrickRed}{(}c\textcolor{BrickRed}{-\/-)}\textcolor{Red}{\{} \\
\mbox{}\ \ \ \ \textcolor{ForestGreen}{int}\ n\textcolor{BrickRed}{,}\ m\textcolor{BrickRed}{;} \\
\mbox{}\ \ \ \ cin\ \textcolor{BrickRed}{$>$$>$}\ n\ \textcolor{BrickRed}{$>$$>$}\ m\textcolor{BrickRed}{;} \\
\mbox{}\ \ \ \ vector\textcolor{BrickRed}{$<$}edge\textcolor{BrickRed}{$>$}\ e\textcolor{BrickRed}{;} \\
\mbox{}\ \ \ \ \textcolor{ForestGreen}{long}\ \textcolor{ForestGreen}{long}\ total\ \textcolor{BrickRed}{=}\ \textcolor{Purple}{0}\textcolor{BrickRed}{;} \\
\mbox{}\ \ \ \ \textbf{\textcolor{Blue}{while}}\ \textcolor{BrickRed}{(}m\textcolor{BrickRed}{-\/-)}\textcolor{Red}{\{} \\
\mbox{}\ \ \ \ \ \ edge\ t\textcolor{BrickRed}{;} \\
\mbox{}\ \ \ \ \ \ cin\ \textcolor{BrickRed}{$>$$>$}\ t\textcolor{BrickRed}{.}start\ \textcolor{BrickRed}{$>$$>$}\ t\textcolor{BrickRed}{.}end\ \textcolor{BrickRed}{$>$$>$}\ t\textcolor{BrickRed}{.}weight\textcolor{BrickRed}{;} \\
\mbox{}\ \ \ \ \ \ e\textcolor{BrickRed}{.}\textbf{\textcolor{Black}{push$\_$back}}\textcolor{BrickRed}{(}t\textcolor{BrickRed}{);} \\
\mbox{}\ \ \ \ \ \ total\ \textcolor{BrickRed}{+=}\ t\textcolor{BrickRed}{.}weight\textcolor{BrickRed}{;} \\
\mbox{}\ \ \ \ \textcolor{Red}{\}} \\
\mbox{}\ \ \ \ \textbf{\textcolor{Black}{sort}}\textcolor{BrickRed}{(}e\textcolor{BrickRed}{.}\textbf{\textcolor{Black}{begin}}\textcolor{BrickRed}{(),}\ e\textcolor{BrickRed}{.}\textbf{\textcolor{Black}{end}}\textcolor{BrickRed}{());} \\
\mbox{}\ \ \ \ \textbf{\textcolor{Blue}{for}}\ \textcolor{BrickRed}{(}\textcolor{ForestGreen}{int}\ i\textcolor{BrickRed}{=}\textcolor{Purple}{0}\textcolor{BrickRed}{;}\ i\textcolor{BrickRed}{$<$=}n\textcolor{BrickRed}{;}\ \textcolor{BrickRed}{++}i\textcolor{BrickRed}{)}\textcolor{Red}{\{} \\
\mbox{}\ \ \ \ \ \ \textbf{\textcolor{Black}{make$\_$set}}\textcolor{BrickRed}{(}i\textcolor{BrickRed}{);} \\
\mbox{}\ \ \ \ \textcolor{Red}{\}} \\
\mbox{}\ \ \ \ \textbf{\textcolor{Blue}{for}}\ \textcolor{BrickRed}{(}\textcolor{ForestGreen}{int}\ i\textcolor{BrickRed}{=}\textcolor{Purple}{0}\textcolor{BrickRed}{;}\ i\textcolor{BrickRed}{$<$}e\textcolor{BrickRed}{.}\textbf{\textcolor{Black}{size}}\textcolor{BrickRed}{();}\ \textcolor{BrickRed}{++}i\textcolor{BrickRed}{)}\textcolor{Red}{\{} \\
\mbox{}\ \ \ \ \ \ \textcolor{ForestGreen}{int}\ u\ \textcolor{BrickRed}{=}\ e\textcolor{BrickRed}{[}i\textcolor{BrickRed}{].}start\textcolor{BrickRed}{,}\ v\ \textcolor{BrickRed}{=}\ e\textcolor{BrickRed}{[}i\textcolor{BrickRed}{].}end\textcolor{BrickRed}{,}\ w\ \textcolor{BrickRed}{=}\ e\textcolor{BrickRed}{[}i\textcolor{BrickRed}{].}weight\textcolor{BrickRed}{;} \\
\mbox{}\ \ \ \ \ \ \textbf{\textcolor{Blue}{if}}\ \textcolor{BrickRed}{(}\textbf{\textcolor{Black}{find$\_$set}}\textcolor{BrickRed}{(}u\textcolor{BrickRed}{)}\ \textcolor{BrickRed}{!=}\ \textbf{\textcolor{Black}{find$\_$set}}\textcolor{BrickRed}{(}v\textcolor{BrickRed}{))}\textcolor{Red}{\{} \\
\mbox{}\ \ \ \ \ \ \ \ \textit{\textcolor{Brown}{//printf("{}Joining\ \%d\ with\ \%d\ using\ weight\ \%d\textbackslash{}n"{},\ u,\ v,\ w);}} \\
\mbox{}\ \ \ \ \ \ \ \ total\ \textcolor{BrickRed}{-=}\ w\textcolor{BrickRed}{;} \\
\mbox{}\ \ \ \ \ \ \ \ \textbf{\textcolor{Black}{merge}}\textcolor{BrickRed}{(}u\textcolor{BrickRed}{,}\ v\textcolor{BrickRed}{);} \\
\mbox{}\ \ \ \ \ \ \textcolor{Red}{\}} \\
\mbox{}\ \ \ \ \textcolor{Red}{\}} \\
\mbox{}\ \ \ \ cout\ \textcolor{BrickRed}{$<$$<$}\ total\ \textcolor{BrickRed}{$<$$<$}\ endl\textcolor{BrickRed}{;} \\
\mbox{} \\
\mbox{}\ \ \textcolor{Red}{\}} \\
\mbox{}\ \ \textbf{\textcolor{Blue}{return}}\ \textcolor{Purple}{0}\textcolor{BrickRed}{;} \\
\mbox{}\textcolor{Red}{\}} \\

} \normalfont\normalsize
%.tex

\subsection{Algoritmo de Floyd-Warshall}
\emph{Complejidad:} $ O(n^3) $ \\
Se asume que no hay ciclos de costo negativo.
% Generator: GNU source-highlight, by Lorenzo Bettini, http://www.gnu.org/software/src-highlite

{\ttfamily \raggedright {
\noindent
\mbox{}\textit{\textcolor{Brown}{/*}} \\
\mbox{}\textit{\textcolor{Brown}{\ \ g[i][j]\ =\ Distancia\ entre\ el\ nodo\ i\ y\ el\ j.}} \\
\mbox{}\textit{\textcolor{Brown}{\ */}} \\
\mbox{}\textcolor{ForestGreen}{unsigned}\ \textcolor{ForestGreen}{long}\ \textcolor{ForestGreen}{long}\ g\textcolor{BrickRed}{[}\textcolor{Purple}{101}\textcolor{BrickRed}{][}\textcolor{Purple}{101}\textcolor{BrickRed}{];} \\
\mbox{} \\
\mbox{}\textcolor{ForestGreen}{void}\ \textbf{\textcolor{Black}{floyd}}\textcolor{BrickRed}{()}\textcolor{Red}{\{} \\
\mbox{}\ \ \textit{\textcolor{Brown}{//Llenar\ g}} \\
\mbox{}\ \ \textit{\textcolor{Brown}{//...}} \\
\mbox{} \\
\mbox{}\ \ \textbf{\textcolor{Blue}{for}}\ \textcolor{BrickRed}{(}\textcolor{ForestGreen}{int}\ k\textcolor{BrickRed}{=}\textcolor{Purple}{0}\textcolor{BrickRed}{;}\ k\textcolor{BrickRed}{$<$}n\textcolor{BrickRed}{;}\ \textcolor{BrickRed}{++}k\textcolor{BrickRed}{)}\textcolor{Red}{\{} \\
\mbox{}\ \ \ \ \textbf{\textcolor{Blue}{for}}\ \textcolor{BrickRed}{(}\textcolor{ForestGreen}{int}\ i\textcolor{BrickRed}{=}\textcolor{Purple}{0}\textcolor{BrickRed}{;}\ i\textcolor{BrickRed}{$<$}n\textcolor{BrickRed}{;}\ \textcolor{BrickRed}{++}i\textcolor{BrickRed}{)}\textcolor{Red}{\{} \\
\mbox{}\ \ \ \ \ \ \textbf{\textcolor{Blue}{for}}\ \textcolor{BrickRed}{(}\textcolor{ForestGreen}{int}\ j\textcolor{BrickRed}{=}\textcolor{Purple}{0}\textcolor{BrickRed}{;}\ j\textcolor{BrickRed}{$<$}n\textcolor{BrickRed}{;}\ \textcolor{BrickRed}{++}j\textcolor{BrickRed}{)}\textcolor{Red}{\{} \\
\mbox{}\ \ \ \ \ \ \ \ g\textcolor{BrickRed}{[}i\textcolor{BrickRed}{][}j\textcolor{BrickRed}{]}\ \textcolor{BrickRed}{=}\ \textbf{\textcolor{Black}{min}}\textcolor{BrickRed}{(}g\textcolor{BrickRed}{[}i\textcolor{BrickRed}{][}j\textcolor{BrickRed}{],}\ g\textcolor{BrickRed}{[}i\textcolor{BrickRed}{][}k\textcolor{BrickRed}{]}\ \textcolor{BrickRed}{+}\ g\textcolor{BrickRed}{[}k\textcolor{BrickRed}{][}j\textcolor{BrickRed}{]);} \\
\mbox{}\ \ \ \ \ \ \textcolor{Red}{\}} \\
\mbox{}\ \ \ \ \textcolor{Red}{\}} \\
\mbox{}\ \ \textcolor{Red}{\}} \\
\mbox{}\ \ \textit{\textcolor{Brown}{/*}} \\
\mbox{}\textit{\textcolor{Brown}{\ \ \ \ Acá\ se\ cumple\ que\ g[i][j]\ =\ Longitud\ de\ la\ ruta\ más\ corta\ de\ i\ a\ j.}} \\
\mbox{}\textit{\textcolor{Brown}{\ \ \ */}} \\
\mbox{}\textcolor{Red}{\}} \\

} \normalfont\normalsize
%.tex

\subsection{Algoritmo de Bellman-Ford}
Si no hay ciclos de coste negativo, encuentra la distancia más corta entre un nodo
y todos los demás. Si sí hay, permite saberlo. \\
El coste de las aristas \underline{sí} puede ser negativo.
% Generator: GNU source-highlight, by Lorenzo Bettini, http://www.gnu.org/software/src-highlite

{\ttfamily \raggedright {
\noindent
\mbox{}\textbf{\textcolor{Blue}{struct}}\ edge\textcolor{Red}{\{} \\
\mbox{}\ \ \textcolor{ForestGreen}{int}\ u\textcolor{BrickRed}{,}\ v\textcolor{BrickRed}{,}\ w\textcolor{BrickRed}{;} \\
\mbox{}\textcolor{Red}{\}}\textcolor{BrickRed}{;} \\
\mbox{} \\
\mbox{}edge\ \textcolor{BrickRed}{*}\ e\textcolor{BrickRed}{;}\ \textit{\textcolor{Brown}{//e\ =\ Arreglo\ de\ todas\ las\ aristas}} \\
\mbox{}\textcolor{ForestGreen}{long}\ \textcolor{ForestGreen}{long}\ d\textcolor{BrickRed}{[}\textcolor{Purple}{300}\textcolor{BrickRed}{];}\ \textit{\textcolor{Brown}{//Distancias}} \\
\mbox{}\textcolor{ForestGreen}{int}\ n\textcolor{BrickRed}{;}\ \textit{\textcolor{Brown}{//Cantidad\ de\ nodos}} \\
\mbox{}\textcolor{ForestGreen}{int}\ m\textcolor{BrickRed}{;}\ \textit{\textcolor{Brown}{//Cantidad\ de\ aristas}} \\
\mbox{} \\
\mbox{}\textit{\textcolor{Brown}{/*}} \\
\mbox{}\textit{\textcolor{Brown}{\ \ Retorna\ falso\ si\ hay\ un\ ciclo\ de\ costo\ negativo.}} \\
\mbox{} \\
\mbox{}\textit{\textcolor{Brown}{\ \ Si\ retorna\ verdadero,\ entonces\ d[i]\ contiene\ la\ distancia\ más\ corta\ entre\ el\ s\ y\ el\ nodo\ i.}} \\
\mbox{}\textit{\textcolor{Brown}{\ */}} \\
\mbox{}\textcolor{ForestGreen}{bool}\ \textbf{\textcolor{Black}{bellman}}\textcolor{BrickRed}{(}\textcolor{ForestGreen}{int}\ \textcolor{BrickRed}{\&}s\textcolor{BrickRed}{)}\textcolor{Red}{\{} \\
\mbox{}\ \ \textit{\textcolor{Brown}{//Llenar\ e}} \\
\mbox{}\ \ e\ \textcolor{BrickRed}{=}\ \textbf{\textcolor{Blue}{new}}\ edge\textcolor{BrickRed}{[}n\textcolor{BrickRed}{];} \\
\mbox{}\ \ \textit{\textcolor{Brown}{//...}} \\
\mbox{} \\
\mbox{}\ \ \textbf{\textcolor{Blue}{for}}\ \textcolor{BrickRed}{(}\textcolor{ForestGreen}{int}\ i\textcolor{BrickRed}{=}\textcolor{Purple}{0}\textcolor{BrickRed}{;}\ i\textcolor{BrickRed}{$<$}n\textcolor{BrickRed}{;}\ \textcolor{BrickRed}{++}i\textcolor{BrickRed}{)}\ d\textcolor{BrickRed}{[}i\textcolor{BrickRed}{]}\ \textcolor{BrickRed}{=}\ INT$\_$MAX\textcolor{BrickRed}{;} \\
\mbox{}\ \ d\textcolor{BrickRed}{[}s\textcolor{BrickRed}{]}\ \textcolor{BrickRed}{=}\ 0LL\textcolor{BrickRed}{;} \\
\mbox{} \\
\mbox{}\ \ \textbf{\textcolor{Blue}{for}}\ \textcolor{BrickRed}{(}\textcolor{ForestGreen}{int}\ i\textcolor{BrickRed}{=}\textcolor{Purple}{0}\textcolor{BrickRed}{;}\ i\textcolor{BrickRed}{$<$}n\textcolor{BrickRed}{-}\textcolor{Purple}{1}\textcolor{BrickRed}{;}\ \textcolor{BrickRed}{++}i\textcolor{BrickRed}{)}\textcolor{Red}{\{} \\
\mbox{}\ \ \ \ \textcolor{ForestGreen}{bool}\ cambio\ \textcolor{BrickRed}{=}\ \textbf{\textcolor{Blue}{false}}\textcolor{BrickRed}{;} \\
\mbox{}\ \ \ \ \textbf{\textcolor{Blue}{for}}\ \textcolor{BrickRed}{(}\textcolor{ForestGreen}{int}\ j\textcolor{BrickRed}{=}\textcolor{Purple}{0}\textcolor{BrickRed}{;}\ j\textcolor{BrickRed}{$<$}m\textcolor{BrickRed}{;}\ \textcolor{BrickRed}{++}j\textcolor{BrickRed}{)}\textcolor{Red}{\{} \\
\mbox{}\ \ \ \ \ \ \textcolor{ForestGreen}{int}\ u\ \textcolor{BrickRed}{=}\ e\textcolor{BrickRed}{[}j\textcolor{BrickRed}{].}u\textcolor{BrickRed}{,}\ v\ \textcolor{BrickRed}{=}\ e\textcolor{BrickRed}{[}j\textcolor{BrickRed}{].}v\textcolor{BrickRed}{;} \\
\mbox{}\ \ \ \ \ \ \textcolor{ForestGreen}{long}\ \textcolor{ForestGreen}{long}\ w\ \textcolor{BrickRed}{=}\ e\textcolor{BrickRed}{[}j\textcolor{BrickRed}{].}w\textcolor{BrickRed}{;} \\
\mbox{}\ \ \ \ \ \ \textbf{\textcolor{Blue}{if}}\ \textcolor{BrickRed}{(}d\textcolor{BrickRed}{[}u\textcolor{BrickRed}{]}\ \textcolor{BrickRed}{+}\ w\ \textcolor{BrickRed}{$<$}\ d\textcolor{BrickRed}{[}v\textcolor{BrickRed}{])}\textcolor{Red}{\{} \\
\mbox{}\ \ \ \ \ \ \ \ d\textcolor{BrickRed}{[}v\textcolor{BrickRed}{]}\ \textcolor{BrickRed}{=}\ d\textcolor{BrickRed}{[}u\textcolor{BrickRed}{]}\ \textcolor{BrickRed}{+}\ w\textcolor{BrickRed}{;} \\
\mbox{}\ \ \ \ \ \ \ \ cambio\ \textcolor{BrickRed}{=}\ \textbf{\textcolor{Blue}{true}}\textcolor{BrickRed}{;} \\
\mbox{}\ \ \ \ \ \ \textcolor{Red}{\}} \\
\mbox{}\ \ \ \ \textcolor{Red}{\}} \\
\mbox{}\ \ \ \ \textbf{\textcolor{Blue}{if}}\ \textcolor{BrickRed}{(!}cambio\textcolor{BrickRed}{)}\ \textbf{\textcolor{Blue}{break}}\textcolor{BrickRed}{;}\ \textit{\textcolor{Brown}{//Early-exit}} \\
\mbox{}\ \ \textcolor{Red}{\}} \\
\mbox{} \\
\mbox{}\ \ \textbf{\textcolor{Blue}{for}}\ \textcolor{BrickRed}{(}\textcolor{ForestGreen}{int}\ j\textcolor{BrickRed}{=}\textcolor{Purple}{0}\textcolor{BrickRed}{;}\ j\textcolor{BrickRed}{$<$}m\textcolor{BrickRed}{;}\ \textcolor{BrickRed}{++}j\textcolor{BrickRed}{)}\textcolor{Red}{\{} \\
\mbox{}\ \ \ \ \textcolor{ForestGreen}{int}\ u\ \textcolor{BrickRed}{=}\ e\textcolor{BrickRed}{[}j\textcolor{BrickRed}{].}u\textcolor{BrickRed}{,}\ v\ \textcolor{BrickRed}{=}\ e\textcolor{BrickRed}{[}j\textcolor{BrickRed}{].}v\textcolor{BrickRed}{;} \\
\mbox{}\ \ \ \ \textcolor{ForestGreen}{long}\ \textcolor{ForestGreen}{long}\ w\ \textcolor{BrickRed}{=}\ e\textcolor{BrickRed}{[}j\textcolor{BrickRed}{].}w\textcolor{BrickRed}{;} \\
\mbox{}\ \ \ \ \textbf{\textcolor{Blue}{if}}\ \textcolor{BrickRed}{(}d\textcolor{BrickRed}{[}u\textcolor{BrickRed}{]}\ \textcolor{BrickRed}{+}\ w\ \textcolor{BrickRed}{$<$}\ d\textcolor{BrickRed}{[}v\textcolor{BrickRed}{])}\ \textbf{\textcolor{Blue}{return}}\ \textbf{\textcolor{Blue}{false}}\textcolor{BrickRed}{;} \\
\mbox{}\ \ \textcolor{Red}{\}} \\
\mbox{} \\
\mbox{}\ \ \textbf{\textcolor{Blue}{delete}}\ \textcolor{BrickRed}{[]}\ e\textcolor{BrickRed}{;} \\
\mbox{}\ \ \textbf{\textcolor{Blue}{return}}\ \textbf{\textcolor{Blue}{true}}\textcolor{BrickRed}{;} \\
\mbox{}\textcolor{Red}{\}} \\

} \normalfont\normalsize
%.tex

\subsection{Puntos de articulación}
% Generator: GNU source-highlight, by Lorenzo Bettini, http://www.gnu.org/software/src-highlite

{\ttfamily \raggedright {
\noindent
\mbox{}\textbf{\textcolor{RoyalBlue}{\#include}}\ \texttt{\textcolor{Red}{$<$vector$>$}} \\
\mbox{}\textbf{\textcolor{RoyalBlue}{\#include}}\ \texttt{\textcolor{Red}{$<$set$>$}} \\
\mbox{}\textbf{\textcolor{RoyalBlue}{\#include}}\ \texttt{\textcolor{Red}{$<$map$>$}} \\
\mbox{}\textbf{\textcolor{RoyalBlue}{\#include}}\ \texttt{\textcolor{Red}{$<$algorithm$>$}} \\
\mbox{}\textbf{\textcolor{RoyalBlue}{\#include}}\ \texttt{\textcolor{Red}{$<$iostream$>$}} \\
\mbox{}\textbf{\textcolor{RoyalBlue}{\#include}}\ \texttt{\textcolor{Red}{$<$iterator$>$}} \\
\mbox{} \\
\mbox{}\textbf{\textcolor{Blue}{using}}\ \textbf{\textcolor{Blue}{namespace}}\ std\textcolor{BrickRed}{;} \\
\mbox{} \\
\mbox{}\textbf{\textcolor{Blue}{typedef}}\ string\ node\textcolor{BrickRed}{;} \\
\mbox{}\textbf{\textcolor{Blue}{typedef}}\ map\textcolor{BrickRed}{$<$}node\textcolor{BrickRed}{,}\ vector\textcolor{BrickRed}{$<$}node\textcolor{BrickRed}{$>$}\ \textcolor{BrickRed}{$>$}\ graph\textcolor{BrickRed}{;} \\
\mbox{}\textbf{\textcolor{Blue}{typedef}}\ \textcolor{ForestGreen}{char}\ color\textcolor{BrickRed}{;} \\
\mbox{} \\
\mbox{}\textbf{\textcolor{Blue}{const}}\ color\ WHITE\ \textcolor{BrickRed}{=}\ \textcolor{Purple}{0}\textcolor{BrickRed}{,}\ GRAY\ \textcolor{BrickRed}{=}\ \textcolor{Purple}{1}\textcolor{BrickRed}{,}\ BLACK\ \textcolor{BrickRed}{=}\ \textcolor{Purple}{2}\textcolor{BrickRed}{;} \\
\mbox{} \\
\mbox{}graph\ g\textcolor{BrickRed}{;} \\
\mbox{}map\textcolor{BrickRed}{$<$}node\textcolor{BrickRed}{,}\ color\textcolor{BrickRed}{$>$}\ colors\textcolor{BrickRed}{;} \\
\mbox{}map\textcolor{BrickRed}{$<$}node\textcolor{BrickRed}{,}\ \textcolor{ForestGreen}{int}\textcolor{BrickRed}{$>$}\ d\textcolor{BrickRed}{,}\ low\textcolor{BrickRed}{;} \\
\mbox{} \\
\mbox{}set\textcolor{BrickRed}{$<$}node\textcolor{BrickRed}{$>$}\ cameras\textcolor{BrickRed}{;} \\
\mbox{} \\
\mbox{}\textcolor{ForestGreen}{int}\ timeCount\textcolor{BrickRed}{;} \\
\mbox{} \\
\mbox{}\textcolor{ForestGreen}{void}\ \textbf{\textcolor{Black}{dfs}}\textcolor{BrickRed}{(}node\ v\textcolor{BrickRed}{,}\ \textcolor{ForestGreen}{bool}\ isRoot\ \textcolor{BrickRed}{=}\ \textbf{\textcolor{Blue}{true}}\textcolor{BrickRed}{)}\textcolor{Red}{\{} \\
\mbox{}\ \ colors\textcolor{BrickRed}{[}v\textcolor{BrickRed}{]}\ \textcolor{BrickRed}{=}\ GRAY\textcolor{BrickRed}{;} \\
\mbox{}\ \ d\textcolor{BrickRed}{[}v\textcolor{BrickRed}{]}\ \textcolor{BrickRed}{=}\ low\textcolor{BrickRed}{[}v\textcolor{BrickRed}{]}\ \textcolor{BrickRed}{=}\ \textcolor{BrickRed}{++}timeCount\textcolor{BrickRed}{;} \\
\mbox{}\ \ vector\textcolor{BrickRed}{$<$}node\textcolor{BrickRed}{$>$}\ neighbors\ \textcolor{BrickRed}{=}\ g\textcolor{BrickRed}{[}v\textcolor{BrickRed}{];} \\
\mbox{}\ \ \textcolor{ForestGreen}{int}\ count\ \textcolor{BrickRed}{=}\ \textcolor{Purple}{0}\textcolor{BrickRed}{;} \\
\mbox{}\ \ \textbf{\textcolor{Blue}{for}}\ \textcolor{BrickRed}{(}\textcolor{ForestGreen}{int}\ i\textcolor{BrickRed}{=}\textcolor{Purple}{0}\textcolor{BrickRed}{;}\ i\textcolor{BrickRed}{$<$}neighbors\textcolor{BrickRed}{.}\textbf{\textcolor{Black}{size}}\textcolor{BrickRed}{();}\ \textcolor{BrickRed}{++}i\textcolor{BrickRed}{)}\textcolor{Red}{\{} \\
\mbox{}\ \ \ \ \textbf{\textcolor{Blue}{if}}\ \textcolor{BrickRed}{(}colors\textcolor{BrickRed}{[}neighbors\textcolor{BrickRed}{[}i\textcolor{BrickRed}{]]}\ \textcolor{BrickRed}{==}\ WHITE\textcolor{BrickRed}{)}\textcolor{Red}{\{}\ \textit{\textcolor{Brown}{//\ \ (v,\ neighbors[i])\ is\ a\ tree\ edge}} \\
\mbox{}\ \ \ \ \ \ \textbf{\textcolor{Black}{dfs}}\textcolor{BrickRed}{(}neighbors\textcolor{BrickRed}{[}i\textcolor{BrickRed}{],}\ \textbf{\textcolor{Blue}{false}}\textcolor{BrickRed}{);} \\
\mbox{}\ \ \ \ \ \ \textbf{\textcolor{Blue}{if}}\ \textcolor{BrickRed}{(!}isRoot\ \textcolor{BrickRed}{\&\&}\ low\textcolor{BrickRed}{[}neighbors\textcolor{BrickRed}{[}i\textcolor{BrickRed}{]]}\ \textcolor{BrickRed}{$>$=}\ d\textcolor{BrickRed}{[}v\textcolor{BrickRed}{])}\textcolor{Red}{\{} \\
\mbox{}\ \ \ \ \ \ \ \ cameras\textcolor{BrickRed}{.}\textbf{\textcolor{Black}{insert}}\textcolor{BrickRed}{(}v\textcolor{BrickRed}{);} \\
\mbox{}\ \ \ \ \ \ \textcolor{Red}{\}} \\
\mbox{}\ \ \ \ \ \ low\textcolor{BrickRed}{[}v\textcolor{BrickRed}{]}\ \textcolor{BrickRed}{=}\ \textbf{\textcolor{Black}{min}}\textcolor{BrickRed}{(}low\textcolor{BrickRed}{[}v\textcolor{BrickRed}{],}\ low\textcolor{BrickRed}{[}neighbors\textcolor{BrickRed}{[}i\textcolor{BrickRed}{]]);} \\
\mbox{}\ \ \ \ \ \ \textcolor{BrickRed}{++}count\textcolor{BrickRed}{;} \\
\mbox{}\ \ \ \ \textcolor{Red}{\}}\textbf{\textcolor{Blue}{else}}\textcolor{Red}{\{}\ \textit{\textcolor{Brown}{//\ (v,\ neighbors[i])\ is\ a\ back\ edge}} \\
\mbox{}\ \ \ \ \ \ low\textcolor{BrickRed}{[}v\textcolor{BrickRed}{]}\ \textcolor{BrickRed}{=}\ \textbf{\textcolor{Black}{min}}\textcolor{BrickRed}{(}low\textcolor{BrickRed}{[}v\textcolor{BrickRed}{],}\ d\textcolor{BrickRed}{[}neighbors\textcolor{BrickRed}{[}i\textcolor{BrickRed}{]]);} \\
\mbox{}\ \ \ \ \textcolor{Red}{\}} \\
\mbox{}\ \ \textcolor{Red}{\}} \\
\mbox{}\ \ \textbf{\textcolor{Blue}{if}}\ \textcolor{BrickRed}{(}isRoot\ \textcolor{BrickRed}{\&\&}\ count\ \textcolor{BrickRed}{$>$}\ \textcolor{Purple}{1}\textcolor{BrickRed}{)}\textcolor{Red}{\{}\ \textit{\textcolor{Brown}{//Is\ root\ and\ has\ two\ neighbors\ in\ the\ DFS-tree}} \\
\mbox{}\ \ \ \ cameras\textcolor{BrickRed}{.}\textbf{\textcolor{Black}{insert}}\textcolor{BrickRed}{(}v\textcolor{BrickRed}{);} \\
\mbox{}\ \ \textcolor{Red}{\}} \\
\mbox{}\ \ colors\textcolor{BrickRed}{[}v\textcolor{BrickRed}{]}\ \textcolor{BrickRed}{=}\ BLACK\textcolor{BrickRed}{;} \\
\mbox{}\textcolor{Red}{\}} \\
\mbox{} \\
\mbox{}\textcolor{ForestGreen}{int}\ \textbf{\textcolor{Black}{main}}\textcolor{BrickRed}{()}\textcolor{Red}{\{} \\
\mbox{}\ \ \textcolor{ForestGreen}{int}\ n\textcolor{BrickRed}{;} \\
\mbox{}\ \ \textcolor{ForestGreen}{int}\ map\ \textcolor{BrickRed}{=}\ \textcolor{Purple}{1}\textcolor{BrickRed}{;} \\
\mbox{}\ \ \textbf{\textcolor{Blue}{while}}\ \textcolor{BrickRed}{(}cin\ \textcolor{BrickRed}{$>$$>$}\ n\ \textcolor{BrickRed}{\&\&}\ n\ \textcolor{BrickRed}{$>$}\ \textcolor{Purple}{0}\textcolor{BrickRed}{)}\textcolor{Red}{\{} \\
\mbox{}\ \ \ \ \textbf{\textcolor{Blue}{if}}\ \textcolor{BrickRed}{(}map\ \textcolor{BrickRed}{$>$}\ \textcolor{Purple}{1}\textcolor{BrickRed}{)}\ cout\ \textcolor{BrickRed}{$<$$<$}\ endl\textcolor{BrickRed}{;} \\
\mbox{}\ \ \ \ g\textcolor{BrickRed}{.}\textbf{\textcolor{Black}{clear}}\textcolor{BrickRed}{();} \\
\mbox{}\ \ \ \ colors\textcolor{BrickRed}{.}\textbf{\textcolor{Black}{clear}}\textcolor{BrickRed}{();} \\
\mbox{}\ \ \ \ d\textcolor{BrickRed}{.}\textbf{\textcolor{Black}{clear}}\textcolor{BrickRed}{();} \\
\mbox{}\ \ \ \ low\textcolor{BrickRed}{.}\textbf{\textcolor{Black}{clear}}\textcolor{BrickRed}{();} \\
\mbox{}\ \ \ \ timeCount\ \textcolor{BrickRed}{=}\ \textcolor{Purple}{0}\textcolor{BrickRed}{;} \\
\mbox{}\ \ \ \ \textbf{\textcolor{Blue}{while}}\ \textcolor{BrickRed}{(}n\textcolor{BrickRed}{-\/-)}\textcolor{Red}{\{} \\
\mbox{}\ \ \ \ \ \ node\ v\textcolor{BrickRed}{;} \\
\mbox{}\ \ \ \ \ \ cin\ \textcolor{BrickRed}{$>$$>$}\ v\textcolor{BrickRed}{;} \\
\mbox{}\ \ \ \ \ \ colors\textcolor{BrickRed}{[}v\textcolor{BrickRed}{]}\ \textcolor{BrickRed}{=}\ WHITE\textcolor{BrickRed}{;} \\
\mbox{}\ \ \ \ \ \ g\textcolor{BrickRed}{[}v\textcolor{BrickRed}{]}\ \textcolor{BrickRed}{=}\ vector\textcolor{BrickRed}{$<$}node\textcolor{BrickRed}{$>$();} \\
\mbox{}\ \ \ \ \textcolor{Red}{\}} \\
\mbox{}\ \ \ \  \\
\mbox{}\ \ \ \ cin\ \textcolor{BrickRed}{$>$$>$}\ n\textcolor{BrickRed}{;} \\
\mbox{}\ \ \ \ \textbf{\textcolor{Blue}{while}}\ \textcolor{BrickRed}{(}n\textcolor{BrickRed}{-\/-)}\textcolor{Red}{\{} \\
\mbox{}\ \ \ \ \ \ node\ v\textcolor{BrickRed}{,}u\textcolor{BrickRed}{;} \\
\mbox{}\ \ \ \ \ \ cin\ \textcolor{BrickRed}{$>$$>$}\ v\ \textcolor{BrickRed}{$>$$>$}\ u\textcolor{BrickRed}{;} \\
\mbox{}\ \ \ \ \ \ g\textcolor{BrickRed}{[}v\textcolor{BrickRed}{].}\textbf{\textcolor{Black}{push$\_$back}}\textcolor{BrickRed}{(}u\textcolor{BrickRed}{);} \\
\mbox{}\ \ \ \ \ \ g\textcolor{BrickRed}{[}u\textcolor{BrickRed}{].}\textbf{\textcolor{Black}{push$\_$back}}\textcolor{BrickRed}{(}v\textcolor{BrickRed}{);} \\
\mbox{}\ \ \ \ \textcolor{Red}{\}} \\
\mbox{}\ \ \ \  \\
\mbox{}\ \ \ \ cameras\textcolor{BrickRed}{.}\textbf{\textcolor{Black}{clear}}\textcolor{BrickRed}{();} \\
\mbox{}\ \ \ \  \\
\mbox{}\ \ \ \ \textbf{\textcolor{Blue}{for}}\ \textcolor{BrickRed}{(}graph\textcolor{BrickRed}{::}iterator\ i\ \textcolor{BrickRed}{=}\ g\textcolor{BrickRed}{.}\textbf{\textcolor{Black}{begin}}\textcolor{BrickRed}{();}\ i\ \textcolor{BrickRed}{!=}\ g\textcolor{BrickRed}{.}\textbf{\textcolor{Black}{end}}\textcolor{BrickRed}{();}\ \textcolor{BrickRed}{++}i\textcolor{BrickRed}{)}\textcolor{Red}{\{} \\
\mbox{}\ \ \ \ \ \ \textbf{\textcolor{Blue}{if}}\ \textcolor{BrickRed}{(}colors\textcolor{BrickRed}{[(*}i\textcolor{BrickRed}{).}first\textcolor{BrickRed}{]}\ \textcolor{BrickRed}{==}\ WHITE\textcolor{BrickRed}{)}\textcolor{Red}{\{} \\
\mbox{}\ \ \ \ \ \ \ \ \textbf{\textcolor{Black}{dfs}}\textcolor{BrickRed}{((*}i\textcolor{BrickRed}{).}first\textcolor{BrickRed}{);} \\
\mbox{}\ \ \ \ \ \ \textcolor{Red}{\}} \\
\mbox{}\ \ \ \ \textcolor{Red}{\}} \\
\mbox{}\ \ \ \ \ \  \\
\mbox{}\ \ \ \ cout\ \textcolor{BrickRed}{$<$$<$}\ \texttt{\textcolor{Red}{"{}City\ map\ \#"{}}}\textcolor{BrickRed}{$<$$<$}map\textcolor{BrickRed}{$<$$<$}\texttt{\textcolor{Red}{"{}:\ "{}}}\ \textcolor{BrickRed}{$<$$<$}\ cameras\textcolor{BrickRed}{.}\textbf{\textcolor{Black}{size}}\textcolor{BrickRed}{()}\ \textcolor{BrickRed}{$<$$<$}\ \texttt{\textcolor{Red}{"{}\ camera(s)\ found"{}}}\ \textcolor{BrickRed}{$<$$<$}\ endl\textcolor{BrickRed}{;} \\
\mbox{}\ \ \ \ \textbf{\textcolor{Black}{copy}}\textcolor{BrickRed}{(}cameras\textcolor{BrickRed}{.}\textbf{\textcolor{Black}{begin}}\textcolor{BrickRed}{(),}\ cameras\textcolor{BrickRed}{.}\textbf{\textcolor{Black}{end}}\textcolor{BrickRed}{(),}\ ostream$\_$iterator\textcolor{BrickRed}{$<$}node\textcolor{BrickRed}{$>$(}cout\textcolor{BrickRed}{,}\texttt{\textcolor{Red}{"{}}}\texttt{\textcolor{CarnationPink}{\textbackslash{}n}}\texttt{\textcolor{Red}{"{}}}\textcolor{BrickRed}{));} \\
\mbox{}\ \ \ \ \textcolor{BrickRed}{++}map\textcolor{BrickRed}{;} \\
\mbox{}\ \ \textcolor{Red}{\}} \\
\mbox{}\ \ \textbf{\textcolor{Blue}{return}}\ \textcolor{Purple}{0}\textcolor{BrickRed}{;} \\
\mbox{}\textcolor{Red}{\}} \\

} \normalfont\normalsize
%.tex

\subsection{Máximo flujo: Método de Ford-Fulkerson, algoritmo de Edmonds-Karp}
\medskip
El algoritmo de Edmonds-Karp es una modificación al método de Ford-Fulkerson. Este último
utiliza DFS para hallar un camino de aumentación, pero la sugerencia de Edmonds-Karp
es utilizar BFS que lo hace más eficiente en algunos grafos.
% Generator: GNU source-highlight, by Lorenzo Bettini, http://www.gnu.org/software/src-highlite

{\ttfamily \raggedright {
\noindent
\mbox{}\textcolor{ForestGreen}{int}\ cap\textcolor{BrickRed}{[}MAXN\textcolor{BrickRed}{+}\textcolor{Purple}{1}\textcolor{BrickRed}{][}MAXN\textcolor{BrickRed}{+}\textcolor{Purple}{1}\textcolor{BrickRed}{],}\ flow\textcolor{BrickRed}{[}MAXN\textcolor{BrickRed}{+}\textcolor{Purple}{1}\textcolor{BrickRed}{][}MAXN\textcolor{BrickRed}{+}\textcolor{Purple}{1}\textcolor{BrickRed}{],}\ prev\textcolor{BrickRed}{[}MAXN\textcolor{BrickRed}{+}\textcolor{Purple}{1}\textcolor{BrickRed}{];} \\
\mbox{} \\
\mbox{}\textit{\textcolor{Brown}{/*}} \\
\mbox{}\textit{\textcolor{Brown}{\ \ cap[i][j]\ =\ Capacidad\ de\ la\ arista\ (i,\ j).}} \\
\mbox{}\textit{\textcolor{Brown}{\ \ flow[i][j]\ =\ Flujo\ actual\ de\ i\ a\ j.}} \\
\mbox{}\textit{\textcolor{Brown}{\ \ prev[i]\ =\ Predecesor\ del\ nodo\ i\ en\ un\ camino\ de\ aumentación.}} \\
\mbox{}\textit{\textcolor{Brown}{\ */}} \\
\mbox{} \\
\mbox{}\textcolor{ForestGreen}{int}\ \textbf{\textcolor{Black}{fordFulkerson}}\textcolor{BrickRed}{(}\textcolor{ForestGreen}{int}\ n\textcolor{BrickRed}{,}\ \textcolor{ForestGreen}{int}\ s\textcolor{BrickRed}{,}\ \textcolor{ForestGreen}{int}\ t\textcolor{BrickRed}{)}\textcolor{Red}{\{} \\
\mbox{}\ \ \textcolor{ForestGreen}{int}\ ans\ \textcolor{BrickRed}{=}\ \textcolor{Purple}{0}\textcolor{BrickRed}{;} \\
\mbox{}\ \ \textbf{\textcolor{Blue}{for}}\ \textcolor{BrickRed}{(}\textcolor{ForestGreen}{int}\ i\textcolor{BrickRed}{=}\textcolor{Purple}{0}\textcolor{BrickRed}{;}\ i\textcolor{BrickRed}{$<$}n\textcolor{BrickRed}{;}\ \textcolor{BrickRed}{++}i\textcolor{BrickRed}{)}\ \textbf{\textcolor{Black}{fill}}\textcolor{BrickRed}{(}flow\textcolor{BrickRed}{[}i\textcolor{BrickRed}{],}\ flow\textcolor{BrickRed}{[}i\textcolor{BrickRed}{]+}n\textcolor{BrickRed}{,}\ \textcolor{Purple}{0}\textcolor{BrickRed}{);} \\
\mbox{}\ \ \textbf{\textcolor{Blue}{while}}\ \textcolor{BrickRed}{(}\textbf{\textcolor{Blue}{true}}\textcolor{BrickRed}{)}\textcolor{Red}{\{} \\
\mbox{}\ \ \ \ \textbf{\textcolor{Black}{fill}}\textcolor{BrickRed}{(}prev\textcolor{BrickRed}{,}\ prev\textcolor{BrickRed}{+}n\textcolor{BrickRed}{,}\ \textcolor{BrickRed}{-}\textcolor{Purple}{1}\textcolor{BrickRed}{);} \\
\mbox{}\ \ \ \ queue\textcolor{BrickRed}{$<$}\textcolor{ForestGreen}{int}\textcolor{BrickRed}{$>$}\ q\textcolor{BrickRed}{;} \\
\mbox{}\ \ \ \ q\textcolor{BrickRed}{.}\textbf{\textcolor{Black}{push}}\textcolor{BrickRed}{(}s\textcolor{BrickRed}{);} \\
\mbox{}\ \ \ \ \textbf{\textcolor{Blue}{while}}\ \textcolor{BrickRed}{(}q\textcolor{BrickRed}{.}\textbf{\textcolor{Black}{size}}\textcolor{BrickRed}{()}\ \textcolor{BrickRed}{\&\&}\ prev\textcolor{BrickRed}{[}t\textcolor{BrickRed}{]}\ \textcolor{BrickRed}{==}\ \textcolor{BrickRed}{-}\textcolor{Purple}{1}\textcolor{BrickRed}{)}\textcolor{Red}{\{} \\
\mbox{}\ \ \ \ \ \ \textcolor{ForestGreen}{int}\ u\ \textcolor{BrickRed}{=}\ q\textcolor{BrickRed}{.}\textbf{\textcolor{Black}{front}}\textcolor{BrickRed}{();} \\
\mbox{}\ \ \ \ \ \ q\textcolor{BrickRed}{.}\textbf{\textcolor{Black}{pop}}\textcolor{BrickRed}{();} \\
\mbox{}\ \ \ \ \ \ \textbf{\textcolor{Blue}{for}}\ \textcolor{BrickRed}{(}\textcolor{ForestGreen}{int}\ v\ \textcolor{BrickRed}{=}\ \textcolor{Purple}{0}\textcolor{BrickRed}{;}\ v\textcolor{BrickRed}{$<$}n\textcolor{BrickRed}{;}\ \textcolor{BrickRed}{++}v\textcolor{BrickRed}{)} \\
\mbox{}\ \ \ \ \ \ \ \ \textbf{\textcolor{Blue}{if}}\ \textcolor{BrickRed}{(}\ v\ \textcolor{BrickRed}{!=}\ s\ \textcolor{BrickRed}{\&\&}\ prev\textcolor{BrickRed}{[}v\textcolor{BrickRed}{]}\ \textcolor{BrickRed}{==}\ \textcolor{BrickRed}{-}\textcolor{Purple}{1}\ \textcolor{BrickRed}{\&\&}\ cap\textcolor{BrickRed}{[}u\textcolor{BrickRed}{][}v\textcolor{BrickRed}{]}\ \textcolor{BrickRed}{$>$}\ flow\textcolor{BrickRed}{[}u\textcolor{BrickRed}{][}v\textcolor{BrickRed}{]}\ \textcolor{BrickRed}{)} \\
\mbox{}\ \ \ \ \ \ \ \ \ \ prev\textcolor{BrickRed}{[}v\textcolor{BrickRed}{]}\ \textcolor{BrickRed}{=}\ u\textcolor{BrickRed}{,}\ q\textcolor{BrickRed}{.}\textbf{\textcolor{Black}{push}}\textcolor{BrickRed}{(}v\textcolor{BrickRed}{);} \\
\mbox{}\ \ \ \ \textcolor{Red}{\}} \\
\mbox{} \\
\mbox{}\ \ \ \ \textbf{\textcolor{Blue}{if}}\ \textcolor{BrickRed}{(}prev\textcolor{BrickRed}{[}t\textcolor{BrickRed}{]}\ \textcolor{BrickRed}{==}\ \textcolor{BrickRed}{-}\textcolor{Purple}{1}\textcolor{BrickRed}{)}\ \textbf{\textcolor{Blue}{break}}\textcolor{BrickRed}{;} \\
\mbox{} \\
\mbox{}\ \ \ \ \textcolor{ForestGreen}{int}\ bottleneck\ \textcolor{BrickRed}{=}\ INT$\_$MAX\textcolor{BrickRed}{;} \\
\mbox{}\ \ \ \ \textbf{\textcolor{Blue}{for}}\ \textcolor{BrickRed}{(}\textcolor{ForestGreen}{int}\ v\ \textcolor{BrickRed}{=}\ t\textcolor{BrickRed}{,}\ u\ \textcolor{BrickRed}{=}\ prev\textcolor{BrickRed}{[}v\textcolor{BrickRed}{];}\ u\ \textcolor{BrickRed}{!=}\ \textcolor{BrickRed}{-}\textcolor{Purple}{1}\textcolor{BrickRed}{;}\ v\ \textcolor{BrickRed}{=}\ u\textcolor{BrickRed}{,}\ u\ \textcolor{BrickRed}{=}\ prev\textcolor{BrickRed}{[}v\textcolor{BrickRed}{])}\textcolor{Red}{\{} \\
\mbox{}\ \ \ \ \ \ bottleneck\ \textcolor{BrickRed}{=}\ \textbf{\textcolor{Black}{min}}\textcolor{BrickRed}{(}bottleneck\textcolor{BrickRed}{,}\ cap\textcolor{BrickRed}{[}u\textcolor{BrickRed}{][}v\textcolor{BrickRed}{]}\ \textcolor{BrickRed}{-}\ flow\textcolor{BrickRed}{[}u\textcolor{BrickRed}{][}v\textcolor{BrickRed}{]);} \\
\mbox{}\ \ \ \ \textcolor{Red}{\}} \\
\mbox{}\ \ \ \ \textbf{\textcolor{Blue}{for}}\ \textcolor{BrickRed}{(}\textcolor{ForestGreen}{int}\ v\ \textcolor{BrickRed}{=}\ t\textcolor{BrickRed}{,}\ u\ \textcolor{BrickRed}{=}\ prev\textcolor{BrickRed}{[}v\textcolor{BrickRed}{];}\ u\ \textcolor{BrickRed}{!=}\ \textcolor{BrickRed}{-}\textcolor{Purple}{1}\textcolor{BrickRed}{;}\ v\ \textcolor{BrickRed}{=}\ u\textcolor{BrickRed}{,}\ u\ \textcolor{BrickRed}{=}\ prev\textcolor{BrickRed}{[}v\textcolor{BrickRed}{])}\textcolor{Red}{\{} \\
\mbox{}\ \ \ \ \ \ flow\textcolor{BrickRed}{[}u\textcolor{BrickRed}{][}v\textcolor{BrickRed}{]}\ \textcolor{BrickRed}{+=}\ bottleneck\textcolor{BrickRed}{;} \\
\mbox{}\ \ \ \ \ \ flow\textcolor{BrickRed}{[}v\textcolor{BrickRed}{][}u\textcolor{BrickRed}{]}\ \textcolor{BrickRed}{=}\ \textcolor{BrickRed}{-}flow\textcolor{BrickRed}{[}u\textcolor{BrickRed}{][}v\textcolor{BrickRed}{];} \\
\mbox{}\ \ \ \ \textcolor{Red}{\}} \\
\mbox{}\ \ \ \ ans\ \textcolor{BrickRed}{+=}\ bottleneck\textcolor{BrickRed}{;} \\
\mbox{}\ \ \textcolor{Red}{\}} \\
\mbox{}\ \ \textbf{\textcolor{Blue}{return}}\ ans\textcolor{BrickRed}{;} \\
\mbox{}\textcolor{Red}{\}} \\

} \normalfont\normalsize
%.tex

\section{Programación dinámica}
\subsection{Longest common subsequence}
% Generator: GNU source-highlight, by Lorenzo Bettini, http://www.gnu.org/software/src-highlite

{\ttfamily \raggedright {
\noindent
\mbox{}\textbf{\textcolor{RoyalBlue}{\#define}}\ \textbf{\textcolor{Black}{MAX}}\textcolor{BrickRed}{(}a\textcolor{BrickRed}{,}b\textcolor{BrickRed}{)}\ \textcolor{BrickRed}{((}a\textcolor{BrickRed}{$>$}b\textcolor{BrickRed}{)?(}a\textcolor{BrickRed}{):(}b\textcolor{BrickRed}{))} \\
\mbox{}\textcolor{ForestGreen}{int}\ dp\textcolor{BrickRed}{[}\textcolor{Purple}{1001}\textcolor{BrickRed}{][}\textcolor{Purple}{1001}\textcolor{BrickRed}{];} \\
\mbox{} \\
\mbox{}\textcolor{ForestGreen}{int}\ \textbf{\textcolor{Black}{lcs}}\textcolor{BrickRed}{(}\textbf{\textcolor{Blue}{const}}\ string\ \textcolor{BrickRed}{\&}s\textcolor{BrickRed}{,}\ \textbf{\textcolor{Blue}{const}}\ string\ \textcolor{BrickRed}{\&}t\textcolor{BrickRed}{)}\textcolor{Red}{\{} \\
\mbox{}\ \ \textcolor{ForestGreen}{int}\ m\ \textcolor{BrickRed}{=}\ s\textcolor{BrickRed}{.}\textbf{\textcolor{Black}{size}}\textcolor{BrickRed}{(),}\ n\ \textcolor{BrickRed}{=}\ t\textcolor{BrickRed}{.}\textbf{\textcolor{Black}{size}}\textcolor{BrickRed}{();} \\
\mbox{}\ \ \textbf{\textcolor{Blue}{if}}\ \textcolor{BrickRed}{(}m\ \textcolor{BrickRed}{==}\ \textcolor{Purple}{0}\ \textcolor{BrickRed}{$|$$|$}\ n\ \textcolor{BrickRed}{==}\ \textcolor{Purple}{0}\textcolor{BrickRed}{)}\ \textbf{\textcolor{Blue}{return}}\ \textcolor{Purple}{0}\textcolor{BrickRed}{;} \\
\mbox{}\ \ \textbf{\textcolor{Blue}{for}}\ \textcolor{BrickRed}{(}\textcolor{ForestGreen}{int}\ i\textcolor{BrickRed}{=}\textcolor{Purple}{0}\textcolor{BrickRed}{;}\ i\textcolor{BrickRed}{$<$=}m\textcolor{BrickRed}{;}\ \textcolor{BrickRed}{++}i\textcolor{BrickRed}{)} \\
\mbox{}\ \ \ \ dp\textcolor{BrickRed}{[}i\textcolor{BrickRed}{][}\textcolor{Purple}{0}\textcolor{BrickRed}{]}\ \textcolor{BrickRed}{=}\ \textcolor{Purple}{0}\textcolor{BrickRed}{;} \\
\mbox{}\ \ \textbf{\textcolor{Blue}{for}}\ \textcolor{BrickRed}{(}\textcolor{ForestGreen}{int}\ j\textcolor{BrickRed}{=}\textcolor{Purple}{1}\textcolor{BrickRed}{;}\ j\textcolor{BrickRed}{$<$=}n\textcolor{BrickRed}{;}\ \textcolor{BrickRed}{++}j\textcolor{BrickRed}{)} \\
\mbox{}\ \ \ \ dp\textcolor{BrickRed}{[}\textcolor{Purple}{0}\textcolor{BrickRed}{][}j\textcolor{BrickRed}{]}\ \textcolor{BrickRed}{=}\ \textcolor{Purple}{0}\textcolor{BrickRed}{;} \\
\mbox{}\ \ \textbf{\textcolor{Blue}{for}}\ \textcolor{BrickRed}{(}\textcolor{ForestGreen}{int}\ i\textcolor{BrickRed}{=}\textcolor{Purple}{0}\textcolor{BrickRed}{;}\ i\textcolor{BrickRed}{$<$}m\textcolor{BrickRed}{;}\ \textcolor{BrickRed}{++}i\textcolor{BrickRed}{)} \\
\mbox{}\ \ \ \ \textbf{\textcolor{Blue}{for}}\ \textcolor{BrickRed}{(}\textcolor{ForestGreen}{int}\ j\textcolor{BrickRed}{=}\textcolor{Purple}{0}\textcolor{BrickRed}{;}\ j\textcolor{BrickRed}{$<$}n\textcolor{BrickRed}{;}\ \textcolor{BrickRed}{++}j\textcolor{BrickRed}{)} \\
\mbox{}\ \ \ \ \ \ \textbf{\textcolor{Blue}{if}}\ \textcolor{BrickRed}{(}s\textcolor{BrickRed}{[}i\textcolor{BrickRed}{]}\ \textcolor{BrickRed}{==}\ t\textcolor{BrickRed}{[}j\textcolor{BrickRed}{])} \\
\mbox{}\ \ \ \ \ \ \ \ dp\textcolor{BrickRed}{[}i\textcolor{BrickRed}{+}\textcolor{Purple}{1}\textcolor{BrickRed}{][}j\textcolor{BrickRed}{+}\textcolor{Purple}{1}\textcolor{BrickRed}{]}\ \textcolor{BrickRed}{=}\ dp\textcolor{BrickRed}{[}i\textcolor{BrickRed}{][}j\textcolor{BrickRed}{]+}\textcolor{Purple}{1}\textcolor{BrickRed}{;} \\
\mbox{}\ \ \ \ \ \ \textbf{\textcolor{Blue}{else}} \\
\mbox{}\ \ \ \ \ \ \ \ dp\textcolor{BrickRed}{[}i\textcolor{BrickRed}{+}\textcolor{Purple}{1}\textcolor{BrickRed}{][}j\textcolor{BrickRed}{+}\textcolor{Purple}{1}\textcolor{BrickRed}{]}\ \textcolor{BrickRed}{=}\ \textbf{\textcolor{Black}{MAX}}\textcolor{BrickRed}{(}dp\textcolor{BrickRed}{[}i\textcolor{BrickRed}{+}\textcolor{Purple}{1}\textcolor{BrickRed}{][}j\textcolor{BrickRed}{],}\ dp\textcolor{BrickRed}{[}i\textcolor{BrickRed}{][}j\textcolor{BrickRed}{+}\textcolor{Purple}{1}\textcolor{BrickRed}{]);} \\
\mbox{}\ \ \textbf{\textcolor{Blue}{return}}\ dp\textcolor{BrickRed}{[}m\textcolor{BrickRed}{][}n\textcolor{BrickRed}{];} \\
\mbox{}\textcolor{Red}{\}} \\

} \normalfont\normalsize
%.tex

\subsection{M\'axima Submatriz de ceros}
% Generator: GNU source-highlight, by Lorenzo Bettini, http://www.gnu.org/software/src-highlite
{\ttfamily \raggedright {
\noindent
\mbox{}\textcolor{ForestGreen}{int}\ n\textcolor{BrickRed}{,}\ m\textcolor{BrickRed}{;} \\
\mbox{}cin\ \textcolor{BrickRed}{$>$$>$}\ n\ \textcolor{BrickRed}{$>$$>$}\ m\textcolor{BrickRed}{;} \\
\mbox{}vector\ \textcolor{BrickRed}{$<$}\ \textcolor{TealBlue}{vector$<$char$>$\ $>$}\ \textbf{\textcolor{Black}{a}}\ \textcolor{BrickRed}{(}n\textcolor{BrickRed}{,}\ vector\textcolor{BrickRed}{$<$}\textcolor{ForestGreen}{char}\textcolor{BrickRed}{$>$}\ \textcolor{BrickRed}{(}m\textcolor{BrickRed}{));} \\
\mbox{}\textbf{\textcolor{Blue}{for}}\ \textcolor{BrickRed}{(}\textcolor{ForestGreen}{int}\ i\textcolor{BrickRed}{=}\textcolor{Purple}{0}\textcolor{BrickRed}{;}\ i\textcolor{BrickRed}{$<$}n\textcolor{BrickRed}{;}\ \textcolor{BrickRed}{++}i\textcolor{BrickRed}{)} \\
\mbox{}\ \ \textbf{\textcolor{Blue}{for}}\ \textcolor{BrickRed}{(}\textcolor{ForestGreen}{int}\ j\textcolor{BrickRed}{=}\textcolor{Purple}{0}\textcolor{BrickRed}{;}\ j\textcolor{BrickRed}{$<$}m\textcolor{BrickRed}{;}\ \textcolor{BrickRed}{++}j\textcolor{BrickRed}{)} \\
\mbox{}\ \ \ \ cin\ \textcolor{BrickRed}{$>$$>$}\ a\textcolor{BrickRed}{[}i\textcolor{BrickRed}{][}j\textcolor{BrickRed}{];} \\
\mbox{} \\
\mbox{}\textcolor{ForestGreen}{int}\ ans\ \textcolor{BrickRed}{=}\ \textcolor{Purple}{0}\textcolor{BrickRed}{;} \\
\mbox{}\textcolor{TealBlue}{vector$<$int$>$}\ \textbf{\textcolor{Black}{d}}\ \textcolor{BrickRed}{(}m\textcolor{BrickRed}{,}\ \textcolor{BrickRed}{-}\textcolor{Purple}{1}\textcolor{BrickRed}{);} \\
\mbox{}\textcolor{TealBlue}{vector$<$int$>$}\ \textbf{\textcolor{Black}{dl}}\ \textcolor{BrickRed}{(}m\textcolor{BrickRed}{),}\ \textbf{\textcolor{Black}{dr}}\ \textcolor{BrickRed}{(}m\textcolor{BrickRed}{);} \\
\mbox{}\textcolor{TealBlue}{stack$<$int$>$}\ st\textcolor{BrickRed}{;} \\
\mbox{}\textbf{\textcolor{Blue}{for}}\ \textcolor{BrickRed}{(}\textcolor{ForestGreen}{int}\ i\textcolor{BrickRed}{=}\textcolor{Purple}{0}\textcolor{BrickRed}{;}\ i\textcolor{BrickRed}{$<$}n\textcolor{BrickRed}{;}\ \textcolor{BrickRed}{++}i\textcolor{BrickRed}{)}\ \textcolor{Red}{\{} \\
\mbox{}\ \ \textbf{\textcolor{Blue}{for}}\ \textcolor{BrickRed}{(}\textcolor{ForestGreen}{int}\ j\textcolor{BrickRed}{=}\textcolor{Purple}{0}\textcolor{BrickRed}{;}\ j\textcolor{BrickRed}{$<$}m\textcolor{BrickRed}{;}\ \textcolor{BrickRed}{++}j\textcolor{BrickRed}{)} \\
\mbox{}\ \ \ \ \textbf{\textcolor{Blue}{if}}\ \textcolor{BrickRed}{(}a\textcolor{BrickRed}{[}i\textcolor{BrickRed}{][}j\textcolor{BrickRed}{]}\ \textcolor{BrickRed}{==}\ \textcolor{Purple}{1}\textcolor{BrickRed}{)} \\
\mbox{}\ \ \ \ \ \ d\textcolor{BrickRed}{[}j\textcolor{BrickRed}{]}\ \textcolor{BrickRed}{=}\ i\textcolor{BrickRed}{;} \\
\mbox{}\ \ \textbf{\textcolor{Blue}{while}}\ \textcolor{BrickRed}{(!}st\textcolor{BrickRed}{.}\textbf{\textcolor{Black}{empty}}\textcolor{BrickRed}{())}\ st\textcolor{BrickRed}{.}\textbf{\textcolor{Black}{pop}}\textcolor{BrickRed}{();} \\
\mbox{}\ \ \textbf{\textcolor{Blue}{for}}\ \textcolor{BrickRed}{(}\textcolor{ForestGreen}{int}\ j\textcolor{BrickRed}{=}\textcolor{Purple}{0}\textcolor{BrickRed}{;}\ j\textcolor{BrickRed}{$<$}m\textcolor{BrickRed}{;}\ \textcolor{BrickRed}{++}j\textcolor{BrickRed}{)}\ \textcolor{Red}{\{} \\
\mbox{}\ \ \ \ \textbf{\textcolor{Blue}{while}}\ \textcolor{BrickRed}{(!}st\textcolor{BrickRed}{.}\textbf{\textcolor{Black}{empty}}\textcolor{BrickRed}{()}\ \textcolor{BrickRed}{\&\&}\ d\textcolor{BrickRed}{[}st\textcolor{BrickRed}{.}\textbf{\textcolor{Black}{top}}\textcolor{BrickRed}{()]}\ \textcolor{BrickRed}{$<$=}\ d\textcolor{BrickRed}{[}j\textcolor{BrickRed}{])}\ \ st\textcolor{BrickRed}{.}\textbf{\textcolor{Black}{pop}}\textcolor{BrickRed}{();} \\
\mbox{}\ \ \ \ dl\textcolor{BrickRed}{[}j\textcolor{BrickRed}{]}\ \textcolor{BrickRed}{=}\ st\textcolor{BrickRed}{.}\textbf{\textcolor{Black}{empty}}\textcolor{BrickRed}{()}\ \textcolor{BrickRed}{?}\ \textcolor{BrickRed}{-}\textcolor{Purple}{1}\ \textcolor{BrickRed}{:}\ st\textcolor{BrickRed}{.}\textbf{\textcolor{Black}{top}}\textcolor{BrickRed}{();} \\
\mbox{}\ \ \ \ st\textcolor{BrickRed}{.}\textbf{\textcolor{Black}{push}}\ \textcolor{BrickRed}{(}j\textcolor{BrickRed}{);} \\
\mbox{}\ \ \textcolor{Red}{\}} \\
\mbox{}\ \ \textbf{\textcolor{Blue}{while}}\ \textcolor{BrickRed}{(!}st\textcolor{BrickRed}{.}\textbf{\textcolor{Black}{empty}}\textcolor{BrickRed}{())}\ st\textcolor{BrickRed}{.}\textbf{\textcolor{Black}{pop}}\textcolor{BrickRed}{();} \\
\mbox{}\ \ \textbf{\textcolor{Blue}{for}}\ \textcolor{BrickRed}{(}\textcolor{ForestGreen}{int}\ j\textcolor{BrickRed}{=}m\textcolor{BrickRed}{-}\textcolor{Purple}{1}\textcolor{BrickRed}{;}\ j\textcolor{BrickRed}{$>$=}\textcolor{Purple}{0}\textcolor{BrickRed}{;}\ \textcolor{BrickRed}{-\/-}j\textcolor{BrickRed}{)}\ \textcolor{Red}{\{} \\
\mbox{}\ \ \ \ \textbf{\textcolor{Blue}{while}}\ \textcolor{BrickRed}{(!}st\textcolor{BrickRed}{.}\textbf{\textcolor{Black}{empty}}\textcolor{BrickRed}{()}\ \textcolor{BrickRed}{\&\&}\ d\textcolor{BrickRed}{[}st\textcolor{BrickRed}{.}\textbf{\textcolor{Black}{top}}\textcolor{BrickRed}{()]}\ \textcolor{BrickRed}{$<$=}\ d\textcolor{BrickRed}{[}j\textcolor{BrickRed}{])}\ \ st\textcolor{BrickRed}{.}\textbf{\textcolor{Black}{pop}}\textcolor{BrickRed}{();} \\
\mbox{}\ \ \ \ dr\textcolor{BrickRed}{[}j\textcolor{BrickRed}{]}\ \textcolor{BrickRed}{=}\ st\textcolor{BrickRed}{.}\textbf{\textcolor{Black}{empty}}\textcolor{BrickRed}{()}\ \textcolor{BrickRed}{?}\ m\ \textcolor{BrickRed}{:}\ st\textcolor{BrickRed}{.}\textbf{\textcolor{Black}{top}}\textcolor{BrickRed}{();} \\
\mbox{}\ \ \ \ st\textcolor{BrickRed}{.}\textbf{\textcolor{Black}{push}}\ \textcolor{BrickRed}{(}j\textcolor{BrickRed}{);} \\
\mbox{}\ \ \textcolor{Red}{\}} \\
\mbox{}\ \ \textbf{\textcolor{Blue}{for}}\ \textcolor{BrickRed}{(}\textcolor{ForestGreen}{int}\ j\textcolor{BrickRed}{=}\textcolor{Purple}{0}\textcolor{BrickRed}{;}\ j\textcolor{BrickRed}{$<$}m\textcolor{BrickRed}{;}\ \textcolor{BrickRed}{++}j\textcolor{BrickRed}{)} \\
\mbox{}\ \ \ \ ans\ \textcolor{BrickRed}{=}\ \textbf{\textcolor{Black}{max}}\ \textcolor{BrickRed}{(}ans\textcolor{BrickRed}{,}\ \textcolor{BrickRed}{(}i\ \textcolor{BrickRed}{-}\ d\textcolor{BrickRed}{[}j\textcolor{BrickRed}{])}\ \textcolor{BrickRed}{*}\ \textcolor{BrickRed}{(}dr\textcolor{BrickRed}{[}j\textcolor{BrickRed}{]}\ \textcolor{BrickRed}{-}\ dl\textcolor{BrickRed}{[}j\textcolor{BrickRed}{]}\ \textcolor{BrickRed}{-}\ \textcolor{Purple}{1}\textcolor{BrickRed}{));} \\
\mbox{}\textcolor{Red}{\}} \\
\mbox{} \\
\mbox{}cout\ \textcolor{BrickRed}{$<$$<$}\ ans\textcolor{BrickRed}{;}
}%.tex

\section{Geometría}
\subsection{Área de un polígono}
Si P es un polígono simple (no se intersecta a sí mismo) su área está dada por: \\

$ A(P) = \frac{1}{2} \displaystyle\sum_{i=0}^{n-1} (x_{i} \cdot y_{i+1} - x_{i+1} \cdot y_{i}) $ \\
\bigskip
% Generator: GNU source-highlight, by Lorenzo Bettini, http://www.gnu.org/software/src-highlite

{\ttfamily \raggedright {
\noindent
\mbox{}\textit{\textcolor{Brown}{//P\ es\ un\ polígono\ ordenado\ anticlockwise.}} \\
\mbox{}\textit{\textcolor{Brown}{//Si\ es\ clockwise,\ retorna\ el\ area\ negativa.}} \\
\mbox{}\textit{\textcolor{Brown}{//Si\ no\ esta\ ordenado\ retorna\ pura\ mierda.}} \\
\mbox{}\textit{\textcolor{Brown}{//P[0]\ !=\ P[n-1]}} \\
\mbox{}\textcolor{ForestGreen}{double}\ \textbf{\textcolor{Black}{PolygonArea}}\textcolor{BrickRed}{(}\textbf{\textcolor{Blue}{const}}\ vector\textcolor{BrickRed}{$<$}point\textcolor{BrickRed}{$>$}\ \textcolor{BrickRed}{\&}p\textcolor{BrickRed}{)}\textcolor{Red}{\{} \\
\mbox{}\ \ \textcolor{ForestGreen}{double}\ r\ \textcolor{BrickRed}{=}\ \textcolor{Purple}{0.0}\textcolor{BrickRed}{;} \\
\mbox{}\ \ \textbf{\textcolor{Blue}{for}}\ \textcolor{BrickRed}{(}\textcolor{ForestGreen}{int}\ i\textcolor{BrickRed}{=}\textcolor{Purple}{0}\textcolor{BrickRed}{;}\ i\textcolor{BrickRed}{$<$}p\textcolor{BrickRed}{.}\textbf{\textcolor{Black}{size}}\textcolor{BrickRed}{();}\ \textcolor{BrickRed}{++}i\textcolor{BrickRed}{)}\textcolor{Red}{\{} \\
\mbox{}\ \ \ \ \textcolor{ForestGreen}{int}\ j\ \textcolor{BrickRed}{=}\ \textcolor{BrickRed}{(}i\textcolor{BrickRed}{+}\textcolor{Purple}{1}\textcolor{BrickRed}{)}\ \textcolor{BrickRed}{\%}\ p\textcolor{BrickRed}{.}\textbf{\textcolor{Black}{size}}\textcolor{BrickRed}{();} \\
\mbox{}\ \ \ \ r\ \textcolor{BrickRed}{+=}\ p\textcolor{BrickRed}{[}i\textcolor{BrickRed}{].}x\textcolor{BrickRed}{*}p\textcolor{BrickRed}{[}j\textcolor{BrickRed}{].}y\ \textcolor{BrickRed}{-}\ p\textcolor{BrickRed}{[}j\textcolor{BrickRed}{].}x\textcolor{BrickRed}{*}p\textcolor{BrickRed}{[}i\textcolor{BrickRed}{].}y\textcolor{BrickRed}{;} \\
\mbox{}\ \ \textcolor{Red}{\}} \\
\mbox{}\ \ \textbf{\textcolor{Blue}{return}}\ r\textcolor{BrickRed}{/}\textcolor{Purple}{2.0}\textcolor{BrickRed}{;} \\
\mbox{}\textcolor{Red}{\}} \\

} \normalfont\normalsize
%.tex

\subsection{Centro de masa de un polígono}
Si P es un polígono simple (no se intersecta a sí mismo) su centro de masa está dado por: \\

$ \displaystyle\bar{C}_{x} = \frac{ \displaystyle\iint_{R} x \, dA }{M} = \frac{1}{6M}\sum_{i=1}^{n} (y_{i+1} - y_{i}) (x_{i+1}^2 + x_{i+1} \cdot x_{i} + x_{i}^2) $

\medskip

$\displaystyle\bar{C}_{y} = \frac{ \displaystyle\iint_{R} y \, dA }{M} = \frac{1}{6M} \sum_{i=1}^{n} (x_{i} - x_{i+1}) (y_{i+1}^2 + y_{i+1} \cdot y_{i} + y_{i}^2)$

\medskip

Donde $ M $ es el área del polígono. \\

Otra posible fórmula equivalente:

$ \displaystyle\bar{C}_{x} = \frac{1}{6A} \sum_{i=0}^{n-1} (x_{i} + x_{i+1}) (x_{i} \cdot y_{i+1} - x_{i+1} \cdot y_{i}) $

\medskip

$ \displaystyle\bar{C}_{y} = \frac{1}{6A} \sum_{i=0}^{n-1} (y_{i} + y_{i+1}) (x_{i} \cdot y_{i+1} - x_{i+1} \cdot y_{i}) $


\subsection{Convex hull: Graham Scan}
\emph{Complejidad:} $ O(n \log_{2}{n}) $
% Generator: GNU source-highlight, by Lorenzo Bettini, http://www.gnu.org/software/src-highlite

{\ttfamily \raggedright {
\noindent
\mbox{}\textit{\textcolor{Brown}{/*}} \\
\mbox{}\textit{\textcolor{Brown}{\ \ Graham\ Scan.}} \\
\mbox{}\textit{\textcolor{Brown}{\ */}} \\
\mbox{}\textbf{\textcolor{RoyalBlue}{\#include}}\ \texttt{\textcolor{Red}{$<$iostream$>$}} \\
\mbox{}\textbf{\textcolor{RoyalBlue}{\#include}}\ \texttt{\textcolor{Red}{$<$vector$>$}} \\
\mbox{}\textbf{\textcolor{RoyalBlue}{\#include}}\ \texttt{\textcolor{Red}{$<$algorithm$>$}} \\
\mbox{}\textbf{\textcolor{RoyalBlue}{\#include}}\ \texttt{\textcolor{Red}{$<$iterator$>$}} \\
\mbox{}\textbf{\textcolor{RoyalBlue}{\#include}}\ \texttt{\textcolor{Red}{$<$math.h$>$}} \\
\mbox{}\textbf{\textcolor{RoyalBlue}{\#include}}\ \texttt{\textcolor{Red}{$<$stdio.h$>$}} \\
\mbox{} \\
\mbox{}\textbf{\textcolor{Blue}{using}}\ \textbf{\textcolor{Blue}{namespace}}\ std\textcolor{BrickRed}{;} \\
\mbox{} \\
\mbox{}\textbf{\textcolor{Blue}{const}}\ \textcolor{ForestGreen}{double}\ pi\ \textcolor{BrickRed}{=}\ \textcolor{Purple}{2}\textcolor{BrickRed}{*}\textbf{\textcolor{Black}{acos}}\textcolor{BrickRed}{(}\textcolor{Purple}{0}\textcolor{BrickRed}{);} \\
\mbox{} \\
\mbox{}\textbf{\textcolor{Blue}{struct}}\ point\textcolor{Red}{\{} \\
\mbox{}\ \ \textcolor{ForestGreen}{int}\ x\textcolor{BrickRed}{,}y\textcolor{BrickRed}{;} \\
\mbox{}\ \ \textbf{\textcolor{Black}{point}}\textcolor{BrickRed}{()}\ \textcolor{Red}{\{\}} \\
\mbox{}\ \ \textbf{\textcolor{Black}{point}}\textcolor{BrickRed}{(}\textcolor{ForestGreen}{int}\ X\textcolor{BrickRed}{,}\ \textcolor{ForestGreen}{int}\ Y\textcolor{BrickRed}{)}\ \textcolor{BrickRed}{:}\ \textbf{\textcolor{Black}{x}}\textcolor{BrickRed}{(}X\textcolor{BrickRed}{),}\ \textbf{\textcolor{Black}{y}}\textcolor{BrickRed}{(}Y\textcolor{BrickRed}{)}\ \textcolor{Red}{\{\}} \\
\mbox{}\textcolor{Red}{\}}\textcolor{BrickRed}{;} \\
\mbox{} \\
\mbox{}point\ pivot\textcolor{BrickRed}{;} \\
\mbox{} \\
\mbox{}ostream\textcolor{BrickRed}{\&}\ \textbf{\textcolor{Blue}{operator}}\textcolor{BrickRed}{$<$$<$}\ \textcolor{BrickRed}{(}ostream\textcolor{BrickRed}{\&}\ out\textcolor{BrickRed}{,}\ \textbf{\textcolor{Blue}{const}}\ point\textcolor{BrickRed}{\&}\ c\textcolor{BrickRed}{)} \\
\mbox{}\textcolor{Red}{\{} \\
\mbox{}\ \ out\ \textcolor{BrickRed}{$<$$<$}\ \texttt{\textcolor{Red}{"{}("{}}}\ \textcolor{BrickRed}{$<$$<$}\ c\textcolor{BrickRed}{.}x\ \textcolor{BrickRed}{$<$$<$}\ \texttt{\textcolor{Red}{"{},"{}}}\ \textcolor{BrickRed}{$<$$<$}\ c\textcolor{BrickRed}{.}y\ \textcolor{BrickRed}{$<$$<$}\ \texttt{\textcolor{Red}{"{})"{}}}\textcolor{BrickRed}{;} \\
\mbox{}\ \ \textbf{\textcolor{Blue}{return}}\ out\textcolor{BrickRed}{;} \\
\mbox{}\textcolor{Red}{\}} \\
\mbox{} \\
\mbox{}\textbf{\textcolor{Blue}{inline}}\ \textcolor{ForestGreen}{int}\ \textbf{\textcolor{Black}{distsqr}}\textcolor{BrickRed}{(}\textbf{\textcolor{Blue}{const}}\ point\ \textcolor{BrickRed}{\&}a\textcolor{BrickRed}{,}\ \textbf{\textcolor{Blue}{const}}\ point\ \textcolor{BrickRed}{\&}b\textcolor{BrickRed}{)}\textcolor{Red}{\{} \\
\mbox{}\ \ \textbf{\textcolor{Blue}{return}}\ \textcolor{BrickRed}{(}a\textcolor{BrickRed}{.}x\ \textcolor{BrickRed}{-}\ b\textcolor{BrickRed}{.}x\textcolor{BrickRed}{)*(}a\textcolor{BrickRed}{.}x\ \textcolor{BrickRed}{-}\ b\textcolor{BrickRed}{.}x\textcolor{BrickRed}{)}\ \textcolor{BrickRed}{+}\ \textcolor{BrickRed}{(}a\textcolor{BrickRed}{.}y\ \textcolor{BrickRed}{-}\ b\textcolor{BrickRed}{.}y\textcolor{BrickRed}{)*(}a\textcolor{BrickRed}{.}y\ \textcolor{BrickRed}{-}\ b\textcolor{BrickRed}{.}y\textcolor{BrickRed}{);} \\
\mbox{}\textcolor{Red}{\}} \\
\mbox{} \\
\mbox{}\textbf{\textcolor{Blue}{inline}}\ \textcolor{ForestGreen}{double}\ \textbf{\textcolor{Black}{dist}}\textcolor{BrickRed}{(}\textbf{\textcolor{Blue}{const}}\ point\ \textcolor{BrickRed}{\&}a\textcolor{BrickRed}{,}\ \textbf{\textcolor{Blue}{const}}\ point\ \textcolor{BrickRed}{\&}b\textcolor{BrickRed}{)}\textcolor{Red}{\{} \\
\mbox{}\ \ \textbf{\textcolor{Blue}{return}}\ \textbf{\textcolor{Black}{sqrt}}\textcolor{BrickRed}{(}\textbf{\textcolor{Black}{distsqr}}\textcolor{BrickRed}{(}a\textcolor{BrickRed}{,}\ b\textcolor{BrickRed}{));} \\
\mbox{}\textcolor{Red}{\}} \\
\mbox{} \\
\mbox{}\textit{\textcolor{Brown}{//retorna\ $>$\ 0\ si\ c\ esta\ a\ la\ izquierda\ del\ segmento\ AB}} \\
\mbox{}\textit{\textcolor{Brown}{//retorna\ $<$\ 0\ si\ c\ esta\ a\ la\ derecha\ del\ segmento\ AB}} \\
\mbox{}\textit{\textcolor{Brown}{//retorna\ ==\ 0\ si\ c\ es\ colineal\ con\ el\ segmento\ AB}} \\
\mbox{}\textbf{\textcolor{Blue}{inline}}\ \textcolor{ForestGreen}{int}\ \textbf{\textcolor{Black}{cross}}\textcolor{BrickRed}{(}\textbf{\textcolor{Blue}{const}}\ point\ \textcolor{BrickRed}{\&}a\textcolor{BrickRed}{,}\ \textbf{\textcolor{Blue}{const}}\ point\ \textcolor{BrickRed}{\&}b\textcolor{BrickRed}{,}\ \textbf{\textcolor{Blue}{const}}\ point\ \textcolor{BrickRed}{\&}c\textcolor{BrickRed}{)}\textcolor{Red}{\{} \\
\mbox{}\ \ \textbf{\textcolor{Blue}{return}}\ \textcolor{BrickRed}{(}b\textcolor{BrickRed}{.}x\textcolor{BrickRed}{-}a\textcolor{BrickRed}{.}x\textcolor{BrickRed}{)*(}c\textcolor{BrickRed}{.}y\textcolor{BrickRed}{-}a\textcolor{BrickRed}{.}y\textcolor{BrickRed}{)}\ \textcolor{BrickRed}{-}\ \textcolor{BrickRed}{(}c\textcolor{BrickRed}{.}x\textcolor{BrickRed}{-}a\textcolor{BrickRed}{.}x\textcolor{BrickRed}{)*(}b\textcolor{BrickRed}{.}y\textcolor{BrickRed}{-}a\textcolor{BrickRed}{.}y\textcolor{BrickRed}{);} \\
\mbox{}\textcolor{Red}{\}} \\
\mbox{} \\
\mbox{}\textit{\textcolor{Brown}{//Self\ $<$\ that\ si\ esta\ a\ la\ derecha\ del\ segmento\ Pivot-That}} \\
\mbox{}\textcolor{ForestGreen}{bool}\ \textbf{\textcolor{Black}{angleCmp}}\textcolor{BrickRed}{(}\textbf{\textcolor{Blue}{const}}\ point\ \textcolor{BrickRed}{\&}self\textcolor{BrickRed}{,}\ \textbf{\textcolor{Blue}{const}}\ point\ \textcolor{BrickRed}{\&}that\textcolor{BrickRed}{)}\textcolor{Red}{\{} \\
\mbox{}\ \ \textcolor{ForestGreen}{int}\ t\ \textcolor{BrickRed}{=}\ \textbf{\textcolor{Black}{cross}}\textcolor{BrickRed}{(}pivot\textcolor{BrickRed}{,}\ that\textcolor{BrickRed}{,}\ self\textcolor{BrickRed}{);} \\
\mbox{}\ \ \textbf{\textcolor{Blue}{if}}\ \textcolor{BrickRed}{(}t\ \textcolor{BrickRed}{$<$}\ \textcolor{Purple}{0}\textcolor{BrickRed}{)}\ \textbf{\textcolor{Blue}{return}}\ \textbf{\textcolor{Blue}{true}}\textcolor{BrickRed}{;} \\
\mbox{}\ \ \textbf{\textcolor{Blue}{if}}\ \textcolor{BrickRed}{(}t\ \textcolor{BrickRed}{==}\ \textcolor{Purple}{0}\textcolor{BrickRed}{)}\textcolor{Red}{\{} \\
\mbox{}\ \ \ \ \textit{\textcolor{Brown}{//Self\ $<$\ that\ si\ está\ más\ cerquita}} \\
\mbox{}\ \ \ \ \textbf{\textcolor{Blue}{return}}\ \textcolor{BrickRed}{(}\textbf{\textcolor{Black}{distsqr}}\textcolor{BrickRed}{(}pivot\textcolor{BrickRed}{,}\ self\textcolor{BrickRed}{)}\ \textcolor{BrickRed}{$<$}\ \textbf{\textcolor{Black}{distsqr}}\textcolor{BrickRed}{(}pivot\textcolor{BrickRed}{,}\ that\textcolor{BrickRed}{));} \\
\mbox{}\ \ \textcolor{Red}{\}} \\
\mbox{}\ \ \textbf{\textcolor{Blue}{return}}\ \textbf{\textcolor{Blue}{false}}\textcolor{BrickRed}{;} \\
\mbox{}\textcolor{Red}{\}} \\
\mbox{} \\
\mbox{}vector\textcolor{BrickRed}{$<$}point\textcolor{BrickRed}{$>$}\ \textbf{\textcolor{Black}{graham}}\textcolor{BrickRed}{(}vector\textcolor{BrickRed}{$<$}point\textcolor{BrickRed}{$>$}\ p\textcolor{BrickRed}{)}\textcolor{Red}{\{} \\
\mbox{}\ \ \textit{\textcolor{Brown}{//Metemos\ el\ más\ abajo\ más\ a\ la\ izquierda\ en\ la\ posición\ 0}} \\
\mbox{}\ \ \textbf{\textcolor{Blue}{for}}\ \textcolor{BrickRed}{(}\textcolor{ForestGreen}{int}\ i\textcolor{BrickRed}{=}\textcolor{Purple}{1}\textcolor{BrickRed}{;}\ i\textcolor{BrickRed}{$<$}p\textcolor{BrickRed}{.}\textbf{\textcolor{Black}{size}}\textcolor{BrickRed}{();}\ \textcolor{BrickRed}{++}i\textcolor{BrickRed}{)}\textcolor{Red}{\{} \\
\mbox{}\ \ \ \ \textbf{\textcolor{Blue}{if}}\ \textcolor{BrickRed}{(}p\textcolor{BrickRed}{[}i\textcolor{BrickRed}{].}y\ \textcolor{BrickRed}{$<$}\ p\textcolor{BrickRed}{[}\textcolor{Purple}{0}\textcolor{BrickRed}{].}y\ \textcolor{BrickRed}{$|$$|$}\ \textcolor{BrickRed}{(}p\textcolor{BrickRed}{[}i\textcolor{BrickRed}{].}y\ \textcolor{BrickRed}{==}\ p\textcolor{BrickRed}{[}\textcolor{Purple}{0}\textcolor{BrickRed}{].}y\ \textcolor{BrickRed}{\&\&}\ p\textcolor{BrickRed}{[}i\textcolor{BrickRed}{].}x\ \textcolor{BrickRed}{$<$}\ p\textcolor{BrickRed}{[}\textcolor{Purple}{0}\textcolor{BrickRed}{].}x\ \textcolor{BrickRed}{))} \\
\mbox{}\ \ \ \ \ \ \textbf{\textcolor{Black}{swap}}\textcolor{BrickRed}{(}p\textcolor{BrickRed}{[}\textcolor{Purple}{0}\textcolor{BrickRed}{],}\ p\textcolor{BrickRed}{[}i\textcolor{BrickRed}{]);} \\
\mbox{}\ \ \textcolor{Red}{\}} \\
\mbox{}\ \  \\
\mbox{}\ \ pivot\ \textcolor{BrickRed}{=}\ p\textcolor{BrickRed}{[}\textcolor{Purple}{0}\textcolor{BrickRed}{];} \\
\mbox{}\ \ \textbf{\textcolor{Black}{sort}}\textcolor{BrickRed}{(}p\textcolor{BrickRed}{.}\textbf{\textcolor{Black}{begin}}\textcolor{BrickRed}{(),}\ p\textcolor{BrickRed}{.}\textbf{\textcolor{Black}{end}}\textcolor{BrickRed}{(),}\ angleCmp\textcolor{BrickRed}{);} \\
\mbox{} \\
\mbox{}\ \ \textit{\textcolor{Brown}{//Ordenar\ por\ ángulo\ y\ eliminar\ repetidos.}} \\
\mbox{}\ \ \textit{\textcolor{Brown}{//Si\ varios\ puntos\ tienen\ el\ mismo\ angulo\ el\ más\ lejano\ queda\ después\ en\ la\ lista}} \\
\mbox{}\ \ vector\textcolor{BrickRed}{$<$}point\textcolor{BrickRed}{$>$}\ \textbf{\textcolor{Black}{chull}}\textcolor{BrickRed}{(}p\textcolor{BrickRed}{.}\textbf{\textcolor{Black}{begin}}\textcolor{BrickRed}{(),}\ p\textcolor{BrickRed}{.}\textbf{\textcolor{Black}{begin}}\textcolor{BrickRed}{()+}\textcolor{Purple}{3}\textcolor{BrickRed}{);} \\
\mbox{} \\
\mbox{}\ \ \textit{\textcolor{Brown}{//Ahora\ sí!!!}} \\
\mbox{}\ \ \textbf{\textcolor{Blue}{for}}\ \textcolor{BrickRed}{(}\textcolor{ForestGreen}{int}\ i\textcolor{BrickRed}{=}\textcolor{Purple}{3}\textcolor{BrickRed}{;}\ i\textcolor{BrickRed}{$<$}p\textcolor{BrickRed}{.}\textbf{\textcolor{Black}{size}}\textcolor{BrickRed}{();}\ \textcolor{BrickRed}{++}i\textcolor{BrickRed}{)}\textcolor{Red}{\{} \\
\mbox{}\ \ \ \ \textbf{\textcolor{Blue}{while}}\ \textcolor{BrickRed}{(}\ chull\textcolor{BrickRed}{.}\textbf{\textcolor{Black}{size}}\textcolor{BrickRed}{()}\ \textcolor{BrickRed}{$>$=}\ \textcolor{Purple}{2}\ \textcolor{BrickRed}{\&\&}\ \textbf{\textcolor{Black}{cross}}\textcolor{BrickRed}{(}chull\textcolor{BrickRed}{[}chull\textcolor{BrickRed}{.}\textbf{\textcolor{Black}{size}}\textcolor{BrickRed}{()-}\textcolor{Purple}{2}\textcolor{BrickRed}{],}\ chull\textcolor{BrickRed}{[}chull\textcolor{BrickRed}{.}\textbf{\textcolor{Black}{size}}\textcolor{BrickRed}{()-}\textcolor{Purple}{1}\textcolor{BrickRed}{],}\ p\textcolor{BrickRed}{[}i\textcolor{BrickRed}{])}\ \textcolor{BrickRed}{$<$=}\ \textcolor{Purple}{0}\textcolor{BrickRed}{)}\textcolor{Red}{\{} \\
\mbox{}\ \ \ \ \ \ chull\textcolor{BrickRed}{.}\textbf{\textcolor{Black}{erase}}\textcolor{BrickRed}{(}chull\textcolor{BrickRed}{.}\textbf{\textcolor{Black}{end}}\textcolor{BrickRed}{()}\ \textcolor{BrickRed}{-}\ \textcolor{Purple}{1}\textcolor{BrickRed}{);} \\
\mbox{}\ \ \ \ \textcolor{Red}{\}} \\
\mbox{}\ \ \ \ chull\textcolor{BrickRed}{.}\textbf{\textcolor{Black}{push$\_$back}}\textcolor{BrickRed}{(}p\textcolor{BrickRed}{[}i\textcolor{BrickRed}{]);} \\
\mbox{}\ \ \textcolor{Red}{\}} \\
\mbox{}\ \ \textit{\textcolor{Brown}{//chull\ contiene\ los\ puntos\ del\ convex\ hull\ ordenados\ anti-clockwise.}} \\
\mbox{}\ \ \textit{\textcolor{Brown}{//No\ contiene\ ningún\ punto\ repetido.}} \\
\mbox{}\ \ \textit{\textcolor{Brown}{//El\ primer\ punto\ no\ es\ el\ mismo\ que\ el\ último,\ i.e,\ la\ última\ arista}} \\
\mbox{}\ \ \textit{\textcolor{Brown}{//va\ de\ chull[chull.size()-1]\ a\ chull[0]}} \\
\mbox{}\ \ \textbf{\textcolor{Blue}{return}}\ chull\textcolor{BrickRed}{;} \\
\mbox{}\textcolor{Red}{\}} \\

} \normalfont\normalsize
%.tex

\subsection{Mínima distancia entre un punto y un segmento}
% Generator: GNU source-highlight, by Lorenzo Bettini, http://www.gnu.org/software/src-highlite

{\ttfamily \raggedright {
\noindent
\mbox{}\textbf{\textcolor{Blue}{struct}}\ point\textcolor{Red}{\{} \\
\mbox{}\ \ \textcolor{ForestGreen}{double}\ x\textcolor{BrickRed}{,}y\textcolor{BrickRed}{;} \\
\mbox{}\textcolor{Red}{\}}\textcolor{BrickRed}{;} \\
\mbox{} \\
\mbox{}\textbf{\textcolor{Blue}{inline}}\ \textcolor{ForestGreen}{double}\ \textbf{\textcolor{Black}{dist}}\textcolor{BrickRed}{(}\textbf{\textcolor{Blue}{const}}\ point\ \textcolor{BrickRed}{\&}a\textcolor{BrickRed}{,}\ \textbf{\textcolor{Blue}{const}}\ point\ \textcolor{BrickRed}{\&}b\textcolor{BrickRed}{)}\textcolor{Red}{\{} \\
\mbox{}\ \ \textbf{\textcolor{Blue}{return}}\ \textbf{\textcolor{Black}{sqrt}}\textcolor{BrickRed}{((}a\textcolor{BrickRed}{.}x\textcolor{BrickRed}{-}b\textcolor{BrickRed}{.}x\textcolor{BrickRed}{)*(}a\textcolor{BrickRed}{.}x\textcolor{BrickRed}{-}b\textcolor{BrickRed}{.}x\textcolor{BrickRed}{)}\ \textcolor{BrickRed}{+}\ \textcolor{BrickRed}{(}a\textcolor{BrickRed}{.}y\textcolor{BrickRed}{-}b\textcolor{BrickRed}{.}y\textcolor{BrickRed}{)*(}a\textcolor{BrickRed}{.}y\textcolor{BrickRed}{-}b\textcolor{BrickRed}{.}y\textcolor{BrickRed}{));} \\
\mbox{}\textcolor{Red}{\}} \\
\mbox{} \\
\mbox{}\textbf{\textcolor{Blue}{inline}}\ \textcolor{ForestGreen}{double}\ \textbf{\textcolor{Black}{distsqr}}\textcolor{BrickRed}{(}\textbf{\textcolor{Blue}{const}}\ point\ \textcolor{BrickRed}{\&}a\textcolor{BrickRed}{,}\ \textbf{\textcolor{Blue}{const}}\ point\ \textcolor{BrickRed}{\&}b\textcolor{BrickRed}{)}\textcolor{Red}{\{} \\
\mbox{}\ \ \textbf{\textcolor{Blue}{return}}\ \textcolor{BrickRed}{(}a\textcolor{BrickRed}{.}x\textcolor{BrickRed}{-}b\textcolor{BrickRed}{.}x\textcolor{BrickRed}{)*(}a\textcolor{BrickRed}{.}x\textcolor{BrickRed}{-}b\textcolor{BrickRed}{.}x\textcolor{BrickRed}{)}\ \textcolor{BrickRed}{+}\ \textcolor{BrickRed}{(}a\textcolor{BrickRed}{.}y\textcolor{BrickRed}{-}b\textcolor{BrickRed}{.}y\textcolor{BrickRed}{)*(}a\textcolor{BrickRed}{.}y\textcolor{BrickRed}{-}b\textcolor{BrickRed}{.}y\textcolor{BrickRed}{);} \\
\mbox{}\textcolor{Red}{\}} \\
\mbox{} \\
\mbox{}\textit{\textcolor{Brown}{/*}} \\
\mbox{}\textit{\textcolor{Brown}{\ \ Returns\ the\ closest\ distance\ between\ point\ pnt\ and\ the\ segment\ that\ goes\ from\ point\ a\ to\ b}} \\
\mbox{}\textit{\textcolor{Brown}{\ \ Idea\ by:\ }}\underline{\texttt{\textcolor{Blue}{http://local.wasp.uwa.edu.au/}}}\textit{\textcolor{Brown}{\textasciitilde{}pbourke/geometry/pointline/}} \\
\mbox{}\textit{\textcolor{Brown}{\ */}} \\
\mbox{}\textcolor{ForestGreen}{double}\ \textbf{\textcolor{Black}{distance$\_$point$\_$to$\_$segment}}\textcolor{BrickRed}{(}\textbf{\textcolor{Blue}{const}}\ point\ \textcolor{BrickRed}{\&}a\textcolor{BrickRed}{,}\ \textbf{\textcolor{Blue}{const}}\ point\ \textcolor{BrickRed}{\&}b\textcolor{BrickRed}{,}\ \textbf{\textcolor{Blue}{const}}\ point\ \textcolor{BrickRed}{\&}pnt\textcolor{BrickRed}{)}\textcolor{Red}{\{} \\
\mbox{}\ \ \textcolor{ForestGreen}{double}\ u\ \textcolor{BrickRed}{=}\ \textcolor{BrickRed}{((}pnt\textcolor{BrickRed}{.}x\ \textcolor{BrickRed}{-}\ a\textcolor{BrickRed}{.}x\textcolor{BrickRed}{)*(}b\textcolor{BrickRed}{.}x\ \textcolor{BrickRed}{-}\ a\textcolor{BrickRed}{.}x\textcolor{BrickRed}{)}\ \textcolor{BrickRed}{+}\ \textcolor{BrickRed}{(}pnt\textcolor{BrickRed}{.}y\ \textcolor{BrickRed}{-}\ a\textcolor{BrickRed}{.}y\textcolor{BrickRed}{)*(}b\textcolor{BrickRed}{.}y\ \textcolor{BrickRed}{-}\ a\textcolor{BrickRed}{.}y\textcolor{BrickRed}{))}\ \textcolor{BrickRed}{/}\ \textbf{\textcolor{Black}{distsqr}}\textcolor{BrickRed}{(}a\textcolor{BrickRed}{,}\ b\textcolor{BrickRed}{);} \\
\mbox{}\ \ point\ intersection\textcolor{BrickRed}{;} \\
\mbox{}\ \ intersection\textcolor{BrickRed}{.}x\ \textcolor{BrickRed}{=}\ a\textcolor{BrickRed}{.}x\ \textcolor{BrickRed}{+}\ u\textcolor{BrickRed}{*(}b\textcolor{BrickRed}{.}x\ \textcolor{BrickRed}{-}\ a\textcolor{BrickRed}{.}x\textcolor{BrickRed}{);} \\
\mbox{}\ \ intersection\textcolor{BrickRed}{.}y\ \textcolor{BrickRed}{=}\ a\textcolor{BrickRed}{.}y\ \textcolor{BrickRed}{+}\ u\textcolor{BrickRed}{*(}b\textcolor{BrickRed}{.}y\ \textcolor{BrickRed}{-}\ a\textcolor{BrickRed}{.}y\textcolor{BrickRed}{);} \\
\mbox{}\ \ \textbf{\textcolor{Blue}{if}}\ \textcolor{BrickRed}{(}u\ \textcolor{BrickRed}{$<$}\ \textcolor{Purple}{0.0}\ \textcolor{BrickRed}{$|$$|$}\ u\ \textcolor{BrickRed}{$>$}\ \textcolor{Purple}{1.0}\textcolor{BrickRed}{)}\textcolor{Red}{\{} \\
\mbox{}\ \ \ \ \textbf{\textcolor{Blue}{return}}\ \textbf{\textcolor{Black}{min}}\textcolor{BrickRed}{(}\textbf{\textcolor{Black}{dist}}\textcolor{BrickRed}{(}a\textcolor{BrickRed}{,}\ pnt\textcolor{BrickRed}{),}\ \textbf{\textcolor{Black}{dist}}\textcolor{BrickRed}{(}b\textcolor{BrickRed}{,}\ pnt\textcolor{BrickRed}{));} \\
\mbox{}\ \ \textcolor{Red}{\}} \\
\mbox{}\ \ \textbf{\textcolor{Blue}{return}}\ \textbf{\textcolor{Black}{dist}}\textcolor{BrickRed}{(}pnt\textcolor{BrickRed}{,}\ intersection\textcolor{BrickRed}{);} \\
\mbox{}\textcolor{Red}{\}} \\

} \normalfont\normalsize
%.tex

\subsection{Mínima distancia entre un punto y una recta}
% Generator: GNU source-highlight, by Lorenzo Bettini, http://www.gnu.org/software/src-highlite

{\ttfamily \raggedright {
\noindent
\mbox{}\textit{\textcolor{Brown}{/*}} \\
\mbox{}\textit{\textcolor{Brown}{\ \ Returns\ the\ closest\ distance\ between\ point\ pnt\ and\ the\ line\ that\ passes\ through\ points\ a\ and\ b}} \\
\mbox{}\textit{\textcolor{Brown}{\ \ Idea\ by:\ }}\underline{\texttt{\textcolor{Blue}{http://local.wasp.uwa.edu.au/}}}\textit{\textcolor{Brown}{\textasciitilde{}pbourke/geometry/pointline/}} \\
\mbox{}\textit{\textcolor{Brown}{\ */}} \\
\mbox{}\textcolor{ForestGreen}{double}\ \textbf{\textcolor{Black}{distance$\_$point$\_$to$\_$line}}\textcolor{BrickRed}{(}\textbf{\textcolor{Blue}{const}}\ point\ \textcolor{BrickRed}{\&}a\textcolor{BrickRed}{,}\ \textbf{\textcolor{Blue}{const}}\ point\ \textcolor{BrickRed}{\&}b\textcolor{BrickRed}{,}\ \textbf{\textcolor{Blue}{const}}\ point\ \textcolor{BrickRed}{\&}pnt\textcolor{BrickRed}{)}\textcolor{Red}{\{} \\
\mbox{}\ \ \textcolor{ForestGreen}{double}\ u\ \textcolor{BrickRed}{=}\ \textcolor{BrickRed}{((}pnt\textcolor{BrickRed}{.}x\ \textcolor{BrickRed}{-}\ a\textcolor{BrickRed}{.}x\textcolor{BrickRed}{)*(}b\textcolor{BrickRed}{.}x\ \textcolor{BrickRed}{-}\ a\textcolor{BrickRed}{.}x\textcolor{BrickRed}{)}\ \textcolor{BrickRed}{+}\ \textcolor{BrickRed}{(}pnt\textcolor{BrickRed}{.}y\ \textcolor{BrickRed}{-}\ a\textcolor{BrickRed}{.}y\textcolor{BrickRed}{)*(}b\textcolor{BrickRed}{.}y\ \textcolor{BrickRed}{-}\ a\textcolor{BrickRed}{.}y\textcolor{BrickRed}{))}\ \textcolor{BrickRed}{/}\ \textbf{\textcolor{Black}{distsqr}}\textcolor{BrickRed}{(}a\textcolor{BrickRed}{,}\ b\textcolor{BrickRed}{);} \\
\mbox{}\ \ point\ intersection\textcolor{BrickRed}{;} \\
\mbox{}\ \ intersection\textcolor{BrickRed}{.}x\ \textcolor{BrickRed}{=}\ a\textcolor{BrickRed}{.}x\ \textcolor{BrickRed}{+}\ u\textcolor{BrickRed}{*(}b\textcolor{BrickRed}{.}x\ \textcolor{BrickRed}{-}\ a\textcolor{BrickRed}{.}x\textcolor{BrickRed}{);} \\
\mbox{}\ \ intersection\textcolor{BrickRed}{.}y\ \textcolor{BrickRed}{=}\ a\textcolor{BrickRed}{.}y\ \textcolor{BrickRed}{+}\ u\textcolor{BrickRed}{*(}b\textcolor{BrickRed}{.}y\ \textcolor{BrickRed}{-}\ a\textcolor{BrickRed}{.}y\textcolor{BrickRed}{);} \\
\mbox{}\ \ \textbf{\textcolor{Blue}{return}}\ \textbf{\textcolor{Black}{dist}}\textcolor{BrickRed}{(}pnt\textcolor{BrickRed}{,}\ intersection\textcolor{BrickRed}{);} \\
\mbox{}\textcolor{Red}{\}} \\

} \normalfont\normalsize
%.tex

\subsection{Determinar si un polígono es convexo}
% Generator: GNU source-highlight, by Lorenzo Bettini, http://www.gnu.org/software/src-highlite

{\ttfamily \raggedright {
\noindent
\mbox{}\textit{\textcolor{Brown}{/*}} \\
\mbox{}\textit{\textcolor{Brown}{\ \ Returns\ positive\ if\ a-b-c\ make\ a\ left\ turn.}} \\
\mbox{}\textit{\textcolor{Brown}{\ \ Returns\ negative\ if\ a-b-c\ make\ a\ right\ turn.}} \\
\mbox{}\textit{\textcolor{Brown}{\ \ Returns\ 0.0\ if\ a-b-c\ are\ colineal.}} \\
\mbox{}\textit{\textcolor{Brown}{\ */}} \\
\mbox{}\textcolor{ForestGreen}{double}\ \textbf{\textcolor{Black}{turn}}\textcolor{BrickRed}{(}\textbf{\textcolor{Blue}{const}}\ point\ \textcolor{BrickRed}{\&}a\textcolor{BrickRed}{,}\ \textbf{\textcolor{Blue}{const}}\ point\ \textcolor{BrickRed}{\&}b\textcolor{BrickRed}{,}\ \textbf{\textcolor{Blue}{const}}\ point\ \textcolor{BrickRed}{\&}c\textcolor{BrickRed}{)}\textcolor{Red}{\{} \\
\mbox{}\ \ \textcolor{ForestGreen}{double}\ z\ \textcolor{BrickRed}{=}\ \textcolor{BrickRed}{(}b\textcolor{BrickRed}{.}x\ \textcolor{BrickRed}{-}\ a\textcolor{BrickRed}{.}x\textcolor{BrickRed}{)*(}c\textcolor{BrickRed}{.}y\ \textcolor{BrickRed}{-}\ a\textcolor{BrickRed}{.}y\textcolor{BrickRed}{)}\ \textcolor{BrickRed}{-}\ \textcolor{BrickRed}{(}b\textcolor{BrickRed}{.}y\ \textcolor{BrickRed}{-}\ a\textcolor{BrickRed}{.}y\textcolor{BrickRed}{)*(}c\textcolor{BrickRed}{.}x\ \textcolor{BrickRed}{-}\ a\textcolor{BrickRed}{.}x\textcolor{BrickRed}{);} \\
\mbox{}\ \ \textbf{\textcolor{Blue}{if}}\ \textcolor{BrickRed}{(}\textbf{\textcolor{Black}{fabs}}\textcolor{BrickRed}{(}z\textcolor{BrickRed}{)}\ \textcolor{BrickRed}{$<$}\ \textcolor{Purple}{1e-9}\textcolor{BrickRed}{)}\ \textbf{\textcolor{Blue}{return}}\ \textcolor{Purple}{0.0}\textcolor{BrickRed}{;} \\
\mbox{}\ \ \textbf{\textcolor{Blue}{return}}\ z\textcolor{BrickRed}{;} \\
\mbox{}\textcolor{Red}{\}} \\
\mbox{} \\
\mbox{}\textit{\textcolor{Brown}{/*}} \\
\mbox{}\textit{\textcolor{Brown}{\ \ Returns\ true\ if\ polygon\ p\ is\ convex.}} \\
\mbox{}\textit{\textcolor{Brown}{\ \ False\ if\ it's\ concave\ or\ it\ can't\ be\ determined}} \\
\mbox{}\textit{\textcolor{Brown}{\ \ (For\ example,\ if\ all\ points\ are\ colineal\ we\ can't\ }} \\
\mbox{}\textit{\textcolor{Brown}{\ \ make\ a\ choice).}} \\
\mbox{}\textit{\textcolor{Brown}{\ */}} \\
\mbox{}\textcolor{ForestGreen}{bool}\ \textbf{\textcolor{Black}{isConvexPolygon}}\textcolor{BrickRed}{(}\textbf{\textcolor{Blue}{const}}\ vector\textcolor{BrickRed}{$<$}point\textcolor{BrickRed}{$>$}\ \textcolor{BrickRed}{\&}p\textcolor{BrickRed}{)}\textcolor{Red}{\{} \\
\mbox{}\ \ \textcolor{ForestGreen}{int}\ mask\ \textcolor{BrickRed}{=}\ \textcolor{Purple}{0}\textcolor{BrickRed}{;} \\
\mbox{}\ \ \textcolor{ForestGreen}{int}\ n\ \textcolor{BrickRed}{=}\ p\textcolor{BrickRed}{.}\textbf{\textcolor{Black}{size}}\textcolor{BrickRed}{();} \\
\mbox{}\ \ \textbf{\textcolor{Blue}{for}}\ \textcolor{BrickRed}{(}\textcolor{ForestGreen}{int}\ i\textcolor{BrickRed}{=}\textcolor{Purple}{0}\textcolor{BrickRed}{;}\ i\textcolor{BrickRed}{$<$}n\textcolor{BrickRed}{;}\ \textcolor{BrickRed}{++}i\textcolor{BrickRed}{)}\textcolor{Red}{\{} \\
\mbox{}\ \ \ \ \textcolor{ForestGreen}{int}\ j\textcolor{BrickRed}{=(}i\textcolor{BrickRed}{+}\textcolor{Purple}{1}\textcolor{BrickRed}{)\%}n\textcolor{BrickRed}{;} \\
\mbox{}\ \ \ \ \textcolor{ForestGreen}{int}\ k\textcolor{BrickRed}{=(}i\textcolor{BrickRed}{+}\textcolor{Purple}{2}\textcolor{BrickRed}{)\%}n\textcolor{BrickRed}{;} \\
\mbox{}\ \ \ \ \textcolor{ForestGreen}{double}\ z\ \textcolor{BrickRed}{=}\ \textbf{\textcolor{Black}{turn}}\textcolor{BrickRed}{(}p\textcolor{BrickRed}{[}i\textcolor{BrickRed}{],}\ p\textcolor{BrickRed}{[}j\textcolor{BrickRed}{],}\ p\textcolor{BrickRed}{[}k\textcolor{BrickRed}{]);} \\
\mbox{}\ \ \ \ \textbf{\textcolor{Blue}{if}}\ \textcolor{BrickRed}{(}z\ \textcolor{BrickRed}{$<$}\ \textcolor{Purple}{0.0}\textcolor{BrickRed}{)}\textcolor{Red}{\{} \\
\mbox{}\ \ \ \ \ \ mask\ \textcolor{BrickRed}{$|$=}\ \textcolor{Purple}{1}\textcolor{BrickRed}{;} \\
\mbox{}\ \ \ \ \textcolor{Red}{\}}\textbf{\textcolor{Blue}{else}}\ \textbf{\textcolor{Blue}{if}}\ \textcolor{BrickRed}{(}z\ \textcolor{BrickRed}{$>$}\ \textcolor{Purple}{0.0}\textcolor{BrickRed}{)}\textcolor{Red}{\{} \\
\mbox{}\ \ \ \ \ \ mask\ \textcolor{BrickRed}{$|$=}\ \textcolor{Purple}{2}\textcolor{BrickRed}{;} \\
\mbox{}\ \ \ \ \textcolor{Red}{\}} \\
\mbox{}\ \ \ \ \textbf{\textcolor{Blue}{if}}\ \textcolor{BrickRed}{(}mask\ \textcolor{BrickRed}{==}\ \textcolor{Purple}{3}\textcolor{BrickRed}{)}\ \textbf{\textcolor{Blue}{return}}\ \textbf{\textcolor{Blue}{false}}\textcolor{BrickRed}{;} \\
\mbox{}\ \ \textcolor{Red}{\}} \\
\mbox{}\ \ \textbf{\textcolor{Blue}{return}}\ mask\ \textcolor{BrickRed}{!=}\ \textcolor{Purple}{0}\textcolor{BrickRed}{;} \\
\mbox{}\textcolor{Red}{\}} \\

} \normalfont\normalsize
%.tex

\end{document}
